%\chapter*{}
%\pagenumbering{Roman}

\begin{acknowledgements}
Gracias a:\\

El Dr. Luis Benet Fernández, quién me guió en la elaboración de esta tesis. Por su tiempo, paciencia y experiencia, que me ayudaron a completar mi educación como universitaria. Por contagiarme de nuevo el entusiasmo por la ciencia en general, la pasión por la física y la enseñanza de la misma. \\

Gracias al Dr. Christof Jung, al Dr. David P. Sanders, al Dr. Renato Calleja y al Dr. Ricardo A. Solórzano, que con sus valiosos comentarios nutrieron el contenido de este trabajo. Algunas de las secciones se crearon a partir de sus preguntas o sugerencias. \\

Al Dr. Marcos Ley Koo quién me mostró una cara de la física que no se aprende en las aulas, esa que se descubre en todos lados y que te hace un físico fuera de la academia. Que con sus pláticas me llenaba la mente de preguntas que muchas veces me quitaron el sueño. Quién siempre me sorprendía con una nueva forma de entender la termodinámica, a partir de la mecánica o del electromagnetismo y hasta de la geometría. Por compartir sus locas ideas conmigo. \\

A mi familia, sobre todo a mis padres, que con su esfuerzo, valentía y trabajo lograron que cumpliera con mis estudios universitarios. Por la educación que me dieron, con la que aprendí a disfrutar las cosas más simplemente bellas y con la que me he guiado hasta este momento. Por enseñarme a ser crítica, buscar, preguntar y observar, mostrándome que todo eso no se obtiene de un título universitario y tampoco va acompañado de traje, que se puede encontrar en una cabina de trailer o en un puesto ambulante. A mis hermanos por ser mi apoyo constante, mis compañeros eternos y mis cómplices.\\

A \textquotedblleft el lugar\textquotedblright, a mis amigos, por compartir conmigo la vida escolar, por hacerla divertida y memorable. Por intercalar horas de estudio con interminables charlas, risas, con juegos de mesa de 6 horas o con domingos de series.\\

A quién vio esta tesis desde que apenas tenía el nombre y que estuvo presente en todo el proceso, quién me escuchó maldecir cuando el código no funcionaba y me vio pelear con el módulo de álgebra lineal muchas noches. Por ser mi confidente y enseñarme que \textquotedblleft una pasión es una pasión\textquotedblright.\\

Finalmente, gracias al Programa de Apoyo a Proyectos de Investigación e Innovación Tecnológica(PAPIIT) de la UNAM con clave IG100616. Agradezco a la DGPA-UNAM la beca recibida.

\end{acknowledgements}




