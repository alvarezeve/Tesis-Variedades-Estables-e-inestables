
\chapter{Resumen y perspectivas}
Una característica importante a estudiar en los mapeos en general son las variedades estables e inestables asociadas a puntos fijos. En el caso de los mapeos de dos dimensiones resulta manejable, hasta cierto punto, encontrarlas de manera analítica usando el método de parametrización. Como vimos el método tiene como núcleo de desarrollo la ecuación de invariancia y la linearización del sistema alrededor de un punto fijo. Sin embargo decir manejable en términos matemáticos y físicos no resulta suficiente si lo que necesitamos es estudiar propiedades de los sistemas a partir de las variedades o el comportamiento de puntos fijos. Por ello es que la implementación del método resultaba llamativa. Tener un módulo escrito en software libre que calcula las variedades asociadas a puntos fijos hiperbólicos va más allá de generar las relaciones de recurrencia en casos particulares. El método automatizado es capaz de generar las parametrizaciones de las variedades alrededor de un punto fijo hiperbólico conocido, en cualquier mapeo Hamiltoniano de dos dimensiones. La idea detrás de la automatización se basa en que la computadora haga las veces de la recurrencia en lugar de calcularlas de manera analítica. Esto de ninguna manera modifica el modo del método. Todo esto da como resultado un método semianalítico con el cual tenemos las variedades de manera polinomial. \\

Dado que es un método parte analítico y parte computacional que involucra series de Taylor es crucial decir de alguna manera qué tan confiable es la parametrización que resulta. Por ello se incluyeron tres formas de evaluar el comportamiento de las variedades, tales involucran al error de la solución en serie de la ecuación de invarianza y el estudio de la convergencia mediante dos métodos. Conocer que tanto podemos afirmar sobre el comportamiento de las variedades depende de estas tres funciones.\\

Como pudimos notar en el capítulo anterior en los tres ejemplos que se presentan observamos comportamientos muy variados, el mapeo estándar tiene una función elemental además de estar acotado en el espacio fase, en el mapeo exponencial también se tiene una función elemental mientras que en el de Hénon se tiene un polinomio. En el mapeo estándar se buscaba mostrar un ejemplo bastante conocido, en el de Henón mostrar el cruce de variedades mientras que en el exponencial que tantos cruces (o tentáculos) se pueden observar. Puede pasar que para otros mapeos se busque estudiar un comportamiento particular de las variedades. Por ello consideramos que el método se puede explotar en varias direcciones que se adapten a cada mapeo en particular. En el caso por ejemplo de observar la dependencia de los tentáculos de las variedades con respecto a los parámetros del mapeo tendrá que ver con qué tan grande podemos hacer el orden del polinomio y qué tan lejos llegamos con un error relativamente pequeño. Además como mostramos hay formas de mejorar la parametrización usando el mapeo inverso. \\

En los tres mapeos mostramos que es posible calcular las intersecciones de las variedades. Tal resultado nos muestra una ventana hacía la \emph{demostración} de que hay un corte entre ambas y con ello a resultados más importantes y áreas más amplias como el estudio de bifurcaciones y caos topológco. Esta idea resulta de que al conocer las variedades de manera analítica podemos buscar el cero de las funciones correctas para encontrar un cruce. Con el orden suficiente se puede aislar el rectángulo fundamental, en mapeos abiertos, encontrar ahí los cruces que representan todos los cruces de las variedades. Para hacer esto se debe trabajar de manera más rigurosa con el método de parametrización, es decir implementar el método ahora usando el álgebra de conjuntos, para así tener un polinomio con coeficientes garantizados.   \\

En el proceso de este trabajo surgieron varias dudas, una de ellas se formuló en principio como sigue: ¿es posible reparametrizar a partir de un cierto punto la variedad?. Por ejemplo en los casos en los que se tienen puntos homoclínicos se quisiera comenzar una nueva parametrización a partir del mismo. Esto permitiría construir las variedades por pedazos en los que se tenga un error mínimo y por tanto tener una mayor parte de la misma. Otra de las preguntas que surguieron tiene que ver con los polinomios de Chebyshev. Los polinomios de Taylor no tienen una dirección preferencial, mientras que los de Chebyshev sí la tienen, el tener una dirección preferencial puede servir para mejorar el error y conseguir un polinomio que describa mejor la variedad a valores grandes del parámetro.\\


Todos los comentarios anteriores intentan mostrar que hay aún preguntas sin responder y sobre todo curiosidad por hacer más con el método de parametrización tanto de manera computacional como de manera analítica. 

