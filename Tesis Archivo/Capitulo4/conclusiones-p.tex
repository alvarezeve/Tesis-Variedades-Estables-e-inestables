
\chapter{Panorama}
Una característica importante a estudiar en los mapeos en general son las variedades estables e inestables asociadas a puntos fijos. En el caso de los mapeos de dos dimensiones resulta manejable, hasta cierto punto, encontrarlas de manera analítica usando el método de parametrización. Como vimos el método tiene como núcleo de desarrollo la ecuación de invariancia y la linearización del sistema al rededor de un punto fijo. Sin embargo decir manejable en términos matemáticos y físicos no resulta suficiente si lo que necesitamos es estudiar propiedades de los sistemas a partir de las variedades o el comportamiento de puntos fijos. Por ello es que la implementación del método resultaba llamativa. Tener un módulo escrito en software libre que calcula las variedades asociadas a puntos fijos hiperbólicos va más allá de generar las relaciones de recurrencia en casos particulares. El método automatizado es capaz de generar las parametrizaciones de las variedades al rededor de un punto fijo conocido, en cualquier mapeo de dos dimensiones que sea simpléctico. La idea detrás de la automatización se basa en que la computadora haga las veces de la recurrencia en lugar de calcularlas de manera analítica. Esto de ninguna manera modifica el modo del método. Todo esto da como resultado un método semianalítico con el cual tenemos las variedades de manera polinómica. \\

Dado que es un método parte analítico y parte computacional que involucra series de Taylor es crucial decir de alguna manera que tan confiable es la parametrización que resulta. Por ello se incluyeron tres formas de evaluar el comportamiento de las variedades, tales involucran al error y la convergencia mediante dos métodos. Conocer que tanto podemos afirmar sobre el comportamiento de las variedades depende de estas tres funciones.\\

Como pudimos notar en el capítulo anterior en los tres ejemplos que se presentan observamos comportamientos muy variados. En el mapeo estándar se buscaba mostrar un ejemplo bastante conocido, en el de henón mostrar el cruce de variedades mientras que en el exponencial que tantos cruces ( o tentáculos) se pueden observar. Puede pasar que para otros mapeos se busque estudiar un comportamiento particular de las variedades. Por ello consideramos que el método se puede explotar en varias direcciones que se adapten a cada mapeo en particular. En el caso por ejemplo de observar la dependencia de los tentáculos de las variedades con respecto a los parámetros del mapeo tendrá que ver con que tan grande podemos hacer el orden del polinomio  y que tan lejos llegamos con un error relativamente pequeño. Además como mostramos hay formas de mejorar la parametrización usando el mapeo inverso. \\

En el mapeo de Hénon mostramos que es posible calcular las intersecciones de las variedades. Tal resultado nos muestra una ventana hacía la \emph{demostración} de que hay un corte entre ambas y con ello a resultados más importantes y áreas más amplias como el estudio de bifurcaciones y caos. Esta idea resulta de que al conocer las variedades de manera analítica podemos buscar el cero de las funciones correctas para encontrar un cruce. Aún si no se conoce si en el sistema las variedades se cruzan en uno o más puntos, con el orden suficiente se podría asegurar. Para ello habrá que trabajar de manera analítica primero con las parametrizaciones y asociar un error al cálculo de la intersección . \\

En el proceso de este trabajo surgió una duda que se formuló en principio como sigue :¿es posible seleccionar una dirección preferente en la parametrización?. Es decir, es claro que en términos del vector propio hay una dirección en la que se desarrolla la variedad, en algunos casos resulta que la variedad puede pasar cerca de algún otro punto fijo del sistema, ¿cómo seleccionar una dirección preferente para el desarrollo del polinomio?. Resulta que los polinomios de Taylor no tienen una dirección preferencial , mientras que los polinomios de Chebyshev sí. En términos burdos uno puede usar polinomios diferentes que tengan preferencias topológicas con respecto a los de Taylor. Eso puede ayudar a que la parametrización se encamine a la dirección necesaria desde un inicio. Se piensa que el método de parametrización se puede desarrollar usando tales polinomios en lugar de usar Taylor. Sería una manera de parametrizar más enfocada. \\
Todos los comentarios anteriores intentan mostrar que hay aún preguntas sin responder del todo y sobre todo curiosidad por hacer más con el método de parametrización tanto de manera computacional como de manera analítica. Sin duda puede que dentro del estudio de variedades en mapeos de dos dimensiones haya muchas más cosas que se estén estudiando con diferentes y variados propósitos. 


