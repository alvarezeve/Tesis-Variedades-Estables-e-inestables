
\chapter{Resumen y perspectivas}
Una característica importante a estudiar en los mapeos en general son las variedades estables e inestables asociadas a puntos fijos inestables. En el caso de los mapeos de dos dimensiones resulta manejable, hasta cierto punto, encontrarlas de manera semi-analítica usando el método de parametrización. Como se vio el método tiene como núcleo de desarrollo la ecuación de invariancia y la linealización del sistema alrededor de un punto fijo. Sin embargo decir manejable en términos matemáticos y físicos no resulta suficiente si lo que se necesita es estudiar propiedades de los sistemas a partir de las variedades o el comportamiento de puntos fijos. Por ello es que la implementación del método resulta llamativa. Tener un módulo escrito en software libre que calcula las variedades asociadas a puntos fijos hiperbólicos va más allá de generar las relaciones de recurrencia en casos particulares. El método automatizado es capaz de generar las parametrizaciones de las variedades alrededor de un punto fijo hiperbólico conocido, en cualquier mapeo Hamiltoniano de dos dimensiones. La idea detrás de la automatización se basa en que la computadora haga las veces de la recurrencia en lugar de calcularlas de manera analítica. Esto de ninguna manera modifica el modo del método, dando como resultado un método semianalítico con el cual se obtienen las variedades de manera polinomial. \\

Dado que es un método en parte analítico y en parte computacional que involucra series de Taylor; es crucial decir de alguna manera qué tan confiable es la parametrización. Por ello se incluyeron tres formas de evaluar el comportamiento de las variedades, involucrando al error de la solución en serie de la ecuación de invariancia y el estudio de la convergencia mediante dos métodos. Conocer qué tanto es posible afirmar sobre el comportamiento de las variedades depende de estas tres funciones.\\

Los tres ejemplos presentados en el capítulo anterior, presentan comportamientos muy variados: el mapeo estándar tiene una función exponencial además de estar acotado en el espacio fase, mientras en el mapeo exponencial también se tiene una función elemental, pero no está acotado, en el caso de Hénon se tiene un polinomio y no es acotado. El mapeo estándar, por ser bastante conocido, sirvió como referencia para programar el método; el de Hénon por su parte se pensó que sería más fácil de parametrizar, puesto que tiene forma polinomial, lo cual concuerda pues se pudo llegar a valores grandes en el parámetro. En el mapeo exponencial se buscó un reto para el método, pues al ser una función exponencial es más sutil aproximarla por polinomios. Fue notable que en el mapeo de Hénon se pudo observar más sobre las variedades que en los otros dos casos, mientras que el más complicado fue el exponencial. Se puede decir que aquellos mapeos que tengan formas polinomiales serán más dóciles de tratar por el método, debido a que el método consiste en aproximar las variedades por un polinomio.\\

Al explotar el método en los tres mapeos se encontró, con ayuda de los métodos para raíces, los cruces entre variedades para los tres casos, con lo cual se puede saber si hay conexiones homoclínicas o heteroclínicas. Usando aritmética de intervalos se puede tener un método numérico que garantice (matemáticamente hablando), la existencia de puntos homoclínicos o heteroclínicos. En este caso no es estricto el cálculo, pues los coeficientes de la parametrización no son calculados de manera rigurosa con aritmética de intervalos. Esta idea sería una ventana hacia resultados más importantes y amplios, como son el estudio de bifurcaciones y caos topológico.\\

Una característica importante a estudiar también es el comportamiento de los tentáculos formados por las variedades en términos de los parámetros de los mapeos. Como aparece en la sección \ref{SeccionRectanguloFundamental} se pueden obtener tentáculos a partir de iterar la parametrización mediante el mapeo. Así se puede calcular un polinomio de orden no tan grande, dependiendo del mapeo, e iterando hasta llegar a estructuras difíciles de alcanzar sólo con la parametrización.\\

Una de las dudas que surgió durante el proceso de este trabajo se formuló en principio como sigue: ¿es posible reparametrizar a partir de un cierto punto la variedad?. Por ejemplo en los casos en los que se tienen puntos homoclínicos, se quisiera comenzar una nueva parametrización a partir del mismo. Esto permitiría construir las variedades por pedazos en los que se tenga un error más controlable y por tanto tener una mayor parte de la misma. Otra de las preguntas que surgieron tiene que ver con los polinomios de Chebyshev. Los polinomios de Taylor no tienen una dirección preferencial en el plano complejo, mientras que los de Chebyshev sí la tienen; el tener una dirección preferencial puede servir para mejorar el error y conseguir un polinomio que describa mejor la variedad a valores grandes del parámetro. 
