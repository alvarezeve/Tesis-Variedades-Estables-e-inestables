
\chapter{Introducción}
En el análisis de sis\-te\-mas di\-ná\-mi\-cos es importante desarrollar aspectos que describan cualitativa y cuantitativamente el comportamiento de un sistema, esto ya que muchos de ellos tienen una dinámica rica en el sentido matemático y físico. Conocer por ejemplo los puntos fijos, las órbitas periódicas, el diagrama de bifurcación, entre otras características es una manera de poder decir mucho sobre el sistema en general. Entre esas características está el estudio de \textit{las variedades estables e inestables} alrededor de puntos fijos o puntos de periodo $n$. La importancia de las variedades radica en que mediante éstas se puede conocer el comportamiento del sistema dinámico en las vecindades del punto periódico; además para algunos casos la intersección de las mismas lleva a resultados interesantes sobre caos. Se puede por ejemplo estudiar el problema de los tres cuerpos mediante el estudio de sus variedades cerca del punto fijo, haciendo una linealización del problema. \citep{Meyer}\\


El estudio analítico de las variedades cerca de puntos periódicos (puntos fijos y de periodo igual o mayor a dos) se ha complementado con el numérico. Dentro de los métodos semianalíticos se encuentra el de parametrización \citep{Haro}. El método de pa\-ra\-me\-tri\-za\-ción, dicho de manera simple, consiste en aproximar mediante series de Taylor las variedades alrededor de puntos periódicos usando que las variedades son solución a la ecuación de invariancia. Los coeficientes de los polinomios de Taylor se van calculando de manera recursiva. El método se describe en el trabajo de J.D. Mireles \citep{Mireles}, quién  lo aplicó de manera particular al mapeo estándar, describiendo muy claramente cómo se obtienen las relaciones de recurrencia; su trabajo es la motivación de esta tesis.\\



El objetivo de este trabajo es ir más allá de implementar el método de pa\-ra\-me\-tri\-za\-ción para el mapeo estándar. Siguiendo las notas mencionadas se automatizó el método de manera computacional, primero para el mapeo estándar y luego se hizo de manera general para cualquier mapeo de dos dimensiones. Con el método implementado se obtienen las variedades estable e inestable alrededor de puntos fijos hi\-per\-bó\-li\-cos, pa\-ra\-me\-tri\-za\-das por medio de un polinomio de orden $n$.
Teniendo las variedades se hizo un análisis de las intersecciones entre ellas y de cómo explotar el método para mejorar el error.
 
Este escrito se divide en tres partes. En el primer capítulo se presenta la teoría de los sistemas dinámicos Hamiltonianos que se usará a lo largo del método, junto con el método de parametrización. Primero introduciendo qué es un sistema dinámico, junto con las definiciones de puntos fijos y órbitas periódicas, llegando a la teoría detrás de la ecuación de invariancia. Las matemáticas que se utilizan en el proceso de parametrización están al alcance de un estudiante de licenciatura de Física o Ma\-te\-má\-ti\-cas; sin embargo la teoría detrás del funcionamiento del método es un tanto más elevada de nivel, por lo que sólo se mencionan las herramientas más fundamentales.  \\


El segundo capítulo es una descripción breve de cómo J.D Mireles aplica el método para un caso particular, para posteriormente explicar cómo se procedió a implementar el método. En esta parte también se incluye el análisis del error y de la convergencia de las soluciones. El tercer y último capítulo consiste en analizar el mapeo estándar para reproducir algunos de los resultados presentados en \cite{Mireles}. También se analizarán otros mapeos como el de Hénon y uno que representa un objeto pateado, al que llamamos exponencial \cite{Jung}. Por último se presenta una breve perspectiva en la que se habla de algunas ideas con las que se podría seguir trabajando a partir del método automatizado. \\

Dado que es un trabajo semianalítico (parte analítica y parte computacional) se incluirán algunos enlaces que llevarán a ejemplos o códigos que se usen dentro del programa, con la documentación correspondiente. 

