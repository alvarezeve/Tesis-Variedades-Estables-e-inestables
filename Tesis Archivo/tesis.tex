	%%%%%%%%%%%%%%%%%%%%%%%%%%%%%%%%%%%%%%%%%%%%%%%%%%%%%%%%%%%%%%%%%%%%%%%%%%%%%%%%
%                         FORMATO DE TESIS FC UNAM                             %
%%%%%%%%%%%%%%%%%%%%%%%%%%%%%%%%%%%%%%%%%%%%%%%%%%%%%%%%%%%%%%%%%%%%%%%%%%%%%%%%
% based on Harish Bhanderi's PhD/MPhil template, then Uni Cambridge
% http://www-h.eng.cam.ac.uk/help/tpl/textprocessing/ThesisStyle/
% corrected and extended in 2007 by Jakob Suckale, then MPI-iCBG PhD programme
% and made available through OpenWetWare.org - the free biology wiki

%                     Under GNU License v3

% ADAPTADO PARA FI-UNAM:  Jesús Velázquez y Marco Ruiz
\sloppy
\documentclass[twoside,11pt]{Latex/Classes/PhDthesisPSnPDF} 
\usepackage{blindtext}
\usepackage[spanish]{babel}
\usepackage{babelbib}
%\usepackage{multicols}
\usepackage{natbib} 
\usepackage{hyperref}
\usepackage{url}
\usepackage{hyperref}
\usepackage{breakurl}
%\usepackage[dvips]{graphicx}
\usepackage{graphicx}
\usepackage{fancyhdr}
\usepackage{multicol}
\usepackage{mathtools}
\usepackage{float}
\usepackage{theorem}
\setlength{\theorempreskipamount}{5mm}
\setlength{\theorempostskipamount}{5mm}
\theoremstyle{break}
\theorembodyfont{\normalfont}
\theoremheaderfont{\scshape}
\newtheorem{thm}{\bf \textit{Teorema} }
\newtheorem{lem}{\bf \textit{Lema}    }
\newtheorem{defini}{\bf \textit{Definición}    }
\newtheorem{corola}{\bf \textit{Corolario}    }
\usepackage{algorithm}
\usepackage{algpseudocode}
\usepackage[all]{xy} %paquete de diagramas
\setcitestyle{square}
\setlength{\parindent}{0cm}
%\usepackage{lipsum}% http://ctan.org/pkg/lipsum
%\usepackage{showframe}% http://ctan.org/pkg/showframe
%\usepackage{eso-pic}% http://ctan.org/pkg/eso-pic
\usepackage{graphicx}% http://ctan.org/pkg/graphicx
%\usepackage{algorithm2e}
%[ruled,vlined,lined,linesnumbered,algochapter,portugues]

\makeatletter
\def\BState{\State\hskip-\ALG@thistlm}
\makeatother

                     % Para insertar texto dummy, de ejemplo, pues.
% Note:
% The \blindtext or \Blindtext commands throughout this template generate dummy text
% to fill the template out. These commands should all be removed when 
% writing thesis content.
\include{Latex/Macros/MacroFile1}           % Archivo con funciones útiles
%%%%%%%%%%%%%%%%%%%%%%%%%%%%%%%%%%%%%%%%%%%%%%%%%%%%%%%%%%%%%%%%%%%%%%%%%%%%%%%%
%                                   DATOS                                      %
%%%%%%%%%%%%%%%%%%%%%%%%%%%%%%%%%%%%%%%%%%%%%%%%%%%%%%%%%%%%%%%%%%%%%%%%%%%%%%%%
\title{Método de parametrización: Variedades estables e inestables en mapeos hamiltonianos de dos dimensiones}
\LARGE\author{Evelyn Álvarez Cruz}        
\degree{Física}               % Carrera
\director{Dr. Luis Benet Fernández}               % Director de tesis
\degreedate{2019}                           % Año de la fecha del examen
\lugar{Ciudad Universitaria, CD.MX. }                        % Lugar
%\portadafalse                              % Portada en NEGRO, descomentar y comentar la línea siguiente si se quiere utilizar
\portadatrue                                % Portada en COLOR

\keywords{tesis,autor,tutor,etc}            % Palablas clave para los metadatos del PDF
\subject{tema_1,tema_2}    

           % Tema para metadatos del PDF  

%%%%%%%%%%%%%%%%%%%%%%%%%%%%%%%%%%%%%%%%%%%%%%%%%%%%%
%                   PORTADA                         %
%%%%%%%%%%%%%%%%%%%%%%%%%%%%%%%%%%%%%%%%%%%%%%%%%%%%%
\begin{document}
\maketitle									% Se redefinió este comando en el archivo de la clase para generar automáticamente la portada a partir de los datos

%%%%%%%%%%%%%%%%%%%%%%%%%%%%%%%%%%%%%%%%%%%%%%%%%%%%%
%                  PRÓLOGO                          %
%%%%%%%%%%%%%%%%%%%%%%%%%%%%%%%%%%%%%%%%%%%%%%%%%%%%%
\frontmatter
%%\chapter*{}
%\pagenumbering{Roman}

\begin{tabular}{ll}
\textbf{1. Datos de la alumna}	&  \\ 
Apellido paterno:	& Álvarez  \\ 
Apellido materno:	& Cruz \\ 
Nombre(s):	& Evelyn \\ 
Teléfono:	& 5521664066 \\ 
Universidad	:& Universidad Nacional Autónoma de México  \\ 
Facultad:	& Facultad de Ciencias  \\ 
Carrera:	& Física \\ 
Número de cuenta:	& 309112426 \\ 
	&  \\ 
\textbf{2. Datos del tutor	}&  \\ 
Grado:	&  Dr.\\ 
Nombre(s):	&  Luis\\ 
Apellido paterno	& Benet \\ 
Apellido materno	&  Fernández\\ 
	&  \\ 
\textbf{3. Datos del sinodal 1	}&  \\ 
Grado:	&  Dr.\\ 
Nombre(s):	&  David\\ 
Apellido paterno	& Philip\\ 
Apellido materno	&  Sanders\\ 
	&  \\ 
\textbf{4. Datos del sinodal 2	}&  \\ 
Grado:	&  Dr.\\ 
Nombre(s):	&  Christof Friedrich\\ 
Apellido paterno	& Jung\\ 
Apellido materno	&  Kohl\\ 
&  \\
\textbf{3. Datos del sinodal 3	}&  \\ 
Grado:	&  Dr.\\ 
Nombre(s):	&  Renato Carlos\\ 
Apellido paterno	& Calleja\\ 
Apellido materno	&  Castillo\\ 
&  \\  
\textbf{3. Datos del sinodal 4	}&  \\ 
Grado:	&  Dr.\\ 
Nombre(s):	&  Ricardo Atahualpa\\ 
Apellido paterno	& Solórzano\\ 
Apellido materno	&  Kraemer\\ 
&  \\  	
\textbf{Datos del trabajo escrito}& \\
Título:& Método de parametrización: Variedades estables e inestables \\
& en mapeos Hamiltonianos de dos dimensiones.\\
Número de páginas:& 61\\
Año:&2019\\
\end{tabular} \\

















\begin{dedication}
A mis padres y mis hermanos, quienes me enseñaron \\


Gracias al Programa de Apoyo a Proyectos de Investigación e Innovación Tecnológica(PAPIIT) de la UNAM con clave IG100616. Agradezco a la DGPA-UNAM la beca recibida.
\end{dedication}
 
\include{Declaracion/Declaracion}  
%\chapter*{}
%\pagenumbering{Roman}

\begin{acknowledgements}
Gracias a:\\

El Dr. Luis Benet Fernández, quién me guió en la elaboración de esta tesis. Por su tiempo, paciencia y experiencia, que me ayudaron a completar mi educación como universitaria. Por contagiarme de nuevo el entusiasmo por la ciencia en general, la pasión por la física y la enseñanza de la misma. \\

Al  Dr. Christof Jung, al Dr. David P. Sanders, al Dr. Renato Calleja y al Dr. Ricardo A. Solórzano, que con sus valiosos comentarios nutrieron el contenido de este trabajo. Algunas de las secciones se crearon a partir de sus preguntas o sugerencias. \\

Al Dr. Marcos Ley Koo quién me mostró una cara de la física que no se aprende en las aulas, esa que se descubre en todos lados y que te hace un físico fuera de la academia. Que con sus pláticas me llenaba la mente de preguntas que muchas veces me quitaron el sueño. Quién siempre me sorprendía con una nueva forma de entender la termodinámica, a partir de la mecánica o del electromagnetismo y hasta de la geometría. Por compartir sus locas ideas conmigo. \\

A mi familia, sobre todo a mis padres, que con su esfuerzo, valentía y trabajo lograron que cumpliera con mis estudios universitarios. Por la educación que me dieron, con la que aprendí a disfrutar las cosas más simplemente bellas y con la que me he guiado hasta este momento. Por enseñarme a ser crítica, buscar, preguntar y observar, mostrándome que todo eso no se obtiene de un título universitario y tampoco va acompañado de traje, que se puede encontrar en un conductor de tráiler o en un comerciante. A mis hermanos por ser mi apoyo constante, mis compañeros eternos y mis cómplices.\\

A \textquotedblleft el lugar\textquotedblright, a mis amigos, los que están cerca o al otro lado del mundo, por compartir conmigo la vida escolar, por hacerla divertida y memorable. Por intercalar horas de estudio con interminables charlas, risas, con juegos de mesa de 6 horas o con domingos de series.\\

A quién vio esta tesis desde que apenas tenía el nombre y que estuvo presente en todo el proceso, quién me escuchó maldecir cuando el código no funcionaba y me vio pelear con el módulo de álgebra lineal muchas noches. Por ser mi confidente y enseñarme que \textquotedblleft una pasión es una pasión\textquotedblright.\\

Finalmente, al Programa de Apoyo a Proyectos de Investigación e Innovación Tecnológica(PAPIIT) de la Dirección General de Asuntos del Personal Académico (DGAPA) de la UNAM con clave IG100616, por la beca recibida.

\end{acknowledgements}




     % Comentar línea si no se usa
\include{introduccion/introduccion}            % ~10 páginas - Explicar el propósito de la tesis
  % Comentar línea si no se usa 
%\include{Declaracion/Declaracion}           % Comentar línea si no se usa
%\include{Resumen/Resumen}                   % Comentar línea si no se usa

%%%%%%%%%%%%%%%%%%%%%%%%%%%%%%%%%%%%%%%%%%%%%%%%%%%%%
%Esta sección genera el índice
\setcounter{secnumdepth}{3} % organisational level that receives a numbers%                   ÍNDICES                         %
%%%%%%%%%%%%%%%%%%%%%%%%%%%%%%%%%%%%%%%%%%%%%%%%%%%%%
\setcounter{tocdepth}{3}    % print table of contents for level 3
\tableofcontents            % Genera el índice 
%: ----------------------- list of figures/tables ------------------------
%\listoffigures              % Genera el ínidce de figuras, comentar línea si no se usa
%\listoftables               % Genera índice de tablas, comentar línea si no se usa


%%%%%%%%%%%%%%%%%%%%%%%%%%%%%%%%%%%%%%%%%%%%%%%%%%%%%
%                   CONTENIDO                       %
%%%%%%%%%%%%%%%%%%%%%%%%%%%%%%%%%%%%%%%%%%%%%%%%%%%%%
% the main text starts here with the introduction, 1st chapter,...
\mainmatter
\def\baselinestretch{1.5}                   % Interlineado de 1.5
%%%%%%%%%%%%%%%%%%%%%%%%%%%%%%%%%%%%%%%%%%%%%%%%%%%%%%%%%%%%%%%%%%%%%%%%%
%           Capítulo 2: MARCO TEÓRICO - REVISIÓN DE LITERATURA
%%%%%%%%%%%%%%%%%%%%%%%%%%%%%%%%%%%%%%%%%%%%%%%%%%%%%%%%%%%%%%%%%%%%%%%%%
\chapter{Sistemas Hamiltonianos de dos dimensiones}
Para llegar al análisis de los sistemas de nuestro interés mediante el método de parametrización es fundamental conocer varios conceptos sobre la dinámica así como del análisis numérico del mismo. En este capítulo se hace una breve descripción de los teoremas y definiciones que nos ayudarán a entender el método. Las siguientes secciones no son la manera más formal de introducir al lector a cada uno de los temas expuestos. Sin embargo proporcionan una visión puntual de lo indispensable. 

\section{Sistemas dinámicos}
Cuando uno habla sobre sistemas dinámicos lo que le viene a la mente es alguna relación que describe el comportamiento temporal de un sistema físico. Ya sea el movimiento de péndulos o de cargas, nos interesa saber más acerca de las características de su comportamiento. 
En el estudio de los mismos se hace una clasificación en términos de sus propiedades vistas en el espacio fase o por la forma del sistema. Hay sistemas dinámicos discretos y sistemas dinámicos continuos. \\

 Dentro de éstas características de clasificación también puede estar que el sistema sea de tipo determinista o estocástico. La diferencia entre ambos es que para el determinista dado un punto en el espacio fase existe uno y sólo un punto subsecuente bien definido, mientras que en el estocástico para un estado puede haber varios estados subsecuentes posibles.
En algunos casos los sistemas resultan ser simples, por ejemplo si su movimiento es regular. Sin embargo hay muchos otros sistemas que no son regulares en los que para dos condiciones iniciales parecidas los resultados de la dinámica después de cierto tiempo son diferentes. Para este trabajo nos enfocaremos en los  sistemas deterministas discretos.\\

Para formalizar usaremos la siguiente definición de sistema dinámico.  \\
\begin{defini}[\textsc{\textit{Sistema dinámico}}\cite{gerald}]
\textit{Un sistema dinámico es un semigrupo $G$ actuando en un espacio $M$}
\begin{equation}
T : G \times M \rightarrow M; \quad
T_{g}\circ T_{h}=T_{g\circ h} . \label{def sistema dinamico}
\end{equation}
\end{defini}


Un ejemplo típico de un sistema dinámico continuo es el flujo de una ecuación diferencial autónoma , mientras que de uno discreto es por ejemplo el mapeo de un intervalo cerrado en $\mathbb{R}$ en sí mismo, o simplemente una función iterada. En el primer caso
\begin{equation}
\dot{x} =  f(x); \quad  
x(0)=x_{0} , \label{ec dif}
\end{equation}
suponiendo que $f \in C^{k}(M,\mathbb{R}^{n})$ con $k \geq 1$ y donde $M$ es un subconjunto abierto de $\mathbb{R}^{n}$. Las soluciones de este tipo de sistemas son curvas contenidas en $\mathbb{R}^{n}$ a las que llamamos trayectorias y denotamos por $\phi$.\\
En el segundo caso  
\begin{eqnarray}
\pmb x_{n+1}= \pmb F (\pmb x_{n}) \quad \pmb x\in \mathbb{R}^{n}. \label{sistema discreto}
\end{eqnarray}
Para la ecuación \ref{sistema discreto} dada una condición inicial $\pmb x_{0}$ podemos obtener el estado siguiente evaluando el lado derecho con tal punto. De manera sucesiva se puede obtener el estado $\pmb x_{n+1}$ del estado $\pmb x_{n}$. La aplicación consecutiva de la función proporciona la trayectoria u órbita del punto inicial. Podemos decir que el mapeo es un isomorfismo entre la condición inicial y los puntos de la trayectoria,
\begin{eqnarray*}
\pmb x_{k+1}=F\circ F \circ F \circ F ... \circ F (\pmb x_{0})\quad (k \quad \textrm{veces})
\end{eqnarray*}
o
\begin{eqnarray*}
\pmb x_{k+1} = F^{k}(\pmb x_{0}).
\end{eqnarray*}
Al conjunto $\lbrace \pmb x_{0},\pmb x_{1},...,\pmb x_{n} \rbrace$ se le llama la órbita de $\pmb x_{0}$.  En este tipo de sistemas es posible que exista un $\pmb x_{*}$ tal que $\mathbf{F}^{p}(\pmb x_{*})=\pmb x_{*}$ con $p \in \mathbb{Z}$. Es decir después de un número finito de aplicaciones se vuelve al mismo punto ($\pmb x_{*}$) al cual le llamamos un punto de periodo $p$. También es entonces posible  que exista un punto de periodo uno $\pmb x_{*}=\mathbf{F}(\pmb x_{*})$ al cual llamamos punto fijo. Los puntos fijos serán otra forma de clasificar a los sistemas dinámicos, para ello separamos las órbitas en términos de los periodos:

\begin{itemize}
\item  Órbitas fijas (asociadas a puntos con periodo uno).
\item Órbitas periódicas regulares (asociadas a puntos con periodo mayor a uno).
\item Órbitas no cerradas (asociadas a puntos no periódicos).
\end{itemize}
Nos enfocaremos sobre todo en los puntos fijos, sin embargo es posible también analizar en aquellos puntos de periodo mayor a uno.
\section{Mapeos}
Como mencionamos en la introducción existen diferentes características de los sistemas dinámicos, la forma en la que depende un estado del estado anterior es una de ellas. Esa dependencia está determinada por $\pmb F$ en la ecuación \ref{sistema discreto}; gráficamente se describe en el espacio fase, que es el espacio de todos los posibles valores de $\pmb x$. En el mismo espacio la órbita de cualuqier punto se ve como una curva que representa la evolución de un punto $\pmb x$ bajo el mapeo hacia adelante y hacia atrás. \\

Las funciones que aparecen en el mapeo pueden contener términos con potencia mayor a uno, productos entre las variables o peor aún funciones mucho más difíciles de manejar, lo que hace que la suma de soluciones no sea solución, esa característica se llama no linealidad. Una de las primeras cosas a analizar serían los puntos fijos del sistema, los cuales deberán encontrarse algebráicamente o numéricamente dependiendo de la dificultad del mapeo. Supongamos que se tiene un punto fijo de un mapeo, al rededor del punto fijo pueden presentarse diversos comportamientos de las trayectorias; como el sistema es no lineal es difícil decir qué tipo de evolución sufre el sistema visto en el espacio fase.\\


Si tenemos un puno fijo $\pmb x_{*}$ asociado al mapeo entonces la ecuación
\begin{eqnarray}
\pmb x_{n+1} =\mathbf{A}\pmb x_{n}
\end{eqnarray}
representa la linearización del sistema en al punto, donde $\mathbf{A}=D\pmb F(x_{*})$. Es decir el mapeo puede ser representado como una matriz de coeficientes constantes,además si el sistema dinámico es invertible se puede conocer el punto anterior $\pmb x_{n-1}$ a un cierto $\pmb x_{n}$ usando la matriz inversa de $\mathbf{A}$ que representa el mapeo inveso $\pmb F^{-1}$. Es decir las características de tal matriz nos dirán el comportamiento del sistema. \\

Dado que los sistemas tratados en este trabajo son de dos dimensiones entonces analizaremos sólo este caso. Los valores propios $\lambda_{1}, \lambda_{2}$, soluciones del polinomio característico de grado dos, son en general valores complejos clasificados como sigue.
\begin{itemize}
\item $\vert \lambda_{i}\vert<1$
\item $\vert \lambda_{i}\vert>1$
\item $\vert \lambda_{i}\vert=1$
\end{itemize}
En cada uno de los casos anteriores podremos tener dos vectores propios $\pmb x_{p1}, \pmb x_{p2}$ asociados a cada valor propio
\begin{eqnarray}
\mathbf{A}\pmb x_{p1}=\lambda_{1}\pmb x_{p1} ;\qquad \mathbf{A}\pmb x_{p2}=\lambda_{2}\pmb x_{p2}.
\end{eqnarray}
Los vectores serán linealmente independientes si $\lambda_{1},\lambda_{2}$ son diferentes. Si además consideramos que los valores propios son reales, según \cite{Friedberg} existe una matriz $\mathbf{U}$ tal que 
\begin{eqnarray}
\mathbf{U}^{-1}\mathbf{A}\mathbf{U} = \begin{pmatrix}
\lambda_{1} & 0\\
0 & \lambda_{2}\\
\end{pmatrix}
\end{eqnarray}.
A partir de esto no es difícil ver que $\mathbf{U}$ tiene como columnas los vectores propios. Consecuentemente un mapeo lineal de dimensión dos es linealmente conjugado con un mapeo el cual tiene una matriz de representación diagonal. Este tipo de matrices son llamadas formas normales y
permiten representar el sistema en su forma más simple mediante la diagonalización del mismo.\\

Cuando $\mathbf{A}$ es una matriz diagonalizable y sus valores propios son puramente imaginarios entonces decimos que tiene un comportamiento elíptico. Si es el caso que ninguno de los dos valores propios tienen parte real cero entonces la matriz se dice hiperbólica y el punto fijo asociado es un punto fijo hiperbólico. El resultado de estas características visto en el espacio fase es un comportamiento muy particular que se puede muestra en la figura \ref{hiperbolic}. \\

\begin{figure}[h!]
\centering
\includegraphics[scale=0.3]{hyperbolic} 
\caption{Punto fijo elíptico(izquierda) e hiperbólico(derecha) en el espacio fase}\label{hiperbolic}
\end{figure}


%Retomando además que si los valores propios eran ortogonales entonces podemos escribir los subespacios generados por los vectores propios.
%\begin{eqnarray*}
%E^{s}=\lbrace (x,y) : (x,y)=\beta \pmb v_{1} \quad \beta\in \mathbb{R}\rbrace
%\end{eqnarray*}
%\begin{eqnarray*}
%E^{u}=\lbrace (x,y) : (x,y)=\alpha \pmb v_{2}\quad \alpha\in \mathbb{R}\rbrace
%\end{eqnarray*}

En el caso hiperbólico tenemos dos comportamientos que nos interesan, marcados con las líneas azul y verde de la figura anterior, tales corresponden a los eigenespacios asociados a los vectores propios de $\mathbf{A}$. Existe un teorema importante de Hartman-Grobman que nos asegura que hay una vecindad del punto fijo hiperbólico tal que el mapeo es topológicamente conjugado con su linearización \cite{Meiss,Meyer,Juergen}. Dicho de otra manera hay vecindades $U$ de $x_{*}$, $V$ de $0 \in \mathbf{R}^{2}$ y un homeomorfismo $h:U\rightarrow V$ tal que $h$ mapea trayectorias de $\pmb F$ en trayectorias del sistema lineal. De esta manera justificamos el porque se trabaja con un sistema lineal.



\section{Conjuntos invariantes}
Al rededor de un punto fijo existen ciertos conjuntos que nos dirán características globales del sistema, estos conjuntos tienen que ver directamente con lo que se observa en la figura \ref{hiperbolic}. Para entender su comportamiento necesitamos definir un conjunto invariante.

\begin{defini}[\textit{\textsc{Conjunto invariante}}\cite{Ott}]
\textit{Un conjunto invariante es un subconjunto $\mathbf{I} \subset \mathbf{E}$ del espacio fase tal que para cualquier $\pmb x_{i}\in  \mathbf{I}$  y $ n\in\mathbb{N}$ \\
\begin{center}
$\mathbf{F}^{n}(\pmb x_{i}) \in \mathbf{I}$.
\end{center}
}
\end{defini}
Es decir que cualquier elemento tomado en el conjunto se queda en el conjunto bajo la aplicación del mapeo. \\
Estudiaremos los conjuntos invariantes asociados a puntos fijos hiperbólicos, si $\pmb x_{*}$ es un punto fijo hiperbólico entonces definimos las variedades estable e inestable como
\begin{eqnarray}
W^{s}=\lbrace \pmb x : \pmb F^{n}(\pmb x)\rightarrow \pmb x_{*} \quad cuando \quad n\rightarrow \infty \rbrace
\label{variedad estable}
\end{eqnarray}

\begin{eqnarray}
W^{u}=\lbrace \pmb x : \pmb F^{n}(\pmb x)\rightarrow \pmb x_{*} \quad cuando \quad n\rightarrow -\infty \rbrace.
\label{variedad inestable}
\end{eqnarray}
Localmente las variedades resultan ser tangentes a los subespacios generados por los vectores propios
\begin{eqnarray*}
E^{s}=\lbrace (x,y) : (x,y)=\beta \pmb v_{1} \quad \beta\in \mathbb{R}\rbrace
\end{eqnarray*}
\begin{eqnarray*}
E^{u}=\lbrace (x,y) : (x,y)=\alpha \pmb v_{2}\quad \alpha\in \mathbb{R}\rbrace
\end{eqnarray*}
Esto se resume en el siguiente teorema.


%\begin{defini}[Variedad estable e inestable para un punto fijo]
%Sea \textit{$F:\mathbb{R}^{2}\rightarrow\mathbb{R}^{2}$ un  mapeo y sea $\pmb x_{*}$ un punto fijo %del mapeo , si $C \in \mathbb{R}^{2}$ es una vecindad del punto fijo  definimos}
%\begin{eqnarray*}
%W_{loc}^{s}(\pmb x_{*})=\lbrace \epsilon \in  C : \mathbf{F}^{n}(\pmb x_{i}) \in C \quad \forall n\in \mathbb{N} \rbrace
%\end{eqnarray*}
%\label{Variedad  defini}
%\end{defini}
%la definición $4$ es análoga para la variedad inestable $W_{loc}^{u}$ , pero sólo nos da una relación local. Si queremos ir más alejados del punto fijo entonces necesitamos una variedad que sea más general.Para ello existe el teorema siguiente.
%\begin{thm}[De la variedad estable(inestable)]
%\textit{Sea $\mathbf{A}$ una matriz de $2\times 2$ con dos valores propios $\lambda_{1},\lambda_{2}$ donde uno tiene parte real positiva y otro negativa, tal que si $\epsilon $ es lo suficientemente pequeña y positiva existe $ W^{s}(\pmb x_{i})$  entonces $F^{n}(\pmb x_{i})\rightarrow x_{*}$ cuando $n\rightarrow\infty$.  Respectivamente para $W^{u}(\pmb x_{i}
%)$ tenemos que $F^{n}(\pmb x_{i})\rightarrow x_{*}$ cuando $n\rightarrow -\infty$. }
%\end{thm}






%Con esto podemos asegurarar que existen funciones lo sificientemente suaves que se extienden por la %variedad, no solo de manera local. Esto es de suma importancia en el método de parametrización que %aplicaremos,ya que la parametrización se hace por medio de polinomios que son funciones suaves y %que con lo que acabamos de mencionar no sólo se aproximan de manera local. Si escribimos el %conjunto de la definición anterior
%\begin{eqnarray}
%W^{s}(\pmb x_{i})=\lbrace \pmb x_{i} \in \mathbb{R}^{2}: \lim_{n\rightarrow\infty} \mathbf{F}^{n}%(\pmb x_{i}) =x_{*}\rbrace \label{omega s}
%\end{eqnarray}
%podemos notar que además tal conjunto consiste en todas las órbitas que se acumulan bajo la %aplicación iterada del mapeo al punto fijo. En el caso en que el mapeo se a invertible entonces %podemos relacionar 
%\begin{eqnarray*}
%W^{s}(\pmb x_{*})=\cup_{n=0}^{\infty}\mathbf{F}^{n}[W_{loc}^{s}(\pmb x_{i})]
%\end{eqnarray*}
%que nos dice que la variedad estable se obtiene de la union de todas las aplicaciones hacia atrás de la variedad local estable. Para escribir de manerá análoga la variedad inestable digamos primero que la órbita hacia atrás de $\pmb x_{k}$  es $\mathbf{F}(\pmb x_{k})=\pmb x_{k+1}$ con $k\leq -1 $ donde además
%\begin{eqnarray*}
%\lim_{k\rightarrow -\infty} \pmb F(\pmb x_{k})=x_{*}
%\end{eqnarray*}
%si el mapeo es invertible entonces la órbita hacia atrás es simplemente la aplicación iterada de la %inversa. Entonces
%\begin{eqnarray*}
%W^{u}(\pmb x_{*})=\lbrace \pmb x_{i} \in \mathbb{R}^{2}: \lim_{n\rightarrow\infty} \pmb F^{n}(\pmb %x_{i})=\pmb x_{*}\rbrace \label{omega u}
%\end{eqnarray*}

%Para relacionar esto con los sistemas de interés regresemos a la ecuación \ref{sistema discreto} %donde es evidente que el origen $[0,0]$ es un punto fijo de cualquier sistema que tenga la misma forma. Si ninguno de los dos eigenvalores $\lambda_{1,2}$ de la matriz $\mathbb{A}$ tiene módulo uno  y se tiene que ambos son de signo contrario entonces la matriz es llamada hiperbólica y $x_{*}$ un punto fijo hiperbólico. Mientras que si por ejemplo los dos valores propios de $A$ tienen solo parte imaginaria entonces el sistema es llamado elíptico. El comportamiento de ambos casos se puede ver en la figura \ref{hiperbolic}. Estos conjuntos invariantes relacionados con el punto fijo resultan ser espacios propios generalizados de la matriz $\mathbb{A}$ generados por los vectores propios. Es decir si $\pmb x_{pj}=\pmb u_{j}+i\pmb v_{j}$ proviene de una base 
%\begin{eqnarray}
%\mathbf{B}=\lbrace \pmb u_{1},\pmb v_{1},\pmb u_{2},\pmb v_{2} \rbrace
%\end{eqnarray}
%entonces hay diferentes direcciones asociadas al punto fijo de acuerdo con la base $\mathbf{B}$. Para los sistemas que trabajaremos resulta que no sólo la norma de los valores propios son menores o mayores a uno, si no que además la parte imaginaria es cero para ambos. Lo que quiere decir que la base $\mathbf{B}$ tiene dos elementos únicamente. Los elementos de la misma forman subespacios, un subespacio asociado al valor propio con norma mayor a uno y otro con norma menor a uno. 

%Tomado de MAteo wirth
%\begin{thm}
%\textit{Dado un sistema de la forma \ref{sistema discreto} donde los valores propios de $\mathbb{A}$ tengan parte real diferente de cero entonces}
%\begin{eqnarray*}
%W^{s} = E^{s} 
%\end{eqnarray*}
%\begin{eqnarray*}
%W^{u}=E^{u}
%\end{eqnarray*}
%\end{thm}
%\textit{donde $E^{s}$ es la suma directa de los eigenespacios asociados al valor propio con parte real negativa , $E^{u}$ es análogo pero con positivo. En particular}
%\begin{eqnarray*}
%\mathbb{R}^{n}= W^{s}\oplus W^{u}
%\end{eqnarray*}

%Puesto que los valores propios son reales,  y entonces también los vectores propios. Este teorema nos resume que la suma directa de los conjuntos invariantes del sistema forman al espacio de soluciones. Además de mostrarnos que los conjuntos invariantes son para este caso las variedades invariantes. En el caso de sistemas más generales tenemos el siguiente teorema, que además nos aclara a qué llamamos una variedad inestable. \\


\begin{thm}[\underline{\textit{De la variedad estable}} \cite{Mateo}]
\textit{Sea un sistema de la forma $\pmb x_{n+1}=\mathbf{F}(\pmb x_{n})$ con un punto fijo en el origen. Sean $E^{s}$ y $E^{u}$ los subespacios estables e inestables de la linearización del sistema,$F(\pmb x_{n})\rightarrow\mathbb{J}\pmb x_{n}$,  donde $\mathbb{J}$ es la matriz jacobiana en el origen . Si $\mid F(\pmb x_{n})-\mathbb{J}\pmb x_{n}\mid =O(x^{2})$ entonces existen localmente variedades estables e inestables con las mismas dimensiones que $E^{s}$,$E^{u}$ y que son tangentes a estos en cero respectivamente.}
\begin{eqnarray*}
W^{s}_{loc}(\pmb x_{*})= \lbrace \pmb x_{n} : \mathbf{F}^{k}(\pmb x_{n})\rightarrow \pmb x_{*} cuando\quad k \rightarrow \infty \rbrace
\end{eqnarray*}
\begin{eqnarray*}
W^{u}_{loc}(\pmb x_{*}) = \lbrace \pmb x_{*} : \mathbb{F}^{k}(\pmb x_{n})\rightarrow \pmb x_{*} cuando\quad k \rightarrow -\infty \rbrace
\end{eqnarray*}
\end{thm}

Es necesario mencionar que una variedad estable no puede cruzarse con otra variedad estable, lo mismo sucede con las inestables, así como con la intersección de una variedad con sí misma. Para entender esto consideremos que se tienen dos puntos fijos diferentes con sus respectivas variedades inestables asociadas. Supongamos que las variedades se cruzan en algún punto, si esto pasa la órbita hacia atrás de cualquiera de los dos puntos empezando en la intersección debería aproximarse a ambos puntos fijos, lo cual es imposible pues eran diferentes. Los argumentos para la intersección de las estables son similares. Lo que sí puede suceder es la intersección de una variedad estable con una inestable, a esto se le llama una intersección homoclínica. Si la intersección se da en otro punto fijo entonces se llama heteroclínica \cite{Ott}. Resulta además que si dos variedades , estable e inestable, se cortan en un punto se cortarán una infinidad de veces más. \\


El cálculo de variedades alrededor de un punto fijo es un problema difícil de atacar analíticamente, pues su comportamiento puede ser muy complejo, por ello es necesario explotar al máximo la linearización que se hace del sistema para poder, con métodos numéricos o semianalíticos, calcular las variedades. Es justo esto lo que nos lleva a la siguiente sección.




\section{Sistemas Hamiltonianos}
Los sistemas Hamiltonianos son una clase particular de los sistemas dinámicos. En 1834 William R. Hamilton reformuló la ecuación de Newton ($F=ma$) para un conjunto de partículas puntuales en un campo de fuerzas. Cuando la fuerza $\mathbf{F}$ es conservativa es posible escribir a la fuerza como  menos el gradiente de una función potencial. 
\begin{eqnarray}
\mathbf{F}=-\nabla V \label{fuerza potencial}
\end{eqnarray}
Podemos convertir la ecuación \ref{fuerza potencial} en un sistemas de ecuaciones diferenciales.
\begin{eqnarray}
\frac{dx_{i}}{dt}=v_{i};   \quad m_{i}\frac{dv_{i}}{dt}=-\nabla_{i} V
\label{fuerza sistema dif}
\end{eqnarray}

Aquí es donde Hamilton notó que estas ecuaciones pueden obtenerse a partir de una función muy particular.
\begin{eqnarray}
H(q,p)=\sum_{i=i}^{n} \frac{p_{i}^{2}}{2m_{i}}+V(q) \label{ec de hamilton}
\end{eqnarray}
A la función \ref{ec de hamilton} actualmente se le llama Hamiltoniana, donde $p_{i}$ denota el momento y \textbf{$q$} la colección de las posiones en forma de vector. Al comparar \ref{ec de hamilton} con \ref{fuerza sistema dif} tenemos que las ecuaciones de movimiento son
\begin{eqnarray}
\frac{dq_{i}}{dt}=\frac{\partial H}{\partial p_{i}}; \quad
\frac{dp_{i}}{dt}=-\frac{\partial H}{\partial q_{i}}
\label{ec de mov hamilton}
\end{eqnarray}

Físicamente hablando H es la energía total del sistema, que es invariante en el tiempo, $\frac{dH}{dt}=0$.La formulación Hamiltoniana de la mecánica no está limitadad a sistemas que son de la forma "energía cinética más energía potencial". Ya que de manera más general una Hamiltoniana es cualquier funcion $C^{1}$, $H:M\rightarrow \mathbb{R}$ donde en nuestro caso $M$ es una variedad 2n-dimensional con coordenadas  $z=(q,p)$. Para escribir las ecuaciones de Hamilton de manera resumida
\begin{eqnarray}
\frac{dz}{dt}=J\nabla H \label{Hamilton-poisson}
\end{eqnarray}

Donde $I$ es la matriz identidad de $n\times n$ por lo que $J$ es de $2n\times 2n$ antisimétrica, llamada matriz de Poisson. 
\begin{eqnarray*}
J=  \begin{pmatrix}
0 & I\\
-I & 0
\end{pmatrix}
\end{eqnarray*}
Si por otro lado pensamos que el cambio de una función escalar $F$ que depende del tiempo se puede calcular usando la ecuacion \ref{Hamilton-poisson} mediante la regla de la cadena 
\begin{eqnarray*}
\frac{dF}{dt}=\frac{\partial F}{\partial t}+ \lbrace F,h\rbrace
\end{eqnarray*}
Aquí la expresión $\lbrace F,H \rbrace$ es llamada el paréntesis de Poisson definido como:
\begin{eqnarray}
\lbrace F,H \rbrace=\nabla F^{T}J\nabla H \label{Parentesis Poisson}
\end{eqnarray}
que nos sirve para escribir las ecuaciones de movimiento \ref{ec de mov hamilton}
\begin{eqnarray}
\dot{z}=\lbrace z,H\rbrace \label{ec. de movimiento 2}
\end{eqnarray}
Algunas de las cosas que nos interesan en este tipo de sistemas son las cantidades conservadas. La primera de ellas que nos interesa es la energía.\\

\begin{lem}(\emph{\textit{Conservación de energía}})
\textit{Si $H$ es independiente del tiempo entonces la energía se conserva a lo largo de trayectorias. $H(q(t),p(t))=E$.}
\end{lem}
\textit{\textbf{Demostración:}}
\begin{equation*}
\frac{dH}{dt}=\lbrace H, H \rbrace = \nabla H^{T} J \nabla H=0  
\end{equation*}
ya que J es antisimétrica. $\blacksquare$ \\

La otra cantidad que nos interesa es el volumen, Joseph Liouville mostró que los flujos Hamiltonianos preservan el volumen.

\begin{lem}(\emph{\textit{Liouville}})
\textit{Si $H$ es $C^{2}$ entonces su flujo preserva el volumen.}
\end{lem}

Una carcaterística más de los sistemas Hamiltonianos es que sus puntos criticos son equivalentemente sus puntos fijos.Lo cual se enuncia en el siguiente lema. 
\begin{lem}(\emph{Equilibrio})
\textit{Un punto $z^{*}$ es un punto fijo del flujo autónomo Hamiltoniano si y sólo si es un punto crítico de $H$}.
\end{lem}

Consecuentemente de eso tenemos que al ser iguales sus puntos críticos y fijos entonces la estabilidad de los mismos se puede estudiar a partir de la matriz Hessiana de $H$. Lo cuál analizaremos en una sección posterior. Mientras podemos pensar que una consecuencia de este lema es que cualquier máximo o mínimo no degenerdado de $H$ es Lyapunov estable. \\
Pero ¿cómo es que este tipo de sistemas se relacionan con los sistemas lineales que analizamos antes?. La clave está en escribir a nuestro sistema de manera linearizada mediante la matriz $\mathbb{J}$. Si escribimos \ref{ec de mov hamilton} como
\begin{eqnarray*}
\frac{dq_{i}}{dt}=\frac{q_{i+1}-q_{i}}{\Delta t}
\end{eqnarray*}
donde $q_{i}=q(t)$ y $q_{i+1}=q(t+\Delta t)$. Entoncs las ecuaciones de movimiento se pueden reescribir 
\begin{eqnarray}
q_{i+1}=q_{i}+\Delta t p_{i} ;\quad p_{i+1}=p_{i}-\Delta t\left( \frac{\partial V}{\partial q_{i} } \right)_{q=q_{i+1}} \label{hamilton sistema dinamico}
\end{eqnarray}
lo cual ya está en forma de un sistema de los que estudiamos anteriormente. Para linearizar el sistema calculamos el jacobiano
\begin{eqnarray}
\mathbf{J}=\frac{\partial(q_{i+1},p_{i+1})}{\partial(q_{i},p_{i})}=
det \begin{pmatrix} 
1 & -\Delta t \left( \frac{\partial^{2} V}{\partial q_{i}^{2} } \right)\\
\Delta t & 1 \\
\end{pmatrix}
\end{eqnarray}
Es justo de este sistema linearizado de donde obtendremos información a partir de aplicar los teoremas y resultados de las secciones anteriores. Pero antes, notemos que el determinante es uno
\begin{eqnarray*}
det(\mathbf{J})=1+(\Delta t)^{2}\left( \frac{\partial^{2} V}{\partial q^{2} } \right)_{q=q_{i+1}}
\end{eqnarray*}
 ya que la segunda parcial del potencial vale cero . Por lo que es un sistema que preserva áreas.









\section{Método de parametrización}
Como ya observamos en la sección anterior encontrar las variedades asociadas a un punto fijo no es trivial. Los métodos analíticos se vuelven no sólo tediosos si no que hacen necesario que el análisis de un sistema se haga de forma personalizada. Y para encontrar tales variedades debemos explotar los conocimientos que tenemos sobre los sistemas. Algunas de estas características son realmente simples, por ejemplo sabemos que en un punto hiperbólico resultarán dos variedades asociadas a los valores propios de la matriz que representa el sistema linearizado. Para este trabajo nos concentraremos en los sistemas Hamiltonianos ya que estos preservan el área y son importantes en la física, sin embargo el método funciona también para sistemas no Hamiltonianos. El objetivo de esta sección es hablar sobre el método de parametrización el cual fue desarrollado por X.Cabré, E. Fontich y R. de la Llave \cite{Haro}. El método fue desarrollado de manera general para conjuntos invariantes, se trata de un método semianalítico, es decir parte computacional y parte analítica.\\

Para ahondar en el método recordemos que anteriormente mencionamos que los conjuntos \ref{omega s}, \ref{omega u} son conjuntos invariantes. Por otro lado también recordemos la definición de sistema dinámico que nos dice que se trata de un semigrupo actuando sobre un espacio $M$, la manera en la que se genera el sistema es con un difeomorfismo $F:M \rightarrow M$. En este mismo espacio $M$ definamos una inmersión inyectiva $P:\Theta \rightarrow M$  lo cual nos define una subvariedad $\textbf{P}$  parametrizada por medio de las variables locales en $\theta \in \Theta$. La variedad invariante parametrizada por $\textbf{P}$  junto con $g:\Theta \rightarrow \Theta$ deben cumplir 
\begin{eqnarray}
F \circ P = P \circ g,  \label{Ecua de invariancia}
\end{eqnarray}
llamada ecuación de invariancia \cite{Haro}.
Es decir $\textbf{P}$ y $g$ son de tal forma que hacen que el siguiente diagrama conmute
\begin{eqnarray}
\xymatrix{
\Theta\subset\mathbb{R} \ar[d]^{\textbf{P}} \ar[r]^{g} & \Theta\subset\mathbb{R} \ar[d]^{\textbf{P}} \\
M\subset\mathbb{R}^{n} \ar[r]^{\textbf{F}} & M\subset\mathbb{R}^{n}
}\label{conmutativo}
\end{eqnarray}

En este sentido $g$ representa un subsistema de $F$ , en otras palabras $g$ contiene la dinámica del mapeo pero sobre $\Theta$. El objetivo del método de parametrización es encontrar $P$ y $g$ que cumplan la ecuación de invariancia \ref{Ecua de invariancia}. Aunque no conozcamos la dinámica interna de $g$ sabemos que $P,g$ son soluciones de \ref{Ecua de invariancia}, si observamos el diagrama \ref{conmutativo} es claro que la composición también lo es y eso nos dará una libertad para resolver la ecuación. El obstáculo, de no conocer $g$, se puede pasar si se escoge una forma de parametrización que dependa del sistema; en el método de parametrización se tienen descritas dos formas: la forma gráfica y la forma normal. Usaremos el método de la forma gráfica, que es la forma más simple de parametrización. Consiste en adaptar la forma de la parametrización $P$ a la forma de las variedades, la cual esta relacionada con la dirección que proporcionan los vectores propios. Para el caso de una matriz hiperbólica de $2\times 2$ sus vectores propios nos indicarán, suficientemente cerca del punto fijo, la dirección de cada variedad. \\


La forma en la que se escoge $g$, en la mayoría de las veces, es polinomial de tal manera que se adapte a la forma del mapeo. Sin embargo la elección puede ser diferente dependiendo del sistema. En este caso escogimos la dependencia más sencilla para $x,y$ \ref{fun g} que resulta ser suficiente en el caso hiperbólico
\begin{eqnarray}
g(t) = (\lambda t,\lambda t).
\label{fun g}
\end{eqnarray}
Con esto tendremos del lado derecho de la ecuación \ref{Ecua de invariancia} un polinomio. \\

Supongamos ahora que tenemos ya las variedades parametrizadas,para este punto es importante tener una función que nos indique que tan acertada es nuestra parametrización. La primera y más fácil forma de calcular el error es a partir de la ecuación \ref{Ecua de invariancia}, mediante la resta
\begin{eqnarray}
E_{n}(t) = \parallel F \circ P_{n} - P_{n} \circ g \parallel_{\infty}.  \label{Ecua de invariancia resta}
\end{eqnarray}
Este error será el asociado a la variación con respecto a la ecuación cohomológica. Dado que depende del parámetro esperamos que el error vaya creciendo conforme se evalúa en valores de $t$ lejos del punto fijo. La otra forma de evaluar qué tan lejos podemos llegar con la parametrización es ver la convergencia de los polinomios asociados. Al tener los polinomios de la variedad podemos evaluar el cociente entre los coeficientes de cada término.
\begin{eqnarray}
\lim_{n\rightarrow\infty}\frac{a_{n}}{a_{n+1}}\label{hadamard}
\end{eqnarray} 
donde $(a_{0},b_{0})=x_{*}$ son los coeficientes de orden cero y $a_{n}$ el de orden $n$. Lo que prácticamente estamos haciendo con este cociente es lo que se llama estudiar la convergencia según Hadamard. Si el límite anterior tiende a cero entonces la serie $a_{n}$ converge. Otra forma de evaluarlo es usando tres términos \citep{Chang}.
\begin{eqnarray}
\lim_{i\rightarrow\infty} \left[ i\left(\frac{a_{i+1}}{a_{i}}\right)-(i-1)\left(\frac{a_{i}}{a_{i-1}}\right) \right] \label{tres terminos}
\end{eqnarray}
Aunque el método se aplica de la misma manera para los mapeos analizados, es obvio que la parametrización será diferente, por lo que la convergencia de cada parametrización es distinta.   % ~15 pag.

%%%%%%%%%%%%%%%%%%%%%%%%%%%%%%%%%%%%%%%%%%%%%%%%%%%%%%%%%%%%%%%%%%%%%%%%%
%           Capítulo 2: MARCO TEÓRICO - REVISIÓN DE LITERATURA
%%%%%%%%%%%%%%%%%%%%%%%%%%%%%%%%%%%%%%%%%%%%%%%%%%%%%%%%%%%%%%%%%%%%%%%%%
\chapter{Método de parametrización}
Este capítulo describe cómo se implementó el método de parametrización aplicado a sistemas Hamiltonianos. Se comienza explicando el análisis del mapeo es\-tán\-dar, siguiendo el trabajo de Mireles James \cite{Mireles}. A partir de este trabajo se generalizó el método para los sistemas Hamiltonianos de dos dimensiones, de manera que dado un mapeo el método programado en Julia pudiera calcular de manera recurrente los polinomios asociados a las variedades.
\section{Desarrollo explícito para el mapeo estándar}
Para desarrollar el método de parametrización de manera automática se usó como base el desarrollo que aparece en las notas \cite{Mireles}. En este trabajo se expone de manera explícita cómo se calculan las variedades estables e inestables para el \textit{mapeo estándar}. El \textit{mapeo estándar} tiene la forma \citep{devaney}
\begin{eqnarray}
\mathbf{f}_{k}(\theta,p) = \left[\begin{array}{c}
\theta + p \\
p + k\sin(\theta +p)
\end{array}\right] \mod(2\pi),  \label{mapeo estandar}
\end{eqnarray}
con $k$ un parámetro. Mientras el inverso es
\begin{eqnarray}
\mathbf{f}_{k}^{-1}(p,\theta) = \left[\begin{array}{c}
p  -k\sin(\theta) \\
\theta-p+k\sin{\theta}
\end{array}\right] \mod(2\pi). \label{mapeo estandar inverso}
\end{eqnarray}
Los puntos fijos del mapeo son aquellos $\mathbf{x}$ tales que 
\begin{eqnarray}
\mathbf{f}_{k}(\mathbf{x})=\mathbf{x} \label{ec puntos fijos}
\end{eqnarray}
con $\mathbf{x}=(\theta,p)$. El resultado de esta condición son unicamente los puntos $\mathbf{x}_{1}=(0,0)$ y $\mathbf{x}_{2}=(0,\pi)$. Para analizar la estabilidad lineal del mapeo se calculó
\begin{eqnarray}
D\mathbf{f}_{k}(\theta,p)=\begin{pmatrix}
1 & 1 \\
k\cos(\theta+p)& 1+k\cos(\theta+p)\\ 
\end{pmatrix}.\label{mapeo linearizado}
\end{eqnarray}
Al evaluar en $\mathbf{x_{1}},\mathbf{x_{2}}$; \eqref{mapeo linearizado} resulta 
\begin{eqnarray}
D\mathbf{f}_{k}(0,0)=
\begin{pmatrix}
1 & 1\\
k & 1+k\\
\end{pmatrix}, \qquad D\mathbf{f}_{k}(0,\pi)= \begin{pmatrix}
1 & 1\\
-k & 1-k\\
\end{pmatrix}.
\end{eqnarray}
A partir de esto se obtienen los valores propios para $\mathbf{x_{1}}$ que resultan
\begin{eqnarray}
\lambda_{1,2}=\frac{2+k\pm \sqrt{k^{2}+4k}}{2},
\end{eqnarray}
cuyos vectores propios $(y_{1},y_{2})$ cumplen que
\begin{eqnarray}
y_{2}=y_{1}\left(\frac{1\pm\sqrt{k^{2}+4k}}{2k}\right).
\label{vectores propios}
\end{eqnarray}
Por lo tanto, $\mathbf{x}$, es hiperbólico para cualquier $k>0$. Para $\mathbf{x}_{2}$ se tiene 
\begin{eqnarray}
\lambda_{1,2}=\frac{-k+2 \pm \sqrt{k^{2}-4k}}{2} \qquad 0<k<4,
\end{eqnarray}
que resultan ser valores complejos, por lo que para el análisis sólo se ocupará el punto $\mathbf{x_{1}}$.\\

Escribimos a las variables ($\theta,p$) como dos polinomios de variable real $t$, para encontrar la parametrización de las variedades
\begin{eqnarray}
\theta(t)=\sum_{n=0}^{\infty}a_{n}t^{n}  ,
\label{theta}
\end{eqnarray}
y
\begin{eqnarray}
p(t)=\sum_{n=0}^{\infty}b_{n}t^{n},
\label{p}
\end{eqnarray}
tal que $\mathcal{P}(t):=(\theta(t),p(t))$. Necesitamos la parametrización también de la dinámica interna $g$, para la cual usamos la ecuación $g(t)=\lambda t$ \eqref{fun g}. Después de sustituir esto en $\mathbf{f}\circ\mathcal{P}=\mathcal{P}\circ g$ \eqref{Ecua de invariancia} para el mapeo estándar obtenemos
\begin{eqnarray}
\mathbf{f}_{k}(\theta,p) = \left[\begin{array}{c}
\theta(t) + p(t) \\
p(t) + k\sin[\theta(t) +p(t)]
\end{array}\right] =\left[ \begin{array}{c}
\theta(\lambda t) \\
p(\lambda t)
\end{array}\right], 
\label{sumas en mapeo}
\end{eqnarray}
que en forma explícita es
\begin{eqnarray}
\left[\begin{array}{c}
\sum_{n=0}^{\infty}a_{n}t^{n} + \sum_{n=0}^{\infty}b_{n}t^{n} \\
\sum_{n=0}^{\infty}b_{n}t^{n} + k\sin(\sum_{n=0}^{\infty}a_{n}t^{n} + \sum_{n=0}^{\infty}b_{n}t^{n})
\end{array}\right] =\left[ \begin{array}{c}
\sum_{n=0}^{\infty}a_{n}\lambda^{n}t^{n} \\
\sum_{n=0}^{\infty}b_{n}\lambda^{n}t^{n}
\end{array}\right].
\label{expandida}
\end{eqnarray}
Desarrollando el primer renglón de la ecuación \eqref{expandida}
\begin{eqnarray}
a_{0}+a_{1}t+a_{2}t^{2}+\cdots +b_{0}+b_{1}t+b_{2}t^{2}+ \cdots=a_{0}+a_{1}\lambda t+\cdots\quad .
\label{primer renglon}
\end{eqnarray}
Agrupamos términos del mismo orden y comparamos primero los de orden cero
\begin{eqnarray}
a_{0}+b_{0}=a_{0},
\end{eqnarray}
que implica $b_{0}=0$. Hacemos lo mismo pero ahora con el renglón dos de \eqref{expandida} 
usando la serie de Taylor del seno
\begin{eqnarray}
\sum_{n=0}^{\infty}b_{n}t^{n} +k\sum_{j=0}^{\infty}\frac{(-1)^{j}}{(2j+1)!}\left[ \sum_{n=0}^{\infty}a_{n}t^{n} +\sum_{n=0}^{\infty}b_{n}t^{n}\right]^{2j+1}=\sum_{n=0}^{\infty}b_{n}\lambda^{n}t^{n}.
\label{seno exapandido 2}
\end{eqnarray}
Desarrollamos cada suma, tomando en cuenta que $b_{0}=0$ :
\begin{eqnarray}
&b_{1}t&+b_{2}t^{2}+\cdots+k\left[a_{0}+(a_{1}+b_{1})t+\cdots\right]-\nonumber\\
&\frac{k}{3!}&\left[a_{0}+(a_{1}+b_{1})t+(a_{2}+b_{2})t^{2}+\cdots\right]^{3}+\cdots\nonumber\\
&=&b_{1}\lambda t+b_{2}\lambda^{2}t^{2}+\cdots
\label{segundo renglon}
\end{eqnarray}
e igualando términos de orden cero :
\begin{eqnarray}
k a_{0}+\frac{k}{3!}a_{0}^{3}+\cdots=0, 
\end{eqnarray}
por lo que $a_{0}=0$, recordemos que $a_{0},b_{0}$ es el punto fijo. Usando los términos de orden uno en la ecuación \eqref{primer renglon}, \eqref{segundo renglon} respectivamente, obtenemos
\begin{eqnarray}
(a_{1}+b_{1})t=a_{1}\lambda t,
\end{eqnarray}

\begin{eqnarray}
b_{1}t+k(a_{1}+b_{1})t=b_{1}\lambda t.
\end{eqnarray}
Al dividir entre $t$ ambas ecuaciones podemos escribir las ecuaciones en forma matricial
\begin{eqnarray}
\begin{pmatrix}
1 & 1\\
k & 1+k
\end{pmatrix}
\begin{pmatrix}
a_{1}\\
b_{1}
\end{pmatrix}=
\lambda \begin{pmatrix}
a_{1}\\
b_{1}
\end{pmatrix}.
\end{eqnarray}
Es posible obtener las soluciones para $a_{1}$ en términos de $b_{1}$ y de $\lambda$ en términos de $k$. Este procedimiento es básicamente el análisis lineal alrededor del punto $(0,0)$. Análogamente se pueden obtener los coeficientes $a_{2},b_{2}$: tomando los términos cuadráticos de la ecuación \ref{primer renglon} y agrupando obtenemos
\begin{eqnarray}
(a_{2}+b_{2})t^{2}=a_{2}\lambda^{2}t^{2}.
\label{segundos_coeficientes_a}
\end{eqnarray}
De la misma forma tomamos los coeficientes de los términos cuadráticos en la ecuación \ref{segundo renglon} y obtenemos
\begin{eqnarray}
b_{2}t^{2}+k(a_{2}+b_{2})t^{2}=b_{2}\lambda^{2}t^{2}.
\label{segundos_coeficientes_b}
\end{eqnarray}
Al dividir \ref{segundos_coeficientes_a} y \ref{segundos_coeficientes_b} entre $t^{2}$ podemos formar el sistema matricial
\begin{eqnarray}
\begin{pmatrix}
1 & 1\\
k & 1+k
\end{pmatrix}
\begin{pmatrix}
a_{2}\\
b_{2}
\end{pmatrix}=
\lambda^{2} \begin{pmatrix}
a_{2}\\
b_{2}
\end{pmatrix}.
\end{eqnarray}

El sistema se resuelve en términos de $\lambda^{2}$ y de $k$, como en el caso de $a_{1},b_{1}$. Sin embargo, obtener los términos de esta manera es un camino tedioso, se opta entonces por encontrar relaciones de recurrencia que calculen los coeficientes de los polinomios. Usando de nuevo las ecuaciones \eqref{theta}, \eqref{p} escribimos 
\begin{eqnarray}
W(t)=\sum_{n=0}^{\infty}\beta_{n}t^{n}=\sin\left(\sum_{n=0}^{\infty}a_{n}t^{n}+\sum_{n=0}^{\infty}
b_{n}t^{n}\right),
\end{eqnarray} 
es decir la parte que aparece en el mapeo $\sin(\theta+p)$ se puede ver como un solo polinomio con coeficientes $\beta_{n}$. Al considerar de forma compleja a $W$ tenemos
\begin{eqnarray}
\overline{W}=\sum_{n=0}^{\infty}(\alpha_{n}+i\beta_{n})t^{n}=\exp[i(\theta(t)+p(t))],
\label{W compleja}
\end{eqnarray}
y calculando la derivada de la ecuación \eqref{W compleja} resulta
\begin{eqnarray}
\overline{W}'=i\overline{W}[\theta '(t)+p'(t)].
\label{W compleja deriv}
\end{eqnarray}
Al desarrollar en potencias de $t$ y usando convolución en \eqref{W compleja deriv}, se obtiene
\begin{eqnarray}
\sum_{n=0}^{\infty}(n+1)(\alpha_{n+1}+i\beta_{n+1})t^{n}=i\sum_{n=0}^{\infty}c_{n}t^{n}+i\sum_{n=0}^{\infty}d_{n}t^{n},
\end{eqnarray}
con
\begin{eqnarray}
c_{n}=\sum_{l=0}^{n}(l+1)(\alpha_{n-l}+i\beta_{n-l})a_{l+1}, \quad
d_{n}=\sum_{l=0}^{n}(l+1)(\alpha_{n-l}+i\beta_{n-l})b_{l+1}.
\end{eqnarray}
Con algo de álgebra se pueden desarrollar las sumas y separar las partes real y compleja de cada lado para compararlas, llegando a que la parte real es
\begin{eqnarray}
\sum_{n=0}^{\infty}(n+1)\alpha_{n+1}t^{n}=\sum_{n=0}^{\infty}\left[-\sum_{l=0}^{n}(l+1)\beta_{n-l}(a_{l+1}+b_{l+1})\right]t^{n},
\label{parte real}
\end{eqnarray}
mientras que la imaginaria
\begin{eqnarray}
\sum_{n=0}^{\infty}(n+1)\beta_{n+1}t^{n}=\sum_{n=0}^{\infty}\left[\sum_{l=0}^{n}(l+1)\alpha_{n-l}(a_{l+1}+b_{l+1})\right]t^{n}.
\label{parte compleja}
\end{eqnarray}
Igualando potencias de $t$ en \eqref{parte real}, \eqref{parte compleja} y despejando $\alpha_{n+1},\beta_{n+1}$ obtenemos
\begin{eqnarray}
\alpha_{n+1}=\frac{-1}{n+1}\sum_{l=0}^{n}(l+1)\beta_{n-l}(a_{l+1}+b_{l+1}),
\label{recurrencia alpha}
\end{eqnarray}
\begin{eqnarray}
\beta_{n+1}=\frac{1}{n+1}\sum_{l=0}^{n}(l+1)\alpha_{n-l}(a_{l+1}+b_{l+1}),
\label{recurrencia beta}
\end{eqnarray}
que son las relaciones de recurrencia para $\alpha,\beta$ en términos de los coeficientes del polinomio, con las que podemos calcular $\sin(\theta+p)$. Se utilizó un truco en el que fue muy importante la forma del mapeo, en el que sólo se usó una expansión en serie de Taylor; sin embargo si en el mapeo aparecieran productos de funciones, no necesariamente se podrán factorizar fácilmente los términos de cada orden.\\

Para obtener las relaciones de recurrencia de $a_{n},b_{n}$ se usó el caso $t=0$, pues ya se saben los primeros valores de las constantes $\alpha_{0}, \beta_{0}, a_{0}, b_{0}$, entonces al sustituir $t=0$ en la ecuación \eqref{W compleja} resulta
\begin{eqnarray}
\overline{W}(0)=\alpha_{0}+i\beta_{0}=\cos(\theta(0)+p(0))+i\sin(\theta(0)+p(0))=1,
\end{eqnarray}
por lo que $\alpha_{0}=1,\beta_{0}=0$. Ahora ya se tienen los valores iniciales de la recursión y por tanto se pueden calcular los otros valores. Para encontrar los demás coeficientes se usó la ecuación \eqref{expandida}, tomando en cuenta la forma en la que escribimos al seno
\begin{eqnarray}
\sum_{n=1}^{\infty}a_{n}t^{n}+\sum_{n=1}^{\infty}b_{n}t^{n}=\sum_{n=1}^{\infty}a_{n}\lambda^{n}t^{n},
\end{eqnarray}
\begin{eqnarray}
\sum_{n=1}^{\infty}b_{n}t^{n}+k\sum_{n=1}^{\infty}\beta_{n}t^{n}=\sum_{n=1}^{\infty}b_{n}
\lambda^{n}t^{n}.
\end{eqnarray}
Se reescriben las ecuaciones anteriores, para comparar términos de la misma potencia
\begin{eqnarray}
\sum_{n=1}^{\infty}(1-\lambda^{n})a_{n}t^{n}=-\sum_{n=1}^{\infty}b_{n}t^{n},
\end{eqnarray}
\begin{eqnarray}
\sum_{n=1}^{\infty}(1-\lambda^{n})b_{n}t^{n}=-k\sum_{n=1}^{\infty}\beta_{n}t^{n},
\end{eqnarray}
entonces los coeficientes de $t^{n+1}$ son
\begin{eqnarray}
(1-\lambda^{n+1})a_{n+1}=-b_{n+1},
\label{coeficiente recursion 1}
\end{eqnarray}
\begin{eqnarray}
(1-\lambda^{n+1})b_{n+1}=-k\beta_{n+1}.
\label{coeficiente recursion 2}
\end{eqnarray}
Sustituyendo \eqref{recurrencia beta} en  \eqref{coeficiente recursion 2}
\begin{eqnarray}
(1-\lambda^{n+1})b_{n+1}=\frac{-k}{n+1}\sum_{l=0}^{n}(l+1)\alpha_{n-l}(a_{l+1}+b_{l+1}).
\label{triangulo}
\end{eqnarray}
Como se busca una ecuación para la recurrencia, se separa el término $l=n$ del lado derecho de \eqref{triangulo}
\begin{eqnarray}
(1-\lambda^{n+1})b_{n+1}=-\frac{k}{n+1}\sum_{l=0}^{n-1}(l+1)\alpha_{n-l}(a_{l+1}+b_{l+1})-k(a_{n+1}+b_{n+1}),
\label{triangulo1}
\end{eqnarray}
y agrupando de manera que los coeficientes $a_{n+1},b_{n+1}$ queden en el mismo lado de la ecuación
\begin{eqnarray}
k a_{n+1}+(1-\lambda^{n+1}+k)b_{n+1}=-\frac{k}{n+1}\sum_{l=0}^{n-1}(l+1)\alpha_{n-l}
(a_{l+1}+b_{l+1}).
\label{triangulo2}
\end{eqnarray}
Usando las ecuaciones \eqref{coeficiente recursion 1} y \eqref{triangulo2} se escribe un sistema de ecuaciones para $a_{n+1},b_{n+1}$ en forma matricial:
\begin{eqnarray}
\mathbf{A}\begin{pmatrix}
a_{n+1}\\
b_{n+1}
\end{pmatrix}=-\frac{k}{n+1}\sum_{l=0}^{n-1}(l+1)\alpha_{n-l}(a_{l+1}+b_{l+1})\begin{pmatrix}
0\\
1
\end{pmatrix},
\label{sistema recurrencia}
\end{eqnarray}
siendo 
\begin{eqnarray}
\mathbf{A}=\begin{pmatrix}
a-\lambda^{n+1} & 1 \\
k & 1-\lambda^{n+1}+k
\end{pmatrix}.
\end{eqnarray}
Escrito de esta forma es claro que el sistema se resulve multiplicando por $\mathbf{A}^{-1}$, siempre que $\det(\mathbf{A})\neq 0$ :
\begin{eqnarray}
\begin{pmatrix}
a_{n+1}\\
b_{n+1}
\end{pmatrix}=-\frac{k}{n+1}\sum_{l=0}^{n-1}\alpha_{n-l}(a_{l+1}+b_{l+1})\mathbf{A}^{-1}\begin{pmatrix}
0\\
1
\end{pmatrix},
\label{Sistema recurrencia}
\end{eqnarray}
siendo
\begin{eqnarray}
\mathbf{A}^{-1}=\frac{1}{(1-\lambda^{n+1})(1-\lambda^{n+1}-k)-k}\begin{pmatrix}
1-\lambda^{n+1}+k & -1\\
-k & 1-\lambda^{n+1}
\end{pmatrix}.
\end{eqnarray}
Al escribir de manera separada la ecuación \eqref{Sistema recurrencia} se obtienen las relaciones de recurrencia para los coeficientes de la parametrización
\begin{eqnarray}
a_{n+1}=\frac{k}{(n+1)[(1-\lambda^{n+1})(1-\lambda^{n+1}+k)-k]}\sum_{l=0}^{n-1}\alpha_{n-l}(l+1)(a_{l+1}+b_{l+1}),
\end{eqnarray}
\begin{eqnarray}
b_{n+1}=\frac{-k 1-\lambda^{n+1}}{(n+1)[(1-\lambda^{n+1})(1-\lambda^{n+1}+k)-k]}\sum_{l=0}^{n-1}\alpha_{n-l}(l+1)(a_{l+1}+b_{l+1}).
\end{eqnarray}
Usando cada uno de los valores propios y las anteriores ecuaciones de recurrencia se obtienen los coeficientes de los polinomios $\theta(t),p(t)$ a cualquier orden. Dependiendo de qué valor de $\lambda$ se escoja, se obtiene la parametrización de la variedad estable o de la inestable.\\

%El ejemplo del cálculo de los polinomios mediante las relaciones de recurrencia se encuenta en la siguiente liga \url{}





\section{Implementación del método}
En esta sección se explica paso a paso cómo se implementó el método. Su\-pon\-dre\-mos que se tiene un mapeo Hamiltoniano $\mathbf{f}_{k}(\mathbf{x})$ donde $k$  es un parámetro, del cual se tiene un punto fijo $\mathbf{x}_{*}=(\theta_{*},p_{*})$. En la siguiente liga se encuentra el archivo llamado \texttt{Implementación.ipynb} que contiene el ejemplo de cómo se aplica el método paso a paso para el mapeo estándar, \url{https://github.com/alvarezeve/Tesis-Variedades-Estables-e-inestables/}. 
\linebreak


\begin{center}
Primer orden
\begin{tabbing}
12\=1234567890123456789012345678901234567890123456\=12345678901234567890123456\kill%
\>............................................................  \>..................................................\\
\>\textbf{1.} Se crean dos variables $\theta,p$ del mapeo \> \\
\>como dos polinomios de grado mayor \>$\mathbf{x}_{1}=(\theta+\cdots ,p+\cdots)$  \\
\>a uno que corresponden a $\mathbf{x}_{1}$ en \eqref{sistema discreto}.  \>   \\
\>............................................................  \>..................................................\\
\>\textbf{2.} Se crean dos polinomios de variable\> \\
\>$t$ de orden uno que representan la va-   \> $\mathcal{P}_{\theta}=\theta_{*}+(\theta+\cdots)t+O(t^{2})$\\
\>riedad. Los coeficientes de orden cero \> $ \mathcal{P}_{p}=p_{*}+(p+\cdots)t+O(t^{2})$\\
\>son el punto fijo. \> \\
\>............................................................  \>..................................................\\
\>\textbf{3.} Se aplica el mapeo $\mathbf{f}_{k}$ a los polino- \> \\
\>mios  anteriores, lo cual corresponde al  \>$C_{1}=\mathbf{f}_{k}(\mathcal{P}_{\theta},\mathcal{P}_{p})$  \\
\>lado izquierdo de \eqref{Ecua de invariancia}. \> \\
\>............................................................  \>..................................................\\
\end{tabbing} 


\end{center}
Hasta aquí se tiene calculada la parte izquierda de la ecuación de invariancia; el lado derecho se retoma más adelante. La razón por la que se escriben los coeficientes de $\mathcal{P}=(\mathcal{P}_{\theta},\mathcal{P}_{p})$ a su vez como polinomios, es que al escribir un polinomio en el coeficiente es posible tratarlo como una variable. Es decir la $\theta$ en $\mathcal{P}_{\theta}=\theta_{*}+(\theta+\Delta \theta)t$ representa la incógnita del coeficiente de orden uno. Para encontrar el primer orden de los polinomios $\mathcal{P}_{\theta},\mathcal{P}_{p}$ se escribe todo en forma matricial:
\begin{eqnarray}
\mathbf{A}\mathbf{v}=\mathbf{w},
\label{lineal A}
\end{eqnarray}
donde la matriz $\mathbf{A}$ contiene a los coeficientes de  orden $n=1$  de $\mathcal{P}$, mientras que $\textbf{v}=(a_{1},b_{1})$ y $\textbf{w}$ tiene los términos independientes de $\mathcal{P}$. 
\begin{center}

  
\begin{tabbing}
12\=34567890123456789012345678901234567890123456\=7890123456789012345678901234567890\kill%
\>............................................................  \>..................................................\\
\>\textbf{4.} La matriz $\mathbf{A}$ se calcula con el jaco- \>  \\
\>biano de $\mathcal{P}_{n}$, permitiendo obtener los \> $\mathbf{A}=\mathbf{J}(\mathbf{f}_{k}(\mathcal{P}_{\theta},\mathcal{P}{p}))$  \\
\>coeficientes de orden uno.   \> \\

\>............................................................  \>..................................................\\
\>............................................................  \>..................................................\\
\>\textbf{5.} Se calculan los valores y vectores  \> $[\lambda_{1},\lambda_{2}]$\\
\>propios de $\mathbf{A}$.  \> $[\mathbf{v_{1}},\mathbf{v_{2}}]$\\
\>............................................................  \>..................................................\\
\>\textbf{6.} Se elige el valor y vector propio aso-\> $\lambda_{2},\mathbf{v_{2}}=(a_{1},b_{1})$\\
\>ciados a la variedad buscada. \> \\

\>............................................................  \>..................................................\\
\end{tabbing}
\end{center}
Los valores de $\mathbf{v_{2}}$ serán los coeficientes de orden uno en los polinomios $\mathcal{P}_{\theta},\mathcal{P}_{p}$, que acompañan a $t$. Al ser los vectores propios proporcionan una dirección tangente a la variedad, que es justo la manera en la que se implementa el método usual.\\

Como se está usando el método gráfico es necesaria una forma polinomial para $g$ y la forma más simple es usar \ref{fun g} con $\lambda_{2}$, $g(t)=(\lambda_{2}t,\lambda_{2}t)$. Además recuerde que nuestro sistema está linealizado para analizarlo y la matriz asociada a la linealización es justo la que contiene los vectores propios como columnas.\\

\begin{center}
Segundo orden
\begin{tabbing}
12\=34567890123456789012345678901234567890123456\=7890123456789012345678901234567890\kill%
\>............................................................  \>..................................................\\
\>\textbf{7.} Se actualizan los coeficientes en los \> $\mathcal{P}_{\theta}=\theta_{*}+a_{1}t$\\
\>polinomios.\> $\mathcal{P}_{p}=p_{*}+b_{1}t$\\
\>............................................................  \>..................................................\\
\>\textbf{8.} Se agregan las variables $\theta,p$ para \> $\mathcal{P}_{\theta}=\theta_{*}+a_{1}t+(\theta)t^{2}+O(t^{3})$ \\
\>calcular el término cuadrático. \> $\mathcal{P}_{p}=p{*}+b_{1}t+(p)t^{2}+O(t^{3})$\\
\>............................................................  \>..................................................\\
\>\textbf{9.} Se aplica el mapeo\> $C_{2}=\mathbf{f}_{k}(\mathcal{P}_{\theta},\mathcal{P}_{p})$\\
\>............................................................  \>..................................................\\
\end{tabbing}
\end{center}
Retomando el lado derecho de la ecuación de invariancia \eqref{Ecua de invariancia}, para el cual se tiene un polinomio con coeficientes $a_{i}$ multiplicados por una potencia del valor propio.
\begin{eqnarray}
a_{0}+a_{1}\lambda t+a_{2}\lambda^{2}t^{2}\\
b_{0}+b_{1}\lambda t+b_{2}\lambda^{2}t^{2}
\end{eqnarray}
\begin{center}


\begin{tabbing}
12\=34567890123456789012345678901234567890123456\=7890123456789012345678901234567890\kill%
\>............................................................  \>..................................................\\
\>\textbf{10.} Se escribe el lado derecho de la \> $\mathcal{P}_{\theta\lambda} = \theta_{*}+a_{1}\lambda t +\theta\lambda^{2}t^{2}+O(t^{3}) $\\
\>ecuación \eqref{Ecua de invariancia} como polinomios en $t$ \>$ \mathcal{P}_{p\lambda} = p_{*}+b_{1}\lambda t +p\lambda^{2}t^{2}+O(t^{3}) $\\ 
\>............................................................  \>..................................................\\
\end{tabbing}
\end{center}

Ahora que se tienen las dos partes de la ecuación \eqref{Ecua de invariancia} para el orden 2 se puede resolver.
\begin{tabbing}
12\=34567890123456789012345678901234567890123456\=7890123456789012345678901234567890\kill%
\>............................................................  \>..................................................\\
\>\textbf{11.} Se define una ecuación que será la  \> \\
\>resta de ambos lados de la expresión \> $R :=C_{2}-P_{\lambda}=\mathbf{0} $\\
\>\eqref{Ecua de invariancia} igualada a cero. Con tal condi-\> $h_{\theta}(\theta,p)t^{2}=(C_{2\theta}-\theta\lambda^{2})t^{2} $\\
\>ción el término de orden dos cumple\> $h_{p}(\theta,p)t^{2}=(C_{2p}-p\lambda^{2})t^{2}$\\
\>una ecuación lineal inhomogénea, en \> \\
\>donde la matriz se obtiene calculando \> $\mathbf{A_{2}}=\mathbf{J}(h_{\theta},h_{p})$\\ 
\>el jacobiano.\>\\
\>............................................................  \>..................................................\\
\>\textbf{12.} Acomodando los valores indepen-\> \\
\>dientes de $h_{\theta},h_{p}$ en un vector se obtie- \>$\mathbf{w}_{2}=(c_{\theta},c_{p})$ \\
\>ne $\mathbf{w}_{2}$. \> \\
\>............................................................  \>..................................................\\
\>\textbf{13.} El sistema se escribe en forma ma- \> \\
\>tricial \eqref{lineal A} y se resuelve multiplicando \> $\mathbf{v}_{2}=\mathbf{A}_{2}^{-1}\mathbf{w}_{2}$\\ 
\>por la inversa del lado izquierdo  \>\\
\>............................................................  \>..................................................\\
\>\textbf{14.} El resultado de esta ecuación serán \> $\mathbf{v}_{2}=(a_{2},b_{2})$ \\
\>los coeficientes cuadráticos de $\mathcal{P}$, es decir, \>$\mathcal{P}_{\theta}=\theta_{*}+a_{1}t+a_{2}t^{2}+O(t^{3})$ \\
\>$a_{2},b_{2}$.\> $\mathcal{P}_{p}=p_{*}+b_{1}t+b_{2}t^{2}+O(t^{3})$ \\
\>............................................................  \>..................................................\\
\end{tabbing}

La manera de proceder con el cálculo de los coeficientes de orden cúbico es la misma que la de orden cuadrático. En cada orden $n$ aparecerá la dependencia de $\lambda^{n}$ debida al lado derecho de la ecuación de invariancia y a la forma de la función $g$. En general una vez actualizados los valores $a_{n},b_{n}$ se agrega un orden más a los polinomios $\mathcal{P}_{\theta},\mathcal{P}_{p}$ así como a los de $\mathcal{P}_{\theta\lambda},\mathcal{P}_{p\lambda}$ en términos de las variables $\theta$ y $p$, se aplica el mapeo a los primeros y se escribe la resta igualada a cero de la ecuación \eqref{Ecua de invariancia}. Calculando el Jacobiano se obtiene la matriz del sistema $\mathbf{A}_{n+1}$ y con los términos independientes $\mathbf{w}_{n+1}$. Se resuelve el sistema mediante la inversa de $\mathbf{A}_{n}$ y se obtienen ahora los términos $a_{n+1},b_{n+1}$.\\

Sólo el primer orden es el que difiere en la forma del cálculo, ya que en el primer paso se necesitan los valores y vectores propios. Salvo esos primeros términos los otros se pueden resumir en un sólo procedimiento. Tales características fueron las que permitieron automatizar el método. Las diferencias que surgen al resolver la ecuación lineal se toman en cuenta en el cálculo, así como el error que se va acumulando en cada paso. \\

Al tener la parametrización $\mathcal{P}$ hasta cierto orden $n$ es necesario calcular el error cometido al evaluar $t$. Teniendo en mente que los polinomios son desarrollos en series de Taylor alrededor del punto fijo, nuestra parametrización es válida sólo en una vecindad cercana. Como ya se vio el error se calcula mediante \eqref{Ecua de invariancia resta}. Con $\mathcal{P}=(\mathcal{P}_{\theta},\mathcal{P}_{p})$, se procede como a continuación.

\begin{tabbing}
12\=34567890123456789012345678901234567890123456\=7890123456789012345678901234567890\kill%
\>............................................................  \>..................................................\\
\>\textbf{I}. Se aplica el mapeo a $\mathcal{P}$.\> $\mathbf{S}=\mathbf{f}_{k}(\mathcal{P})$ \\
\> ............................................................ \>............................................\\
\>\textbf{II}. Se construyen los polinomios $\mathcal{P}_{\lambda}$\> $\mathcal{P}_{\lambda}=(P_{\theta\lambda},P_{p\lambda})$\\
\> ............................................................ \>............................................\\
\>\textbf{III}. Se usa la ecuación \eqref{Ecua de invariancia resta}.\> $\mathbf{E} =\mathbf{S}-\mathcal{P}_{\lambda}$\\
\> ............................................................ \>............................................\\

\end{tabbing}
El error será un conjunto de valores que resulten de evaluar la función \eqref{Ecua de invariancia resta} para un conjunto $\tau = ( t_{0},t_{1},..., t_{n} )$. \\

Usando este procedimiento se automatizó el método, sin necesitar las ecuaciones de recurrencia explícitamente, ya que mediante la manipulación algebraica de las series de Taylor se calcula fácilmente los nuevos términos de la parametrización. En general el método se desarrolló para las variedades inestables, ya que la misma dinámica de tal variedad permite llegar más lejos en la evaluación; tanto de los coeficientes como del parámetro $t$, garantizando una mejor aproximación. La manera en la que se calculan las variedades estables es en esencia la misma, escogiendo el vector y valor propio adecuado se puede hacer el mismo análisis para la inestable. Hacerlo de esta forma no será lo más conveniente, mantenerse en la variedad estable será numéricamente inestable debido a los errores de truncamiento y redondeo, que llevarán a caer en la dinámica inestable del sistema. La forma más adecuada será calcular la variedad estable usando el mismo método para la variedad inestable del mapeo inverso. \\



Con esto se completa la automatización del método; el código junto con la do\-cu\-men\-ta\-ción de cómo usar el programa y algunos ejemplos se encuentran en \url{https://github.com/alvarezeve/}. 






   

%%%%%%%%%%%%%%%%%%%%%%%%%%%%%%%%%%%%%%%%%%%%%%%%%%%%%%%%%%%%%%%%%%%%%%%%%
%           Capítulo 3: Mapeo de Henon
%%%%%%%%%%%%%%%%%%%%%%%%%%%%%%%%%%%%%%%%%%%%%%%%%%%%%%%%%%%%%%%%%%%%%%%%%
\chapter{Ejemplos de aplicación del método}
\label{SeccionEstandar}\section{Mapeo Estándar}
En el capítulo anterior ya mostramos cómo se aplica el método de manera algebráica para el caso de este mapeo. Utilizando el método ya programado se hicieron diferentes cálculos para comparar con los resultados presentados en \citep{Mireles}. Una de las razones de estudiar el mapeo estándar, además de usarlo como una forma de validación, es porque del mapeo conocemos muchas cosas. Por otro lado queremos mostrar lo importante que es tener una parametrización analítica. Aunque el estudio cualitativo del mapeo puede darnos información útil, tener una parametrización de las variedades relacionadas a sus puntos fijos convierte el análisis en algo cuantitativo y semianalítico. El objetivo de esta sección es mostrar algunas de las cosas que son posibles alcanzar en términos de este análisis, además de la forma en la que se usa el método desde Julia.\\

En el mapeo estándar \ref{mapeo estandar} uno de los puntos fijos es el origen de coordenadas $\mathbf{x}_{1}=(0,0)$. Utilizando el método programado se calcularon las variedades estables e inestables para diferentes valores del parámetro en el mapeo. El objetivo de hacer éstas fue reproducir los resultados de J.D. Mireles que presenta en sus notas \cite{Mireles}. En tales notas no aparece el orden del polinomio ni el error específico, sin embargo se intentó reproducir al menos gráficamente los resultados. Dependiendo del orden del polinomio que se calcule y del parámetro del mapeo se podrá llegar más lejos del punto fijo.  
\begin{figure}[H]
 \centering
 \includegraphics[scale=0.6]{estandark03}
 \caption{$W^{s},W^{u}$ de oden $25$ en el mapeo estándar con $\kappa=0.3$ y $t_{max}=3.0$.}
 \label{estandar03}
\end{figure}

\begin{figure}[H]
\centering
\includegraphics[scale=0.6]{error_est_k03} 
\caption{Error en las variedades de la figura \ref{estandar03}.}
\label{error est k03}
\end{figure}


\begin{figure}[H]
\centering
\includegraphics[scale=0.6]{estandark15}
\caption{$W^{s},W^{u}$ de orden $80$ en el mapeo estándar con $\kappa=1.5$ y $t_{max}=13.0$.}
\label{estandar15}
\end{figure}

\begin{figure}[H]
\centering
\includegraphics[scale=0.6]{error_est_k15} 
\caption{Error en las variedades de la figura \ref{estandar15}.}
\label{error est k15}
\end{figure}




\begin{figure}[H]
\centering
\includegraphics[scale=0.6]{estandark07}
\caption{$W^{s},W^{u}$ de orden 70 en el mapeo estándar con $\kappa=0.7$ y $t_{max}=5.5$.}
\label{estandar07}
\end{figure}

\begin{figure}[H]
\centering
\includegraphics[scale=0.6]{error_est_k07} 
\caption{Error en las variedades de la figura \ref{estandar07}.}
\label{error est k07}
\end{figure}


En las figuras \ref{estandar03}-\ref{error est k07} se muestran los resultados de la parametrización de las variedades estable e inestable asociadas al punto fijo $\mathbf{x}_{1}$ para diferentes valores del parámetro $\kappa$, junto con cada una aparece su respectiva gráfica del error numérico.   Los cálculos se hicieron utilizando números de punto flotante de 64 bits(Float64). Para las figuras \ref{estandar03}, \ref{estandar07} se puede ver que las variedades se juntan de manera que parecen ser tangentes, mientras que para el caso de la figura \ref{estandar15} observamos varias intersecciones entre las variedades. En todos los casos el error se comporta de manera similar, manteniéndose prácticamente constante hasta cierto valor del parámetro $t$ y creciendo de forma exponencial después del mismo. La curva será entonces confiable hasta valores del parámetro que no excedan el punto donde el error crece rápidamente.  \\


Para observar como cambiaba el comportamiento del error respecto del orden de la parametrización se calcularon polinomios de diferente orden que parametrizan a la variedad inestable del mapeo con $\kappa=0.3$, el resultado se muestra en la figura \ref{erroresf64}. Observamos que mientras más grande sea el orden del polinomio mejor es la aproximación, pues podemos llegar a valores del parámetro más grandes, que se traduce en ir más lejos en la variedad inestable. \\

\begin{figure}[H]
\centering
\includegraphics[scale=0.6]{error_estandar_orden}
\caption{Curvas de error para diferentes órdenes en el mapeo estándar, $\kappa=0.3$. }
\label{erroresf64}
\end{figure}
A fin de mostrar que el error es de alguna manera controlable se usaron números de precisión extendida para hacer cálculos análogos a los anteriores. En la figura \ref{erroresBig} se muestran los resultados para parametrizaciones de ordenes entre $10$ y $80$. Observamos un comportamiento análogo, de tal manera que para cada orden diferente de parametrización hay un valor diferente del parámetro en el cual pasa de un error que no crece significativamente a un error que crece de manera abrupta. También notamos que al usar precisión extendida el error cerca del punto fijo es imperceptible pero en la parte donde crece, tiene un crecimiento más pronunciado que en el caso de números de punto flotante. 

\begin{figure}[H]
\centering
\includegraphics[scale=0.6]{error_estandar_orden_big}
\caption{Curvas de error para diferentes órdenes usando precisión extendida ,$\kappa=0.3$. }
\label{erroresBig}
\end{figure}

Como ya mencionamos antes la variedad inestable del mapeo inverso corresponde a la variedad estable del mapeo, si se usa el mismo método calculando la variedad inestable del mapeo inverso \ref{mapeo estandar inverso} podemos controlar mejor el error numérico. Para mostrar esto hicimos una comparación parametrizando la variedad estable mediante el mapeo inverso y el mapeo inicial. Los polinomios fueron del mismo orden y lo que se observó en el error se muestra en la figura \ref{erroresinverso}.


\begin{figure}[H]
\centering
\includegraphics[scale=0.6]{mapeoinver}
\caption{Error para las parametrizaciones usando el mapeo y el mapeo inverso con polinomios de orden $20$ y $\kappa=0.3$. }
\label{erroresinverso}
\end{figure}

Usar el mapeo inverso para calcular la variedad estable resulta ser mejor que usar el mapeo normal. El error se mantiene casi sin cambios hasta una valor de $t$ mayor en el caso del mapeo inverso. 




\section{Mapeo de Hénon}
El mapeo de Hénon se define como \cite{devaney}
\begin{eqnarray}
\mathbf{f}_{a,b}(x,y)=\left( \begin{array}{lcc}
             a-by-x^{2}\\
             \\ x
             \end{array}
             \right), \label{Henon}
\end{eqnarray}

siendo el mapeo inverso
\begin{eqnarray}
\mathbf{f}^{-1}_{a,b}(x,y)=\left( \begin{array}{lcc}
             y\\
             \\ (x+y^{2}-a)/-b
             \end{array}.
             \right) \label{HenonI}
\end{eqnarray} 

       
Para poder analizarlo debemos linearizar el sistema. Primero obtenemos el jacobiano 
            
\begin{eqnarray}
D\mathbf{f}_{a,b}(x,y)= \left( \begin{array}{lcc}
                -2x & -b\\
                \\ 1 & 0
                \end{array}
                \right).
\end{eqnarray}

                
Notamos que el determinante del jacobiano no es igual a uno sino $\det(D\mathbf{f}_{a,b}(x,y))=b$.
El determinante es constante, entonces será Hamiltoniano en el caso en que $b$ sea igual a uno o menos uno. Analizaremos estos casos, encontrando los puntos fijos
\begin{eqnarray}
\mathbf{f}_{a,b}(x,y)=\left( \begin{array}{lcc}
               a-by-x^{2}\\
               \\ x
               \end{array}
               \right) = \left(\begin{array}{lc}
               x \\
               \\ y
               \end{array}
               \right),
\end{eqnarray}
              
lo que implica que $a-by-x^{2}=x$ y $x=y$ de donde es claro que la primer ecuación queda
\begin{eqnarray*}
x^{2}+(b+1)x-a=0, 
\end{eqnarray*}
que se puede resolver usando la fórmula general
\begin{eqnarray*}
x=\frac{-(b+1)\pm ((b+1)^{2}+4a)^{1/2} }{2},
\end{eqnarray*}
para el caso en que $b=1$ se tiene
\begin{eqnarray}
x=\frac{-2\pm 2(1+a)^{1/2} }{2}.
\end{eqnarray}
Al escoger $b=1$ garantizamos estar en un sistema Hamiltoniano, mientras que $a$ debe escogerse de manera que resulten puntos fijos hiperbólicos. La figura \ref{Henon1} muestra un ejemplo en cálculos de variedades para el mapeo de Hénon, junto con el error. 
\begin{figure}[H]
\centering
\includegraphics[scale=0.6]{henon1}
\caption{$W^{u}$ y $W^{s}$ de orden 45 con $t_{max}=1000.0$ para el mapeo de Hénon con $a=1.5$,$b=1$.}
\label{Henon1}
\end{figure}

\begin{figure}[H]
\centering
\includegraphics[scale=0.6]{ErrorHenon1}
\caption{Error asociado a las variedades de la figura \ref{Henon1}.}
\label{ErrorHenon1}
\end{figure}
Las variedades que aparecen en la figura \ref{Henon1} fueron calculadas con el mapeo de Hénon inicial \ref{Henon}, las variedades se observan simétricas, aún así las parametrizaciones son diferentes. Puede verse en el error, figura \ref{ErrorHenon1}, que la variedad estable tiene un mayor error que la variedad inestable, mientras en la inestable el error cambia cinco órdenes de magnitud, en todo el intervalo del parámetro, el error de la variedad estable cambia en al menos quince órdenes de magnitud. \\

A diferencia del mapeo estándar en éste mapeo las órbitas pueden irse al infinito, es decir no están  constreñidas en una sección fija, lo que representa un mayor reto en cuanto a la parametrización ya que el polinomio debe ser tal que pueda regresar varias veces. De hecho podemos observar que se necesitan valores realmente grandes, comparados con los del mapeo estándar, para observar los cruces de las variedades. También esta situación hace que el error numérico sea mayor que para el estándar. \\

En las figuras \ref{Henon2}, \ref{Henon3} se muestran las variedades calculadas de la misma manera en la que se calcularon para la figura \ref{Henon1}. En \ref{Henon2} las curvas son más cerradas y se necesita de un polinomio de orden mayor que en el caso de \ref{Henon3} para observar los cortes. 
\begin{figure}[H]
\centering
\includegraphics[scale=0.6]{henon2}
\caption{$W^{u}$, $W^{s}$ de orden $50$ y $t_{max}=800.0$ para el mapeo de Hénon con $a=0.7$,$b=1.$.}
\label{Henon2}
\end{figure}

\begin{figure}[H]
\centering
\includegraphics[scale=0.6]{ErrorHenon2}
\caption{Error asociado a las variedades de la figura \ref{Henon2}.}
\label{Henon2}
\end{figure}


\begin{figure}[H]
\centering
\includegraphics[scale=0.6]{henon3}
\caption{$W^{u}$, $W^{s}$ de orden $78$ y $t_{max}=4000.0$ para el mapeo de Hénon con $a=6.5$,$b=1.$.}
\label{Henon3}
\end{figure}

\begin{figure}[H]
\centering
\includegraphics[scale=0.6]{ErrorHenon3}
\caption{Error asociado a las variedades de la figura \ref{Henon3}.}
\label{Henon3}
\end{figure}


Así que con el orden suficiente es posible observar los cruces de ambas variedades, retomaremos esto en la última sección. 

\section{Mapeo exponencial}
El último mapeo que se estudió fue uno presentado en el artículo \citep{Jung} el cual queda definido por 
\begin{eqnarray}
\mathbf{j}_{a}(x,y)=\left(\begin{array}{lcc}
             x+y\\
             \\ y+af(x+y)
             \end{array}\right)
\label{Jung}
\end{eqnarray}

que describe el movimiento de una partícula pateada, donde la coordenada $x$ representa la posición en una dimensión mientras que la coordenada $y$ es el momento, $a$ es un parámetro libre. La función $f$ es la responsable de describir la fuerza aplicada, en el artículo \cite{Jung} se escoge
\begin{eqnarray*}
f(x)=x(x-1)e^{-x}.
\end{eqnarray*}
A este mapeo le corresponde su mapeo inverso
\begin{eqnarray}
\mathbf{j}^{-1}_{a}(x,y)=\left(\begin{array}{lcc}
             x-y+ax(x-1)e^{-x}\\
             \\ y-ax(x-1)e^{-x}
             \end{array}\right).
             \label{jungI}
\end{eqnarray}
Los puntos fijos encontrados del sistema son $\mathbf{x}_{0}=(1,0), \mathbf{x}_{1}=(0,0)$ donde $\mathbf{x}_{0}$ es un punto fijo hiperbólico. El punto $\mathbf{x}_{1}$ es un punto elíptico mientras el valor del parámetro $a$ sea menor a 4, para valores de $a \geq 4$ se torna inverso hiperbólico.\\

Aplicando el mismo mecanismo que en los casos pasados se obtuvieron las figuras \ref{jung1}-\ref{errorjung2} que muestran cómo se comportan las variedades aún en el caso en que el sistema no sea completamente hiperbólico. Como en los casos anteriores el error asociado a la variedad estable es mayor que el asociado a la inestable. El orden al que se debe llegar en los polinomios para observar algunos de los cortes de las variedades es más alto en comparación con el mapeo de Hénon debido a que en este caso se esta aproximando una función exponencial.

\begin{figure}[H]
\centering
\includegraphics[scale=0.6]{jung34}
\caption{$W^{s}$ de orden 93, $W^{u}$ de orden 86 con $t_{max}=5.5$, con $a=3.4$ en el punto fijo $x_{0}$.}
\label{jung1}
\end{figure}


\begin{figure}[H]
\centering
\includegraphics[scale=0.6]{error_jung34}
\caption{Error asociado a las variedades de la figura \ref{jung1}.}
\label{errorjung1}
\end{figure}




\begin{figure}[H]
\centering
\includegraphics[scale=0.6]{jung57}
\caption{$W^{s}$ de orden 90, $W^{u}$ de orden 87 con $t_{max}=6.5$, con $a=5.7$ en el punto fijo $x_{0}$.}
\label{jung2}
\end{figure}


\begin{figure}[H]
\centering
\includegraphics[scale=0.6]{error_jung57}
\caption{Error asociado las variedades de la figura \ref{jung2}.}
\label{errorjung2}
\end{figure}
Se puede observar en las figuras \ref{errorjung1}, \ref{errorjung2} que en la variedad estable el error crece de manera abrupta antes que el error de la variedad inestable. Ambas curvas del error tienen la misma forma, pero es claro que no se puede llegar tan lejos en la variedad estable. El mapeo es más sensible que los dos mapeos pasados, algunos órdenes resultaban no ajustarse a la variedad más allá de valores del parámetro menor a uno. 


\section{Convergencia}
Además de medir el error asociado a la parametrización consideramos importante tomar en cuenta la convergencia de los coeficientes de los polinomios. En casos como el mapeo exponencial en que las variedades se acercan a puntos fijos de diferente naturaleza puede ocurrir que tal cercanía afecte la forma de parametrización. Para ello se implementaron dos formas de revisar la convergencia, la de Hadamard \ref{hadamard} y la de tres términos \ref{tres terminos}. \\

La figura \ref{convergenciaEst15} muestra la convergencia de los polinomios de orden 25 que parametrizan la variable $\theta$ en el mapeo estándar para las variedades estable e inestable con $\kappa=1.5$ a la que corresponde el espacio fase mostrado en \ref{estandar15}. En el caso de la variedad estable se ve que los coeficientes cambian de manera abrupta al principio pero después de cierta $n$ el cociente es casi cero. Para el caso de la variedad inestable los coeficientes cambian de manera suave y el cociente también se acerca a cero. En ambos casos podemos decir que los coeficientes de la parametrización convergen a cero.  
\begin{figure}[H]
\centering
\includegraphics[scale=0.5]{converEst15}
\caption{Convergencia de Hadamard asociada a los polinomios para $\theta$ en las variedades del mapeo estándar con $\kappa=1.5$.}
\label{convergenciaEst15}
\end{figure}

La figura \ref{convergenciaHenon1} muestra la convergencia de las parametrizaciones de orden 45 para la variable $x$ en el mapeo de Hénon con $a=1.5$. En ambos casos la convergencia parece suave y tiende a cero igual que en el caso del mapeo estándar.

\begin{figure}[H]
\centering
\includegraphics[scale=0.5]{converHenon1}
\caption{Convergencia de Hadamard asociada a los polinomios para $x$ en las variedades del mapeo de Hénon con $a=1.5$.}
\label{convergenciaHenon1}
\end{figure}


Para el mapeo exponencial realizamos los dos criterios de convergencia. La figura \ref{convergenciaJH} muestra el criterio de Hadamard \ref{hadamard}, para los polinomios que parametrizan la variable x en cada variedad, se ve que hay una convergencia clara de los coeficientes. Lo mismo ocurre en la figura \ref{convergenciaJ3} en dónde se usó el criterio de tres términos, \ref{tres terminos}.

\begin{figure}[H]
\centering
\includegraphics[scale=0.5]{convergenciaJungH57}
\caption{Convergencia de Hadamard asociada a las variedades mostradas en la figura \ref{jung2}.}
\label{convergenciaJH}
\end{figure}


\begin{figure}[H]
\centering
\includegraphics[scale=0.5]{convergenciaJungT57}
\caption{Convergencia de tres términos asociada a las variedades en la figura \ref{jung2}.}
\label{convergenciaJ3}
\end{figure}

Estudiar la convergencia de las parametrizaciones puede dar una idea de cómo se va modificando el polinomio si se cambia el orden, en mapeos más sensibles puede ser determinante para saber a qué orden es conveniente aproximar. 

\section{Existencia de puntos homoclínicos/heteroclínicos}

Siendo que el resultado son polinomios se puede aplicar el método de Newton en dos dimensiones o cualquier otro para resolver $P^{u}=P^{s}$ y encontrar puntos homoclínicos o heteroclínicos. Por suerte existe una paquetería en Julia que hace cálculos numéricos validados \texttt{ValidatedNumerics}\cite{validated} dentro de la cuál se tiene un paquete para aritmética de intervalos \texttt{IntervalArithmetic}\citep{interval} y para encontrar raíces \texttt{IntervalRootFinding}\cite{root} además del paquete \texttt{Static Arrays}\cite{static} que permite manipular intervalos.\\ 

El paquete \cite{validated} hace cálculos de computación rigurosa con números de punto flotante usando el paquete de aritmética de intervalos, que efectúa operaciones con intervalos en lugar de números. Mientras que \cite{root} automatiza el métodos populares como el de Newton para encontrar raíces de funciones, en este caso se garantiza que la respuesta correcta se encuentra en un intervalo. Para entender mejor cómo funcionan cada una de las paqueterías así como la teoría rigurosa detrás de éstos recomendamos revisar las lecturas \cite{ramon},\cite{Numerics}. De manera breve podemos decir que los paquetes ya mencionados generalizan las operaciones y funciones ahora en términos de conjuntos, de tal manera que que la solución contenga la respuesta correcta. \\

Las variedades parametrizadas que resultan del método son polinomios, por lo que con las paqueterías  mencionadas se puede analizar cuando dos de ellas se cruzan. Concretamente se tienen  $W^{s}(t)=(P_{x}(t),P_{y}(t))$ y $W^{u}(\tau)=(P_{x}(\tau),P_{y}(\tau))$ de órdenes no necesariamente iguales, y lo que se busca es :
\begin{eqnarray}
W^{s}(t)=W^{u}(\tau)
\end{eqnarray}
que arrojará como resultado un intervalo $I_{t}$ y otro intervalo $I_{\tau}$, la intersección se encontrará en $I_{t}\times I_{\tau}$. En el espacio fase la intersección se verá como el producto cartesiano de $W^{s}(I_{t})\times W^{u}(I_{\tau})$ formando una sección en la que se garantiza hay un punto homoclínico o heteroclínico. 


\subsection{Estándar}
Usando lo descrito anteriormente se calcularon las intersecciones de las variedades estable e inestable en el mapeo estándar para un valor de $\kappa=1.5$ usando polinomios de orden $120$, además de usar el mapeo inverso \ref{mapeo estandar inverso} para calcular la variedad estable. Se encontraron cuatro raíces en el intervalo $[-20.,0.]$ para $t$  y $[-15.,0.]$ para $\tau$, con una tolerancia de $10^{-6}$ usando el método de Newton, las cuales son:
\begin{itemize}
\item Root$([-7.16826, -7.16825] \times [-4.45972, -4.45971]$, :unique)
\item Root$([-4.21757, -4.21756] \times [-8.36029, -8.36028]$, :unique)
\item Root$([-2.24983, -2.24982] \times [-14.2093, -14.2092]$, :unique)
\item Root$([-13.4378, -13.4377] \times [-2.62396, -2.62395]$, :unique)
\end{itemize}
El primer intervalo corresponde al parámetro $t$ mientras que el segundo al parámetro $\tau$, la leyenda \texttt{unique} indica que en el intervalo presentado sólo hay una raíz. Las raíces se representan gráficamente en el espacio fase en la figura \ref{cruce_estandar}. 

\begin{figure}[H]
\centering
\includegraphics[scale=0.6]{cruce_estandar}
\caption{Cruces de $W^{u},W^{s}$ de orden $120$ para el mapeo estándar con $\kappa=1.5$ .}
\label{cruce_estandar}
\end{figure}
El error asociado al cálculo de las variedades se puede ver en la figura \ref{errorEstCruces}, en dónde se aprecia que los valores aceptables de los parámetros están dentro del intervalo inicial. 

\begin{figure}[H]
\centering
\includegraphics[scale=0.6]{error_cruces_estandar}
\caption{Error en las variedades de la figura \ref{cruce_estandar}.}
\label{errorEstCruces}
\end{figure}

El cálculo numérico riguroso garantiza que la solución está en el intervalo que da como resultado, sin embargo no debemos olvidar que nuestras variedades tienen un error asociado, por lo que es importante quedarse en aquellos intervalos del parámetro donde se tenga un error mínimo. Con las raíces obtenidas podemos asegurar que en el mapeo estándar con el valor del parámetro $\kappa=1.5$ se encuentran cuatro puntos homoclínicos. Si se quiere encontrar más puntos se debe considerar un polinomio de orden mayor o en todo caso usar números de precisión extendida para llegar más lejos en las variedades. 



\subsection{Hénon}
Así como se calcularon las intersecciones en las variedades del mapeo estándar se calcularon ahora para el mapeo de Hénon. En este caso se usó un valor del parámetro $a=1.5$ con un polinomio de orden 45 y el método de Newton con una tolerancia de $10^{-6}$ además de usar el mapeo inverso \ref{HenonI}, para calcular la variedad estable. Los resultados fueron los siguientes intervalos:
\begin{itemize}
\item[a)] Root$([-1.36597, -1.36596] \times [166.749, 166.75]$, :unique)
\item[b)] Root$([-5.26555, -5.26554] \times [129.577, 129.578]$, :unique)
\item[c)] Root$([-6.77613, -6.77612] \times [33.6142, 33.6143]$, :unique)
\item[d)] Root$([-5.54438e-07, 0] \times [0, 5.54438e-07]$, :unknown)     
\item[e)] Root$([-26.1208, -26.1207] \times [26.1207, 26.1208]$, :unique)  
\item[f)] Root$([-33.6143, -33.6142] \times [6.77612, 6.77613]$, :unique)  
\item[g)] Root$([-129.578, -129.577] \times [5.26554, 5.26555]$, :unique) 
\item[h)] Root$([-166.75, -166.749] \times [1.36596, 1.36597]4$, :unique)
\end{itemize}

La leyenda \texttt{unknown} nos dice que no puede concluir si hay una o más raíces en el intervalo. Si notamos el tercer intervalo contiene al cero $(t,\tau)=(0,0)$ en el cual se cortan las variedades pues representa el punto fijo. La figura \ref{crucesH} representa con un punto los cortes encontrados en las variedades.
\begin{figure}[H]
\centering
\includegraphics[scale=0.5]{crucesL}
\caption{Cruces de $W^{u},W^{s}$ encontrados en el intervalo $[-400.,0.] \times [0.,400.]$ .}
\label{crucesH}
\end{figure}
Los puntos de color son sólo para indicar cuales intersecciones fueron encontradas. Para los intervalos encontrados se hizo una gráfica que representa la región en el espacio fase donde se encuentra el cruce, las figuras \ref{cruce1H}-\ref{cruce8H} muestran cada una de ellas.

\begin{figure}[H]
\centering
\includegraphics[scale=0.4]{cruce1}
\caption{Cruce de $W^{u},W^{s}$ encontrado con el intervalo a.}
\label{cruce1H}
\end{figure}

\begin{figure}[H]
\centering
\includegraphics[scale=0.4]{cruce2}
\caption{Cruce de $W^{u},W^{s}$ encontrado con el intervalo b.}
\label{cruce2H}
\end{figure}


\begin{figure}[H]
\centering
\includegraphics[scale=0.4]{cruce3}
\caption{Cruce de $W^{u},W^{s}$ encontrado con el intervalo c.}
\label{cruce3H}
\end{figure}


\begin{figure}[H]
\centering
\includegraphics[scale=0.4]{cruce5}
\caption{Cruce de $W^{u},W^{s}$ encontrado con el intervalo e.}
\label{cruce5H}
\end{figure}

\begin{figure}[H]
\centering
\includegraphics[scale=0.4]{cruce6}
\caption{Cruce de $W^{u},W^{s}$ encontrado con el intervalo f.}
\label{cruce6H}
\end{figure}

\begin{figure}[H]
\centering
\includegraphics[scale=0.4]{cruce7}
\caption{Cruce de $W^{u},W^{s}$ encontrado con el intervalo g.}
\label{cruce7H}
\end{figure}

\begin{figure}[H]
\centering
\includegraphics[scale=0.4]{cruce8}
\caption{Cruce de $W^{u},W^{s}$ encontrado con el intervalo h.}
\label{cruce8H}
\end{figure}

La zona rectangular sombreada en cada gráfica representa el producto cartesiano de los intervalos donde se encuentra la solución. Podemos observar que de hecho cada zona contiene el cruce de las variedades garantizando así que en el intervalo propuesto hay un cruce lo cual representa un punto homoclínico. Algunos de los cruces como los de las figuras \ref{cruce1H}, \ref{cruce2H}, \ref{cruce7H} y \ref{cruce8H} son realmente imperceptibles. Este tipo de resultados son útiles para hablar sobre caos topológico \cite{devaney},\cite{gerald}.\\

Para complementar todo este análisis se puede obtener una gráfica de las superficies que forman las variedades al cambiar el parámetro del mapeo lo que nos da una idea de como se ven las superficies y además de como se comportan las intersecciones. La figura \ref{SuperficiesH} muestra las superficies formadas por las variedades para ciertos valores del parámetro $a$. Algunas gráficas más se encuentran en \url{https://github.com/alvarezeve/Tesis-Variedades-Estables-e-inestables/blob/master/Superficies.ipynb}
\begin{figure}[H]
\centering
\includegraphics[scale=0.9]{HenonV}
\caption{Superficies en el mapeo de Hénon formadas por las variedades.}
\label{SuperficiesH}
\end{figure}



\subsection{Mapeo exponencial}
Las intersecciones en el caso del mapeo \ref{Jung} se calcularon usando como en los casos anteriores el mapeo inverso \ref{jungI}. Este mapeo representa un mayor reto en cuanto al orden del polinomio, pes la presencia de la exponencial hace que la parametrización sea sensible al orden del mismo. Para el siguiente ejemplo se utilizó un polinomio de orden 86 y una tolerancia en el método de Newton de $10-6$ y se calcularon los cruces de las variedades como en los otros casos. Las siguientes fueron las secciones en términos del parámetro $t,\tau$ dónde se encontraron los cortes. 
\begin{itemize}
\item[a)] Root$([-0.985068, -0.985067] \times [5.99488, 5.99489]$, :unique)
\item[b)] Root$([-3.46215, -3.46214] \times [5.49229, 5.4923]$, :unique)
\item[c)] Root$([-3.77896, -3.77895] \times [1.56269, 1.5627]$, :unique)
\end{itemize}
La representación de los cortes se puede ver en la figura \ref{jung_cortes}.
\begin{figure}[H]
\centering
\includegraphics[scale=0.5]{cruces_jung1}
\caption{Intersecciones en el mapeo exponencial con $a=5.7$.}
\label{jung_cortes}
\end{figure}

Tomando los cruces a escala del intervalo se obtuvieron las figuras \ref{jung_corte1}-\ref{jung_corte3}.

\begin{figure}[H]
\centering
\includegraphics[scale=0.4]{cruce_a}
\caption{Intersección en el intervalo a.}
\label{jung_corte1}
\end{figure}


\begin{figure}[H]
\centering
\includegraphics[scale=0.4]{cruce_b}
\caption{Intersección en el intervalo b.}
\label{jung_corte2}
\end{figure}


\begin{figure}[H]
\centering
\includegraphics[scale=0.4]{cruce_c}
\caption{Intersección en el intervalo c.}
\label{jung_corte3}
\end{figure}

Las escalas en las que se observan los cortes de las variedades son pequeños comparados con la escala del mapeo, tanto que si se graficaran en el espacio fase no se podrían observar. Como ya se mencionó estos garantiza que existen tres puntos homoclínicos en el intervalo usado. \\


La implementación del método resultó ser eficiente para encontrar los polinomios que representan las variedades estable e inestable asociadas a puntos fijos hiperbólicos de los diferentes mapeos. Se tomaron estos tres mapeos para tener un poco de variedad en la forma de las variedades y también de los puntos fijos, en todos los casos el orden de la parametrización afecta de manera diferente, para el caso del mapeo exponencial resulta ser más sensible. En todos ellos un orden mayor contribuye a llegar más lejos en el valor del parámetro con un error bajo y como consecuencia graficar de mejor manera las variedades. Podemos decir que el error lo manipulamos al mover el orden pero también mostramos que se puede mejorar al usar números de precisión extendida. Una característica que se puede observar es que el error se comporta esencialmente de la misma forma para los tres mapeos, creciendo de manera lenta y luego creciendo rápido a partir de cierto valor. \\


El análisis de la convergencia resulta congruente con lo que se puede observar en el error y en las mismas gráficas de las variedades. La convergencia también ayudó a ver que a partir de cierto orden los coeficientes no hacen un cambio significativo, por lo que no es necesariamente cierto que cada que aumentemos el orden se mejorará la parametrización.\\


El cálculo de puntos homoclínicos resulta una de las aplicaciones de tener las variedades parametrizadas. El método junto con las otras paqueterías de Julia son indispensables para hacer este tipo de cálculos de una manera fácil. A pesar de que la aritmética de intervalos arroja resultados garantizados, no hay que olvidar que nuestros polinomios no son calculados de la misma manera, es decir se debe tener siempre en mente que al rededor de cada variedad hay un error asociado. Aún así las intersecciones pueden ser encontradas controlando bien el error en la parametrización y controlando la tolerancia en el método de Newton.

\section{Rectángulo fundamental}
Para los mapeos abiertos como lo son Hénon y el mapeo de la sección anterior \ref{Jung} se puede tener el conjunto fundamental a partir de los polinomios que se obtienen en el método de parametrización. El conjunto fundamental constituye una parte de las variedades mediante la cual se puede obtener toda la dinámica fuera del mismo, llamado también rectángulo fundamental puesto que forma un polígono de cuatro lados que tiene como vértices el punto fijo hiperbólico y las intersecciones de las variedades, como aristas las secciones de las variedades entre estos puntos. Lo que pasa dentro del área del rectángulo fundamental con las variedades estables e inestables es una ventana a escala de lo que pasa si se extienden las variedades. A partir de esto podemos observar lo que se llaman tentáculos o herraduras en la dinámica del mapeo, en la figura \ref{herradura} se muestra un diagrama de este tipo de comportamiento.

\begin{figure}[H]
\centering
\includegraphics[scale=0.25]{herradura}
\caption{Diagrama ilustrativo de la topología de una herradura en un sistema Hamiltoniano de dos dimensiones, H denota el punto fijo hiperbólico y P la primera intersección. Los tentáculos de la variedad estable son enfatizados con color negro. Tomada de \cite{Ying}, pp.215.}
\label{herradura}
\end{figure}

Una manera de llegar a observar estas estructuras con los polinomios de la parametrización sería usando polinomios de ordenes grandes, pero como hemos visto en la sección \ref{SeccionEstandar} el error no cambia mucho cuando pasamos de cierto grado en el polinomio. Sin embargo si notamos que nuestra parametrización representa, con cierto error, una sección de la variedad estable o inestable, entonces podemos aplicar el mapeo a la parametrización de manera que iteramos, análogamente a iterar puntos. Con esta idea se pudo encontrar parte de las estructuras mostradas en la figura \ref{herradura}, para los mapeos \ref{Henon}, \ref{Jung}.


En el caso del mapeo de Hénon con $a=6.5$ se calcularon polinomios de orden 250 usando números de precisión extendida ($W_{0}^{s},W_{0}^{u}$), además calcular la variedad estable usando el mapeo de Hénon inverso \ref{HenonI}. En este caso se necesitó un valor máximo del parámetro $t=100$ para obtener el rectángulo fundamental, que se muestra en la figura \ref{rectangulo0}, para éstas condiciones el error es menor a $10^{-73}$.

\begin{figure}[H]
\centering
\includegraphics[scale=0.5]{rectangulo-fundamental}
\caption{Variedades estable e inestable de orden 250, para el mapeo de Hénon con $a=6.5,b=1.$, con $t_{max}=100$. El punto c) denota el punto fijo mientras que  a),b),d) son las primeras intersecciones de las variedades.}
\label{rectangulo0}
\end{figure}

\begin{figure}[H]
\centering
\includegraphics[scale=0.5]{error-rectangulo}
\caption{Error asociado al cálculo de las variedades en la figura \ref{rectangulo0}}.
\label{ErrorRectangulo0}
\end{figure} 

Reescribiendo la ecuación de invariancia \ref{Ecua de invariancia} para el caso de la variedad estable, tenemos que:
\begin{eqnarray}
f_{a,b}^{-1}(W_{0}^{s}(t))=W_{0}^{s}(\lambda^{s}t).
\label{InvarianciaEstable1}
\end{eqnarray}
%\begin{eqnarray}
%f_{a,b}(W_{0}^{u}(t))=W_{0}^{u}(\lambda^{u}t)
%\label{InvarianciaInestable1}
%\end{eqnarray}
Si aplicamos el mapeo de Hénon inverso \ref{HenonI} a la ecuación \ref{InvarianciaEstable1} resulta
\begin{eqnarray}
W_{0}^{s}(t)=f_{a,b}(W_{0}^{s}(\lambda^{s}t)),
\label{InvarianciaEstable2}
\end{eqnarray}

%\begin{eqnarray}
%W_{0}^{u}(t)=f_{a,b}^{-1}(W_{0}^{u}(\lambda^{u}t))
%\label{InvarianciaInestable2}
%\end{eqnarray}
como el parámetro es una variable muda podemos reescribir \ref{InvarianciaEstable2}, 
\begin{eqnarray}
W_{0}^{s}(\frac{t}{\lambda^{s}})=f_{a,b}(W_{0}^{s}(t)).
\label{InvarianciaEstable3}
\end{eqnarray}

%\begin{eqnarray}
%W_{0}^{u}(\frac{t}{\lambda^{u}})=f_{a,b}^{-1}(W_{0}^{u}(t)).
%\label{InvarianciaInestable3}
%\end{eqnarray}
Siendo que $\vert \lambda^{s} \vert < 1 $ la ecuación \ref{InvarianciaEstable3} muestra que aplicar el mapeo es análogo a tener la variedad estable evaluada en un valor mayor del parámetro, puesto que $1/\lambda^{s}>1$. Este resultado es análogo para la variedad inestable. Usando esto se obtuvo la figura \ref{Rectangulo1}, que muestra el resultado de la iterar una vez, $(W_{x1}^{s},W_{y1}^{s})=f_{a,b}(W_{x}^{s},W_{y}^{s})$, análogamente para las inestables, evaluando los nuevos polinomios exactamente en los mismos intervalos que en el caso de la figura \ref{rectangulo0}.
\begin{figure}[H]
\centering
\includegraphics[scale=0.5]{rectangulo1}
\caption{Primera aplicación del mapeo a los polinomios de orden 250, $t_{max}=100$}.
\label{Rectangulo1}
\end{figure}
La segunda aplicación de los mapeos a los polinomios, $(W_{x2}^{s},W_{y2}^{s})=f_{a,b}(W_{x1}^{s},W_{y1}^{s})$ se muestra en la figura \ref{Rectangulo2}, de nuevo usando el mismo valor máximo del parámetro $t$.
\begin{figure}[H]
\centering
\includegraphics[scale=0.5]{rectangulo2}
\caption{Segunda aplicación del mapeo a los polinomios de orden 250.}.
\label{Rectangulo2}
\end{figure}

De manera sucesiva se aplicaron los mapeos correspondientes hasta iterar cinco veces, siempre conservando el mismo valor máximo del parámetro $t$.
\begin{figure}[H]
\centering
\includegraphics[scale=0.5]{rectangulo3}
\caption{Tercer aplicación del mapeo a los polinomios de orden 250.}.
\label{Rectangulo3}
\end{figure}

\begin{figure}[H]
\centering
\includegraphics[scale=0.5]{rectangulo4A}
\caption{Cuarta aplicación del mapeo a los polinomios de orden 250.}.
\label{Rectangulo4}
\end{figure}

\begin{figure}[H]
\centering
\includegraphics[scale=0.5]{rectangulo5}
\caption{Quinta aplicación del mapeo a los polinomios de orden 250.}.
\label{Rectangulo5}
\end{figure}

En las primeras dos iteraciones de los polinomios se puede observar un tentáculo más en cada variedad, mientras que en las siguientes aparecen hasta dos más. La profundidad de tentáculos y la forma en la que se cruzan, depende del valor del parámetro $a$. Hay que enfatizar que para lograr observar estos cruces de manera directa, se debería llegar a valores del parámetro muy grandes comparados con el valor que se usó en todos los casos, lo cual no siempre es posible pues como se vio en la sección \ref{SeccionEstandar} después de cierto orden el error se estanca y no es posible llegar tan lejos con errores controlables. Con esto mostramos que sólo se necesita conocer el rectángulo fundamental para conocer mucho de la dinámica de las variedades, y por tanto de los cruces entre ellas. 


 
%\lhead{\begin{picture}(0,0) \put(0,0){\includegraphics[H][width=20mm]{curvas1}} \end{picture}}

\chapter{Panorama}
Una característica importante a estudiar en los mapeos en general son las variedades estables e inestables asociadas a puntos fijos. En el caso de los mapeos de dos dimensiones resulta manejable, hasta cierto punto, encontrarlas de manera analítica usando el método de parametrización. Como vimos el método tiene como núcleo de desarrollo la ecuación de invariancia y la linearización del sistema al rededor de un punto fijo. Sin embargo decir manejable en términos matemáticos y físicos no resulta suficiente si lo que necesitamos es estudiar propiedades de los sistemas a partir de las variedades o el comportamiento de puntos fijos. Por ello es que la implementación del método resultaba llamativa. Tener un módulo escrito en software libre que calcula las variedades asociadas a puntos fijos hiperbólicos va más allá de generar las relaciones de recurrencia en casos particulares. El método automatizado es capaz de generar las parametrizaciones de las variedades al rededor de un punto fijo conocido, en cualquier mapeo de dos dimensiones que sea simpléctico. La idea detrás de la automatización se basa en que la computadora haga las veces de la recurrencia en lugar de calcularlas de manera analítica. Esto de ninguna manera modifica el modo del método. Todo esto da como resultado un método semianalítico con el cual tenemos las variedades de manera polinómica. \\

Dado que es un método parte analítico y parte computacional que involucra series de Taylor es crucial decir de alguna manera que tan confiable es la parametrización que resulta. Por ello se incluyeron tres formas de evaluar el comportamiento de las variedades, tales involucran al error y la convergencia mediante dos métodos. Conocer que tanto podemos afirmar sobre el comportamiento de las variedades depende de estas tres funciones.\\

Como pudimos notar en el capítulo anterior en los tres ejemplos que se presentan observamos comportamientos muy variados. En el mapeo estándar se buscaba mostrar un ejemplo bastante conocido, en el de henón mostrar el cruce de variedades mientras que en el exponencial que tantos cruces ( o tentáculos) se pueden observar. Puede pasar que para otros mapeos se busque estudiar un comportamiento particular de las variedades. Por ello consideramos que el método se puede explotar en varias direcciones que se adapten a cada mapeo en particular. En el caso por ejemplo de observar la dependencia de los tentáculos de las variedades con respecto a los parámetros del mapeo tendrá que ver con que tan grande podemos hacer el orden del polinomio  y que tan lejos llegamos con un error relativamente pequeño. Además como mostramos hay formas de mejorar la parametrización usando el mapeo inverso. \\

En el mapeo de Hénon mostramos que es posible calcular las intersecciones de las variedades. Tal resultado nos muestra una ventana hacía la \emph{demostración} de que hay un corte entre ambas y con ello a resultados más importantes y áreas más amplias como el estudio de bifurcaciones y caos. Esta idea resulta de que al conocer las variedades de manera analítica podemos buscar el cero de las funciones correctas para encontrar un cruce. Aún si no se conoce si en el sistema las variedades se cruzan en uno o más puntos, con el orden suficiente se podría asegurar. Para ello habrá que trabajar de manera analítica primero con las parametrizaciones y asociar un error al cálculo de la intersección . \\

En el proceso de este trabajo surgió una duda que se formuló en principio como sigue :¿es posible seleccionar una dirección preferente en la parametrización?. Es decir, es claro que en términos del vector propio hay una dirección en la que se desarrolla la variedad, en algunos casos resulta que la variedad puede pasar cerca de algún otro punto fijo del sistema, ¿cómo seleccionar una dirección preferente para el desarrollo del polinomio?. Resulta que los polinomios de Taylor no tienen una dirección preferencial , mientras que los polinomios de Chebyshev sí. En términos burdos uno puede usar polinomios diferentes que tengan preferencias topológicas con respecto a los de Taylor. Eso puede ayudar a que la parametrización se encamine a la dirección necesaria desde un inicio. Se piensa que el método de parametrización se puede desarrollar usando tales polinomios en lugar de usar Taylor. Sería una manera de parametrizar más enfocada. \\
Todos los comentarios anteriores intentan mostrar que hay aún preguntas sin responder del todo y sobre todo curiosidad por hacer más con el método de parametrización tanto de manera computacional como de manera analítica. Sin duda puede que dentro del estudio de variedades en mapeos de dos dimensiones haya muchas más cosas que se estén estudiando con diferentes y variados propósitos. 


     % ~20 páginas - Explicar el problema en específico que se va a resolver, la met   % ~20 páginas - Presentar los resultados tal cual son, y analizarlos.
%\include{Capitulo4/Conclusiones}            % ~5 páginas - Resumir lo que se hizo y lo que no y comentar trabajos futuros sobre el tema

%%%%%%%%%%%%%%%%%%%%%%%%%%%%%%%%%%%%%%%%%%%%%%%%%%%%%
%                   APÉNDICES                       %
%%%%%%%%%%%%%%%%%%%%%%%%%%%%%%%%%%%%%%%%%%%%%%%%%%%%%
\appendix
%% this file is called up by thesis.tex
% content in this file will be fed into the main document
\chapter{Código }
% top level followed by section, subsection

\section{Método de Parametrización}
A continuación presentamos el código escrito en Julia , sólo con el fin de apoyar en alguna parte de la lectura del pseudocóigo ya que el mismo se encuentra en el repositorio de GitHub PONER-LIGA y se puede consultar en internet.

               % Colocar los circuitos, manuales, código fuente, pruebas de teoremas, etc.

%%%%%%%%%%%%%%%%%%%%%%%%%%%%%%%%%%%%%%%%%%%%%%%%%%%%%
%                   REFERENCIAS                     %
%%%%%%%%%%%%%%%%%%%%%%%%%%%%%%%%%%%%%%%%%%%%%%%%%%%%%
% existen varios estilos de bilbiografía, pueden cambiarlos a placer
 % otros estilos pueden ser abbrv, acm, alpha, apalike, ieeetr, plain, siam, unsrt

%El formato trae otros estilos, o pueden agregar uno que les guste:
%\bibliographystyle{Latex/Classes/PhDbiblio-case} % title forced lower case
%\bibliographystyle{Latex/Classes/PhDbiblio-bold} % title as in bibtex but bold
%\bibliographystyle{Latex/Classes/PhDbiblio-url} % bold + www link if provided
%\bibliographystyle{Latex/Classes/jmb} % calls style file jmb.bst
\bibliographystyle{unsrt}
\bibliography{referencias}             % Archivo .bib
\nocite{*}
\end{document}
