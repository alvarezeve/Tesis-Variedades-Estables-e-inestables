
% Default to the notebook output style

    


% Inherit from the specified cell style.




    
\documentclass[11pt]{article}

    
    
    \usepackage[T1]{fontenc}
    % Nicer default font (+ math font) than Computer Modern for most use cases
    \usepackage{mathpazo}

    % Basic figure setup, for now with no caption control since it's done
    % automatically by Pandoc (which extracts ![](path) syntax from Markdown).
    \usepackage{graphicx}
    % We will generate all images so they have a width \maxwidth. This means
    % that they will get their normal width if they fit onto the page, but
    % are scaled down if they would overflow the margins.
    \makeatletter
    \def\maxwidth{\ifdim\Gin@nat@width>\linewidth\linewidth
    \else\Gin@nat@width\fi}
    \makeatother
    \let\Oldincludegraphics\includegraphics
    % Set max figure width to be 80% of text width, for now hardcoded.
    \renewcommand{\includegraphics}[1]{\Oldincludegraphics[width=.8\maxwidth]{#1}}
    % Ensure that by default, figures have no caption (until we provide a
    % proper Figure object with a Caption API and a way to capture that
    % in the conversion process - todo).
    \usepackage{caption}
    \DeclareCaptionLabelFormat{nolabel}{}
    \captionsetup{labelformat=nolabel}

    \usepackage{adjustbox} % Used to constrain images to a maximum size 
    \usepackage{xcolor} % Allow colors to be defined
    \usepackage{enumerate} % Needed for markdown enumerations to work
    \usepackage{geometry} % Used to adjust the document margins
    \usepackage{amsmath} % Equations
    \usepackage{amssymb} % Equations
    \usepackage{textcomp} % defines textquotesingle
    % Hack from http://tex.stackexchange.com/a/47451/13684:
    \AtBeginDocument{%
        \def\PYZsq{\textquotesingle}% Upright quotes in Pygmentized code
    }
    \usepackage{upquote} % Upright quotes for verbatim code
    \usepackage{eurosym} % defines \euro
    \usepackage[mathletters]{ucs} % Extended unicode (utf-8) support
    \usepackage[utf8x]{inputenc} % Allow utf-8 characters in the tex document
    \usepackage{fancyvrb} % verbatim replacement that allows latex
    \usepackage{grffile} % extends the file name processing of package graphics 
                         % to support a larger range 
    % The hyperref package gives us a pdf with properly built
    % internal navigation ('pdf bookmarks' for the table of contents,
    % internal cross-reference links, web links for URLs, etc.)
    \usepackage{hyperref}
    \usepackage{longtable} % longtable support required by pandoc >1.10
    \usepackage{booktabs}  % table support for pandoc > 1.12.2
    \usepackage[inline]{enumitem} % IRkernel/repr support (it uses the enumerate* environment)
    \usepackage[normalem]{ulem} % ulem is needed to support strikethroughs (\sout)
                                % normalem makes italics be italics, not underlines
    

    
    
    % Colors for the hyperref package
    \definecolor{urlcolor}{rgb}{0,.145,.698}
    \definecolor{linkcolor}{rgb}{.71,0.21,0.01}
    \definecolor{citecolor}{rgb}{.12,.54,.11}

    % ANSI colors
    \definecolor{ansi-black}{HTML}{3E424D}
    \definecolor{ansi-black-intense}{HTML}{282C36}
    \definecolor{ansi-red}{HTML}{E75C58}
    \definecolor{ansi-red-intense}{HTML}{B22B31}
    \definecolor{ansi-green}{HTML}{00A250}
    \definecolor{ansi-green-intense}{HTML}{007427}
    \definecolor{ansi-yellow}{HTML}{DDB62B}
    \definecolor{ansi-yellow-intense}{HTML}{B27D12}
    \definecolor{ansi-blue}{HTML}{208FFB}
    \definecolor{ansi-blue-intense}{HTML}{0065CA}
    \definecolor{ansi-magenta}{HTML}{D160C4}
    \definecolor{ansi-magenta-intense}{HTML}{A03196}
    \definecolor{ansi-cyan}{HTML}{60C6C8}
    \definecolor{ansi-cyan-intense}{HTML}{258F8F}
    \definecolor{ansi-white}{HTML}{C5C1B4}
    \definecolor{ansi-white-intense}{HTML}{A1A6B2}

    % commands and environments needed by pandoc snippets
    % extracted from the output of `pandoc -s`
    \providecommand{\tightlist}{%
      \setlength{\itemsep}{0pt}\setlength{\parskip}{0pt}}
    \DefineVerbatimEnvironment{Highlighting}{Verbatim}{commandchars=\\\{\}}
    % Add ',fontsize=\small' for more characters per line
    \newenvironment{Shaded}{}{}
    \newcommand{\KeywordTok}[1]{\textcolor[rgb]{0.00,0.44,0.13}{\textbf{{#1}}}}
    \newcommand{\DataTypeTok}[1]{\textcolor[rgb]{0.56,0.13,0.00}{{#1}}}
    \newcommand{\DecValTok}[1]{\textcolor[rgb]{0.25,0.63,0.44}{{#1}}}
    \newcommand{\BaseNTok}[1]{\textcolor[rgb]{0.25,0.63,0.44}{{#1}}}
    \newcommand{\FloatTok}[1]{\textcolor[rgb]{0.25,0.63,0.44}{{#1}}}
    \newcommand{\CharTok}[1]{\textcolor[rgb]{0.25,0.44,0.63}{{#1}}}
    \newcommand{\StringTok}[1]{\textcolor[rgb]{0.25,0.44,0.63}{{#1}}}
    \newcommand{\CommentTok}[1]{\textcolor[rgb]{0.38,0.63,0.69}{\textit{{#1}}}}
    \newcommand{\OtherTok}[1]{\textcolor[rgb]{0.00,0.44,0.13}{{#1}}}
    \newcommand{\AlertTok}[1]{\textcolor[rgb]{1.00,0.00,0.00}{\textbf{{#1}}}}
    \newcommand{\FunctionTok}[1]{\textcolor[rgb]{0.02,0.16,0.49}{{#1}}}
    \newcommand{\RegionMarkerTok}[1]{{#1}}
    \newcommand{\ErrorTok}[1]{\textcolor[rgb]{1.00,0.00,0.00}{\textbf{{#1}}}}
    \newcommand{\NormalTok}[1]{{#1}}
    
    % Additional commands for more recent versions of Pandoc
    \newcommand{\ConstantTok}[1]{\textcolor[rgb]{0.53,0.00,0.00}{{#1}}}
    \newcommand{\SpecialCharTok}[1]{\textcolor[rgb]{0.25,0.44,0.63}{{#1}}}
    \newcommand{\VerbatimStringTok}[1]{\textcolor[rgb]{0.25,0.44,0.63}{{#1}}}
    \newcommand{\SpecialStringTok}[1]{\textcolor[rgb]{0.73,0.40,0.53}{{#1}}}
    \newcommand{\ImportTok}[1]{{#1}}
    \newcommand{\DocumentationTok}[1]{\textcolor[rgb]{0.73,0.13,0.13}{\textit{{#1}}}}
    \newcommand{\AnnotationTok}[1]{\textcolor[rgb]{0.38,0.63,0.69}{\textbf{\textit{{#1}}}}}
    \newcommand{\CommentVarTok}[1]{\textcolor[rgb]{0.38,0.63,0.69}{\textbf{\textit{{#1}}}}}
    \newcommand{\VariableTok}[1]{\textcolor[rgb]{0.10,0.09,0.49}{{#1}}}
    \newcommand{\ControlFlowTok}[1]{\textcolor[rgb]{0.00,0.44,0.13}{\textbf{{#1}}}}
    \newcommand{\OperatorTok}[1]{\textcolor[rgb]{0.40,0.40,0.40}{{#1}}}
    \newcommand{\BuiltInTok}[1]{{#1}}
    \newcommand{\ExtensionTok}[1]{{#1}}
    \newcommand{\PreprocessorTok}[1]{\textcolor[rgb]{0.74,0.48,0.00}{{#1}}}
    \newcommand{\AttributeTok}[1]{\textcolor[rgb]{0.49,0.56,0.16}{{#1}}}
    \newcommand{\InformationTok}[1]{\textcolor[rgb]{0.38,0.63,0.69}{\textbf{\textit{{#1}}}}}
    \newcommand{\WarningTok}[1]{\textcolor[rgb]{0.38,0.63,0.69}{\textbf{\textit{{#1}}}}}
    
    
    % Define a nice break command that doesn't care if a line doesn't already
    % exist.
    \def\br{\hspace*{\fill} \\* }
    % Math Jax compatability definitions
    \def\gt{>}
    \def\lt{<}
    % Document parameters
    \title{N5(Implementaci?n)}
    
    
    

    % Pygments definitions
    
\makeatletter
\def\PY@reset{\let\PY@it=\relax \let\PY@bf=\relax%
    \let\PY@ul=\relax \let\PY@tc=\relax%
    \let\PY@bc=\relax \let\PY@ff=\relax}
\def\PY@tok#1{\csname PY@tok@#1\endcsname}
\def\PY@toks#1+{\ifx\relax#1\empty\else%
    \PY@tok{#1}\expandafter\PY@toks\fi}
\def\PY@do#1{\PY@bc{\PY@tc{\PY@ul{%
    \PY@it{\PY@bf{\PY@ff{#1}}}}}}}
\def\PY#1#2{\PY@reset\PY@toks#1+\relax+\PY@do{#2}}

\expandafter\def\csname PY@tok@w\endcsname{\def\PY@tc##1{\textcolor[rgb]{0.73,0.73,0.73}{##1}}}
\expandafter\def\csname PY@tok@c\endcsname{\let\PY@it=\textit\def\PY@tc##1{\textcolor[rgb]{0.25,0.50,0.50}{##1}}}
\expandafter\def\csname PY@tok@cp\endcsname{\def\PY@tc##1{\textcolor[rgb]{0.74,0.48,0.00}{##1}}}
\expandafter\def\csname PY@tok@k\endcsname{\let\PY@bf=\textbf\def\PY@tc##1{\textcolor[rgb]{0.00,0.50,0.00}{##1}}}
\expandafter\def\csname PY@tok@kp\endcsname{\def\PY@tc##1{\textcolor[rgb]{0.00,0.50,0.00}{##1}}}
\expandafter\def\csname PY@tok@kt\endcsname{\def\PY@tc##1{\textcolor[rgb]{0.69,0.00,0.25}{##1}}}
\expandafter\def\csname PY@tok@o\endcsname{\def\PY@tc##1{\textcolor[rgb]{0.40,0.40,0.40}{##1}}}
\expandafter\def\csname PY@tok@ow\endcsname{\let\PY@bf=\textbf\def\PY@tc##1{\textcolor[rgb]{0.67,0.13,1.00}{##1}}}
\expandafter\def\csname PY@tok@nb\endcsname{\def\PY@tc##1{\textcolor[rgb]{0.00,0.50,0.00}{##1}}}
\expandafter\def\csname PY@tok@nf\endcsname{\def\PY@tc##1{\textcolor[rgb]{0.00,0.00,1.00}{##1}}}
\expandafter\def\csname PY@tok@nc\endcsname{\let\PY@bf=\textbf\def\PY@tc##1{\textcolor[rgb]{0.00,0.00,1.00}{##1}}}
\expandafter\def\csname PY@tok@nn\endcsname{\let\PY@bf=\textbf\def\PY@tc##1{\textcolor[rgb]{0.00,0.00,1.00}{##1}}}
\expandafter\def\csname PY@tok@ne\endcsname{\let\PY@bf=\textbf\def\PY@tc##1{\textcolor[rgb]{0.82,0.25,0.23}{##1}}}
\expandafter\def\csname PY@tok@nv\endcsname{\def\PY@tc##1{\textcolor[rgb]{0.10,0.09,0.49}{##1}}}
\expandafter\def\csname PY@tok@no\endcsname{\def\PY@tc##1{\textcolor[rgb]{0.53,0.00,0.00}{##1}}}
\expandafter\def\csname PY@tok@nl\endcsname{\def\PY@tc##1{\textcolor[rgb]{0.63,0.63,0.00}{##1}}}
\expandafter\def\csname PY@tok@ni\endcsname{\let\PY@bf=\textbf\def\PY@tc##1{\textcolor[rgb]{0.60,0.60,0.60}{##1}}}
\expandafter\def\csname PY@tok@na\endcsname{\def\PY@tc##1{\textcolor[rgb]{0.49,0.56,0.16}{##1}}}
\expandafter\def\csname PY@tok@nt\endcsname{\let\PY@bf=\textbf\def\PY@tc##1{\textcolor[rgb]{0.00,0.50,0.00}{##1}}}
\expandafter\def\csname PY@tok@nd\endcsname{\def\PY@tc##1{\textcolor[rgb]{0.67,0.13,1.00}{##1}}}
\expandafter\def\csname PY@tok@s\endcsname{\def\PY@tc##1{\textcolor[rgb]{0.73,0.13,0.13}{##1}}}
\expandafter\def\csname PY@tok@sd\endcsname{\let\PY@it=\textit\def\PY@tc##1{\textcolor[rgb]{0.73,0.13,0.13}{##1}}}
\expandafter\def\csname PY@tok@si\endcsname{\let\PY@bf=\textbf\def\PY@tc##1{\textcolor[rgb]{0.73,0.40,0.53}{##1}}}
\expandafter\def\csname PY@tok@se\endcsname{\let\PY@bf=\textbf\def\PY@tc##1{\textcolor[rgb]{0.73,0.40,0.13}{##1}}}
\expandafter\def\csname PY@tok@sr\endcsname{\def\PY@tc##1{\textcolor[rgb]{0.73,0.40,0.53}{##1}}}
\expandafter\def\csname PY@tok@ss\endcsname{\def\PY@tc##1{\textcolor[rgb]{0.10,0.09,0.49}{##1}}}
\expandafter\def\csname PY@tok@sx\endcsname{\def\PY@tc##1{\textcolor[rgb]{0.00,0.50,0.00}{##1}}}
\expandafter\def\csname PY@tok@m\endcsname{\def\PY@tc##1{\textcolor[rgb]{0.40,0.40,0.40}{##1}}}
\expandafter\def\csname PY@tok@gh\endcsname{\let\PY@bf=\textbf\def\PY@tc##1{\textcolor[rgb]{0.00,0.00,0.50}{##1}}}
\expandafter\def\csname PY@tok@gu\endcsname{\let\PY@bf=\textbf\def\PY@tc##1{\textcolor[rgb]{0.50,0.00,0.50}{##1}}}
\expandafter\def\csname PY@tok@gd\endcsname{\def\PY@tc##1{\textcolor[rgb]{0.63,0.00,0.00}{##1}}}
\expandafter\def\csname PY@tok@gi\endcsname{\def\PY@tc##1{\textcolor[rgb]{0.00,0.63,0.00}{##1}}}
\expandafter\def\csname PY@tok@gr\endcsname{\def\PY@tc##1{\textcolor[rgb]{1.00,0.00,0.00}{##1}}}
\expandafter\def\csname PY@tok@ge\endcsname{\let\PY@it=\textit}
\expandafter\def\csname PY@tok@gs\endcsname{\let\PY@bf=\textbf}
\expandafter\def\csname PY@tok@gp\endcsname{\let\PY@bf=\textbf\def\PY@tc##1{\textcolor[rgb]{0.00,0.00,0.50}{##1}}}
\expandafter\def\csname PY@tok@go\endcsname{\def\PY@tc##1{\textcolor[rgb]{0.53,0.53,0.53}{##1}}}
\expandafter\def\csname PY@tok@gt\endcsname{\def\PY@tc##1{\textcolor[rgb]{0.00,0.27,0.87}{##1}}}
\expandafter\def\csname PY@tok@err\endcsname{\def\PY@bc##1{\setlength{\fboxsep}{0pt}\fcolorbox[rgb]{1.00,0.00,0.00}{1,1,1}{\strut ##1}}}
\expandafter\def\csname PY@tok@kc\endcsname{\let\PY@bf=\textbf\def\PY@tc##1{\textcolor[rgb]{0.00,0.50,0.00}{##1}}}
\expandafter\def\csname PY@tok@kd\endcsname{\let\PY@bf=\textbf\def\PY@tc##1{\textcolor[rgb]{0.00,0.50,0.00}{##1}}}
\expandafter\def\csname PY@tok@kn\endcsname{\let\PY@bf=\textbf\def\PY@tc##1{\textcolor[rgb]{0.00,0.50,0.00}{##1}}}
\expandafter\def\csname PY@tok@kr\endcsname{\let\PY@bf=\textbf\def\PY@tc##1{\textcolor[rgb]{0.00,0.50,0.00}{##1}}}
\expandafter\def\csname PY@tok@bp\endcsname{\def\PY@tc##1{\textcolor[rgb]{0.00,0.50,0.00}{##1}}}
\expandafter\def\csname PY@tok@fm\endcsname{\def\PY@tc##1{\textcolor[rgb]{0.00,0.00,1.00}{##1}}}
\expandafter\def\csname PY@tok@vc\endcsname{\def\PY@tc##1{\textcolor[rgb]{0.10,0.09,0.49}{##1}}}
\expandafter\def\csname PY@tok@vg\endcsname{\def\PY@tc##1{\textcolor[rgb]{0.10,0.09,0.49}{##1}}}
\expandafter\def\csname PY@tok@vi\endcsname{\def\PY@tc##1{\textcolor[rgb]{0.10,0.09,0.49}{##1}}}
\expandafter\def\csname PY@tok@vm\endcsname{\def\PY@tc##1{\textcolor[rgb]{0.10,0.09,0.49}{##1}}}
\expandafter\def\csname PY@tok@sa\endcsname{\def\PY@tc##1{\textcolor[rgb]{0.73,0.13,0.13}{##1}}}
\expandafter\def\csname PY@tok@sb\endcsname{\def\PY@tc##1{\textcolor[rgb]{0.73,0.13,0.13}{##1}}}
\expandafter\def\csname PY@tok@sc\endcsname{\def\PY@tc##1{\textcolor[rgb]{0.73,0.13,0.13}{##1}}}
\expandafter\def\csname PY@tok@dl\endcsname{\def\PY@tc##1{\textcolor[rgb]{0.73,0.13,0.13}{##1}}}
\expandafter\def\csname PY@tok@s2\endcsname{\def\PY@tc##1{\textcolor[rgb]{0.73,0.13,0.13}{##1}}}
\expandafter\def\csname PY@tok@sh\endcsname{\def\PY@tc##1{\textcolor[rgb]{0.73,0.13,0.13}{##1}}}
\expandafter\def\csname PY@tok@s1\endcsname{\def\PY@tc##1{\textcolor[rgb]{0.73,0.13,0.13}{##1}}}
\expandafter\def\csname PY@tok@mb\endcsname{\def\PY@tc##1{\textcolor[rgb]{0.40,0.40,0.40}{##1}}}
\expandafter\def\csname PY@tok@mf\endcsname{\def\PY@tc##1{\textcolor[rgb]{0.40,0.40,0.40}{##1}}}
\expandafter\def\csname PY@tok@mh\endcsname{\def\PY@tc##1{\textcolor[rgb]{0.40,0.40,0.40}{##1}}}
\expandafter\def\csname PY@tok@mi\endcsname{\def\PY@tc##1{\textcolor[rgb]{0.40,0.40,0.40}{##1}}}
\expandafter\def\csname PY@tok@il\endcsname{\def\PY@tc##1{\textcolor[rgb]{0.40,0.40,0.40}{##1}}}
\expandafter\def\csname PY@tok@mo\endcsname{\def\PY@tc##1{\textcolor[rgb]{0.40,0.40,0.40}{##1}}}
\expandafter\def\csname PY@tok@ch\endcsname{\let\PY@it=\textit\def\PY@tc##1{\textcolor[rgb]{0.25,0.50,0.50}{##1}}}
\expandafter\def\csname PY@tok@cm\endcsname{\let\PY@it=\textit\def\PY@tc##1{\textcolor[rgb]{0.25,0.50,0.50}{##1}}}
\expandafter\def\csname PY@tok@cpf\endcsname{\let\PY@it=\textit\def\PY@tc##1{\textcolor[rgb]{0.25,0.50,0.50}{##1}}}
\expandafter\def\csname PY@tok@c1\endcsname{\let\PY@it=\textit\def\PY@tc##1{\textcolor[rgb]{0.25,0.50,0.50}{##1}}}
\expandafter\def\csname PY@tok@cs\endcsname{\let\PY@it=\textit\def\PY@tc##1{\textcolor[rgb]{0.25,0.50,0.50}{##1}}}

\def\PYZbs{\char`\\}
\def\PYZus{\char`\_}
\def\PYZob{\char`\{}
\def\PYZcb{\char`\}}
\def\PYZca{\char`\^}
\def\PYZam{\char`\&}
\def\PYZlt{\char`\<}
\def\PYZgt{\char`\>}
\def\PYZsh{\char`\#}
\def\PYZpc{\char`\%}
\def\PYZdl{\char`\$}
\def\PYZhy{\char`\-}
\def\PYZsq{\char`\'}
\def\PYZdq{\char`\"}
\def\PYZti{\char`\~}
% for compatibility with earlier versions
\def\PYZat{@}
\def\PYZlb{[}
\def\PYZrb{]}
\makeatother


    % Exact colors from NB
    \definecolor{incolor}{rgb}{0.0, 0.0, 0.5}
    \definecolor{outcolor}{rgb}{0.545, 0.0, 0.0}



    
    % Prevent overflowing lines due to hard-to-break entities
    \sloppy 
    % Setup hyperref package
    \hypersetup{
      breaklinks=true,  % so long urls are correctly broken across lines
      colorlinks=true,
      urlcolor=urlcolor,
      linkcolor=linkcolor,
      citecolor=citecolor,
      }
    % Slightly bigger margins than the latex defaults
    
    \geometry{verbose,tmargin=1in,bmargin=1in,lmargin=1in,rmargin=1in}
    
    

    \begin{document}
    
    
    \maketitle
    
    

    
    \hypertarget{implementaciuxf3n-del-muxe9todo}{%
\section{Implementación del
método}\label{implementaciuxf3n-del-muxe9todo}}

    Este notebook muestra paso a paso cómo funciona el método de
parametrización para el mapeo estándar. Recordemos que antes ya se
calcularon los puntos fijos del mapeo y se hizo un análisis de los
mismos por lo que nos concentraremos en sólo aplicar de manera clara y
paso a paso el método. Hacerlo de esta manera nos ayudará a comprender
mejor cómo se implementó método de manera general. Dado que conocemos ya
los puntos fijos entonces nos concentraremos en calcular la variedad
inestable, la cual se podrá elegir en términos de escoger el valor
propio con el signo negativo que ya se calculó en el N3.

En primer lugar hay que notar que como queremos implementar eso de
manera general tenemos que hacer el cálculo de los valores propios en un
inicio. Luego de esto se puede proceder con el calculo de cada variedad.

    Como ya sabemos cuáles son los puntos fijos necesitamos ahora calcular
los polinomios que serán la parametrización de la variedad.Para ello
debemos resolver la ecuación de invariancia.
\[F_{\epsilon}\circ P=P\circ \lambda\]

    \begin{Verbatim}[commandchars=\\\{\}]
{\color{incolor}In [{\color{incolor}1}]:} \PY{c}{\PYZsh{}usaremos el paquete de TaylorSeries}
        \PY{k}{using} \PY{n}{TaylorSeries}
\end{Verbatim}


    \begin{Verbatim}[commandchars=\\\{\}]
{\color{incolor}In [{\color{incolor}3}]:} \PY{c}{\PYZsh{}necesitamos tambien definir el mapeo estandar }
        \PY{l+s}{\PYZdq{}\PYZdq{}\PYZdq{}}\PY{l+s}{E}\PY{l+s}{s}\PY{l+s}{t}\PY{l+s}{a}\PY{l+s}{n}\PY{l+s}{d}\PY{l+s}{a}\PY{l+s}{r}\PY{l+s}{M}\PY{l+s}{a}\PY{l+s}{p}\PY{l+s}{(}\PY{l+s}{θ}\PY{l+s}{,}\PY{l+s}{p}\PY{l+s}{,}\PY{l+s}{k}\PY{l+s}{)}
        \PY{l+s}{ }\PY{l+s}{ }\PY{l+s}{ }
        \PY{l+s}{ }\PY{l+s}{ }\PY{l+s}{ }\PY{l+s}{F}\PY{l+s}{u}\PY{l+s}{n}\PY{l+s}{c}\PY{l+s}{i}\PY{l+s}{ó}\PY{l+s}{n}\PY{l+s}{ }\PY{l+s}{q}\PY{l+s}{u}\PY{l+s}{e}\PY{l+s}{ }\PY{l+s}{d}\PY{l+s}{e}\PY{l+s}{f}\PY{l+s}{i}\PY{l+s}{n}\PY{l+s}{e}\PY{l+s}{ }\PY{l+s}{e}\PY{l+s}{l}\PY{l+s}{ }\PY{l+s}{m}\PY{l+s}{a}\PY{l+s}{p}\PY{l+s}{e}\PY{l+s}{o}\PY{l+s}{ }\PY{l+s}{e}\PY{l+s}{s}\PY{l+s}{t}\PY{l+s}{á}\PY{l+s}{n}\PY{l+s}{d}\PY{l+s}{a}\PY{l+s}{r}\PY{l+s}{.}
        \PY{l+s}{ }\PY{l+s}{ }
        
        \PY{l+s}{ }\PY{l+s}{ }\PY{l+s}{ }\PY{l+s}{A}\PY{l+s}{r}\PY{l+s}{g}\PY{l+s}{u}\PY{l+s}{m}\PY{l+s}{e}\PY{l+s}{n}\PY{l+s}{t}\PY{l+s}{o}\PY{l+s}{s}\PY{l+s}{:}
        \PY{l+s}{ }\PY{l+s}{ }\PY{l+s}{ }\PY{l+s}{\PYZhy{}}\PY{l+s}{θ}\PY{l+s}{ }\PY{l+s}{:}\PY{l+s}{ }\PY{l+s}{p}\PY{l+s}{o}\PY{l+s}{s}\PY{l+s}{i}\PY{l+s}{c}\PY{l+s}{i}\PY{l+s}{ó}\PY{l+s}{n}
        \PY{l+s}{ }\PY{l+s}{ }\PY{l+s}{ }\PY{l+s}{\PYZhy{}}\PY{l+s}{p}\PY{l+s}{ }\PY{l+s}{:}\PY{l+s}{ }\PY{l+s}{m}\PY{l+s}{o}\PY{l+s}{m}\PY{l+s}{e}\PY{l+s}{n}\PY{l+s}{t}\PY{l+s}{o}
        \PY{l+s}{ }\PY{l+s}{ }\PY{l+s}{ }\PY{l+s}{\PYZhy{}}\PY{l+s}{k}\PY{l+s}{ }\PY{l+s}{:}\PY{l+s}{ }\PY{l+s}{c}\PY{l+s}{o}\PY{l+s}{n}\PY{l+s}{s}\PY{l+s}{t}\PY{l+s}{a}\PY{l+s}{n}\PY{l+s}{t}\PY{l+s}{e}\PY{l+s}{ }\PY{l+s}{d}\PY{l+s}{e}\PY{l+s}{l}\PY{l+s}{ }\PY{l+s}{m}\PY{l+s}{a}\PY{l+s}{p}\PY{l+s}{e}\PY{l+s}{o}
        \PY{l+s}{\PYZdq{}\PYZdq{}\PYZdq{}}
        \PY{k}{function} \PY{n}{EstandarMap}\PY{p}{(}\PY{n}{θ}\PY{p}{,}\PY{n}{p}\PY{p}{,}\PY{n}{k}\PY{p}{)}
            
            \PY{n}{θ\PYZus{}n} \PY{o}{=} \PY{n}{mod2pi}\PY{p}{(}\PY{n}{θ}\PY{o}{+}\PY{n}{p}\PY{p}{)}
            \PY{n}{p\PYZus{}n} \PY{o}{=} \PY{n}{mod2pi}\PY{p}{(}\PY{n}{p}\PY{o}{+}\PY{n}{k}\PY{o}{*}\PY{n}{sin}\PY{p}{(}\PY{n}{θ\PYZus{}n}\PY{p}{)}\PY{p}{)}
            
            \PY{k}{return} \PY{p}{[}\PY{n}{θ\PYZus{}n}\PY{p}{,}\PY{n}{p\PYZus{}n}\PY{p}{]}
            
            
        \PY{k}{end}
\end{Verbatim}


    \begin{Verbatim}[commandchars=\\\{\}]
\textcolor{ansi-yellow-intense}{\textbf{WARNING: }}\textcolor{ansi-yellow}{replacing docs for 'EstandarMap :: Tuple\{Any,Any,Any\}' in module 'Main'.}

    \end{Verbatim}

\begin{Verbatim}[commandchars=\\\{\}]
{\color{outcolor}Out[{\color{outcolor}3}]:} EstandarMap
\end{Verbatim}
            
    Definimos el valor de la constante del mapeo que usaremos en este caso.

    \begin{Verbatim}[commandchars=\\\{\}]
{\color{incolor}In [{\color{incolor}4}]:} \PY{n}{ke}\PY{o}{=}\PY{l+m+mf}{0.3}
\end{Verbatim}


\begin{Verbatim}[commandchars=\\\{\}]
{\color{outcolor}Out[{\color{outcolor}4}]:} 0.3
\end{Verbatim}
            
    \begin{verbatim}
                                                ORDEN 1 
\end{verbatim}

    Dado que queremos que \(P_{x}\), \(P_{p}\) sean polinomios necesitamos
construirlos como dos polinomios de los cuales solo sabemos su primer
coeficiente pero la idea es ir calculando los términos en cada
iteración. Escribirmos los coeficientes que buscamos como un polinomio a
su vez de orden mayor a uno.

    \begin{Verbatim}[commandchars=\\\{\}]
{\color{incolor}In [{\color{incolor}5}]:} \PY{c}{\PYZsh{}Queremos que los polinomios para θ, p sean de orden uno, eso no influye en el orden del coeficiente que estamos calculando}
        \PY{n}{θ}\PY{p}{,}\PY{n}{p} \PY{o}{=} \PY{n}{set\PYZus{}variables}\PY{p}{(}\PY{k+kt}{Float64}\PY{p}{,}\PY{l+s}{\PYZdq{}}\PY{l+s}{θ}\PY{l+s}{ }\PY{l+s}{p}\PY{l+s}{\PYZdq{}}\PY{p}{,}\PY{n}{order}\PY{o}{=}\PY{l+m+mi}{2}\PY{p}{)}
\end{Verbatim}


\begin{Verbatim}[commandchars=\\\{\}]
{\color{outcolor}Out[{\color{outcolor}5}]:} 2-element Array\{TaylorSeries.TaylorN\{Float64\},1\}:
          1.0 θ + 𝒪(‖x‖³)
          1.0 p + 𝒪(‖x‖³)
\end{Verbatim}
            
    \begin{Verbatim}[commandchars=\\\{\}]
{\color{incolor}In [{\color{incolor}6}]:} \PY{c}{\PYZsh{}Ahora necesitamos los polinomios que parametrizan a las variedades en específico, estos serán tipo Taylor1.TaylorN}
        \PY{n}{P\PYZus{}θ}\PY{o}{=}\PY{n}{Taylor1}\PY{p}{(}\PY{p}{[}\PY{n}{θ}\PY{p}{]}\PY{p}{,} \PY{l+m+mi}{1}\PY{p}{)}
        \PY{n}{P\PYZus{}p}\PY{o}{=}\PY{n}{Taylor1}\PY{p}{(}\PY{p}{[}\PY{n}{p}\PY{p}{]}\PY{p}{,} \PY{l+m+mi}{1}\PY{p}{)}
        \PY{c}{\PYZsh{}el orden de estos ya es importante para cada iteración, aquí es uno porque estamos en el coeficiente de primer grado}
        \PY{n}{print}\PY{p}{(}\PY{n}{P\PYZus{}θ}\PY{p}{,}\PY{n}{P\PYZus{}p}\PY{p}{)}
\end{Verbatim}


    \begin{Verbatim}[commandchars=\\\{\}]
  1.0 θ + 𝒪(‖x‖³) + 𝒪(t²)  1.0 p + 𝒪(‖x‖³) + 𝒪(t²)
    \end{Verbatim}

    \begin{Verbatim}[commandchars=\\\{\}]
{\color{incolor}In [{\color{incolor}7}]:} \PY{c}{\PYZsh{}Aplicamos el mapeo a estos dos polinomios}
        \PY{n}{Orden1}\PY{o}{=}\PY{n}{EstandarMap}\PY{p}{(}\PY{n}{P\PYZus{}θ}\PY{p}{,}\PY{n}{P\PYZus{}p}\PY{p}{,}\PY{n}{ke}\PY{p}{)}
\end{Verbatim}


\begin{Verbatim}[commandchars=\\\{\}]
{\color{outcolor}Out[{\color{outcolor}7}]:} 2-element Array\{TaylorSeries.Taylor1\{TaylorSeries.TaylorN\{Float64\}\},1\}:
           1.0 θ + 1.0 p + 𝒪(‖x‖³) + 𝒪(t²)
           0.3 θ + 1.3 p + 𝒪(‖x‖³) + 𝒪(t²)
\end{Verbatim}
            
    Hasta este momento tenemos calculado la parte de la ecuación
cohomológica que corresponde a la composición de
\(f_{\epsilon} \circ P\), es decir creamos un arreglo de polinomios P
que después evaluamos en la función \(f_{\epsilon}\). Construiremos la
otra composición que aparece en la ecuación cohomológica más adelante.
Antes de eso debemos pensar cómo se resolverá esta ecuación, la forma
más fácil es escribirla como un sistema en forma matricial.
\[\mathbb{A}\textbf{v}=\textbf{w} \] Donde la matriz A contiene los
coeficientes que aparecen en los polinomios de acuerdo a cada variable.
Mientras que la x representa el vector \((\theta,p)\) y \(w\) el vector
de términos independientes de los polinomios.

    Para escribirlo de esta manera usaremos una función dentro de
TaylorSeries que es el jacobiano. El jacobiano como recordamos es una
matriz formada por las derivadas parciales de los dos polinomios para
cada una de las variables. En este caso dos variables y dos polinomios
nos llevan a una matriz de \(2 \times 2\) formada por :
\[ \left( \begin{array}{cc}
\frac{\partial f_{x}}{\partial \theta} & \frac{\partial f_{p}}{\partial \theta}  \\
\frac{\partial f_{x}}{\partial p} & \frac{\partial f_{p}}{\partial p}  \end{array} \right) \]

    \begin{Verbatim}[commandchars=\\\{\}]
{\color{incolor}In [{\color{incolor}8}]:} \PY{c}{\PYZsh{}al calcular el jacobiano obtendremos sólo los valores de los coeficientes los cuales }
        \PY{c}{\PYZsh{} son de tipo TaylorN}
        \PY{n}{JPO}\PY{o}{=}\PY{n}{jacobian}\PY{p}{(}\PY{n}{Orden1}\PY{p}{)}
\end{Verbatim}


\begin{Verbatim}[commandchars=\\\{\}]
{\color{outcolor}Out[{\color{outcolor}8}]:} 2×2 Array\{TaylorSeries.Taylor1\{Float64\},2\}:
          1.0 + 𝒪(t²)   1.0 + 𝒪(t²)
          0.3 + 𝒪(t²)   1.3 + 𝒪(t²)
\end{Verbatim}
            
    \begin{Verbatim}[commandchars=\\\{\}]
{\color{incolor}In [{\color{incolor}9}]:} \PY{c}{\PYZsh{}como los valores que contiene el jacobiano son de tipo TaylorN y para usarlos como necesitamos}
        \PY{c}{\PYZsh{} se deben tener en Float64 vamos a obtenerlos de la siguente manera. }
        \PY{n}{JacFl} \PY{o}{=} \PY{k+kt}{Array}\PY{p}{\PYZob{}}\PY{k+kt}{Float64}\PY{p}{\PYZcb{}}\PY{p}{(}\PY{l+m+mi}{2}\PY{p}{,}\PY{l+m+mi}{2}\PY{p}{)}
        \PY{k}{for} \PY{n}{ind} \PY{k+kp}{in} \PY{n}{eachindex}\PY{p}{(}\PY{n}{JPO}\PY{p}{)}
                        \PY{n}{JacFl}\PY{p}{[}\PY{n}{ind}\PY{p}{]} \PY{o}{=} \PY{n}{JPO}\PY{p}{[}\PY{n}{ind}\PY{p}{]}\PY{o}{.}\PY{n}{coeffs}\PY{p}{[}\PY{l+m+mi}{1}\PY{p}{]}
        \PY{k}{end}
        \PY{c}{\PYZsh{}usamos la matriz JacFl como auxiliar para encontrar los valores flotantes}
        \PY{n+nd}{@show}\PY{p}{(}\PY{n}{JacFl}\PY{p}{)}
\end{Verbatim}


    \begin{Verbatim}[commandchars=\\\{\}]
JacFl = [1.0 1.0; 0.3 1.3]

    \end{Verbatim}

\begin{Verbatim}[commandchars=\\\{\}]
{\color{outcolor}Out[{\color{outcolor}9}]:} 2×2 Array\{Float64,2\}:
         1.0  1.0
         0.3  1.3
\end{Verbatim}
            
    La matriz que aparece arriba es una que ya podemos usar para calcular
los valores y vectores propios. Necesitamos sólo uno de ellos.

    \begin{Verbatim}[commandchars=\\\{\}]
{\color{incolor}In [{\color{incolor}10}]:} \PY{c}{\PYZsh{} eig () es una función que calcula los valores y vectores propios de una matriz y los da en ese orden en forma de vector.}
         \PY{n}{JPO}\PY{o}{=}\PY{n}{JacFl}
         \PY{n}{eval}\PY{p}{,}\PY{n}{eve}\PY{o}{=}\PY{n}{eig}\PY{p}{(}\PY{n}{JPO}\PY{p}{)}
\end{Verbatim}


    \begin{Verbatim}[commandchars=\\\{\}]
WARNING: imported binding for eval overwritten in module Main

    \end{Verbatim}

\begin{Verbatim}[commandchars=\\\{\}]
{\color{outcolor}Out[{\color{outcolor}10}]:} ([0.582109, 1.71789], [-0.922675 -0.812346; 0.385578 -0.583176])
\end{Verbatim}
            
    \begin{Verbatim}[commandchars=\\\{\}]
{\color{incolor}In [{\color{incolor}11}]:} \PY{c}{\PYZsh{} dado que usaremos la variedad inestable, entonces usamos el segundo valor propio}
         \PY{n}{λ}\PY{o}{=}\PY{n}{eval}\PY{p}{[}\PY{l+m+mi}{2}\PY{p}{]}
         \PY{n}{print}\PY{p}{(}\PY{n}{λ}\PY{p}{,}\PY{n}{eve}\PY{p}{)}
\end{Verbatim}


    \begin{Verbatim}[commandchars=\\\{\}]
1.7178908345800274[-0.922675 -0.812346; 0.385578 -0.583176]
    \end{Verbatim}

    Tomamos por ende el segundo vector propio, los coeficientes de este
vector serán los coeficientes de primer orden en los polinomios.

    \begin{Verbatim}[commandchars=\\\{\}]
{\color{incolor}In [{\color{incolor}12}]:} \PY{n}{vec}\PY{o}{=}\PY{n}{eve}\PY{p}{[}\PY{o}{:}\PY{p}{,}\PY{l+m+mi}{2}\PY{p}{]}
         \PY{c}{\PYZsh{} decimos que vec es un arreglo de contiene los valores correspondientes a los lugares [1,2], [2,2] de eve.}
\end{Verbatim}


\begin{Verbatim}[commandchars=\\\{\}]
{\color{outcolor}Out[{\color{outcolor}12}]:} 2-element Array\{Float64,1\}:
          -0.812346
          -0.583176
\end{Verbatim}
            
    \begin{Verbatim}[commandchars=\\\{\}]
{\color{incolor}In [{\color{incolor}13}]:} \PY{c}{\PYZsh{}queremos revisar la estabilidad de tipo de las variables Pθ,x}
         \PY{n+nd}{@show}\PY{p}{(}\PY{n}{P\PYZus{}θ}\PY{p}{)}
         \PY{n+nd}{@show}\PY{p}{(}\PY{n}{typeof}\PY{p}{(}\PY{n}{P\PYZus{}θ}\PY{p}{)}\PY{p}{)}
         \PY{n+nd}{@show}\PY{p}{(}\PY{n}{Orden1}\PY{p}{[}\PY{l+m+mi}{1}\PY{p}{]}\PY{p}{)}
         \PY{n+nd}{@show}\PY{p}{(}\PY{n}{typeof}\PY{p}{(}\PY{n}{Orden1}\PY{p}{[}\PY{l+m+mi}{1}\PY{p}{]}\PY{p}{)}\PY{p}{)}
\end{Verbatim}


    \begin{Verbatim}[commandchars=\\\{\}]
P\_θ =   1.0 θ + 𝒪(‖x‖³) + 𝒪(t²)
typeof(P\_θ) = TaylorSeries.Taylor1\{TaylorSeries.TaylorN\{Float64\}\}
Orden1[1] =   1.0 θ + 1.0 p + 𝒪(‖x‖³) + 𝒪(t²)
typeof(Orden1[1]) = TaylorSeries.Taylor1\{TaylorSeries.TaylorN\{Float64\}\}

    \end{Verbatim}

\begin{Verbatim}[commandchars=\\\{\}]
{\color{outcolor}Out[{\color{outcolor}13}]:} TaylorSeries.Taylor1\{TaylorSeries.TaylorN\{Float64\}\}
\end{Verbatim}
            
    \begin{verbatim}
                                              ORDEN 2
\end{verbatim}

    A partir de este término la forma de obtener los coeficientes se vuelve
monótona, esto facilita que podamos sintetizar todos los pasos en
funciones. Antes de eso calcularemos cada término por separado. La forma
de comenzar a calcular cada coeficiente no cambia con respecto al
primero.

    \begin{Verbatim}[commandchars=\\\{\}]
{\color{incolor}In [{\color{incolor}16}]:} \PY{c}{\PYZsh{}Queremos hacer la misma idea para este orden, recordemos que el orden de este polinomio no influye por ahora en el cálculo}
         \PY{c}{\PYZsh{} ya que en los términos que no aparecen explícitamente sirven para justo calcular la parte que no conocemos.}
         
         \PY{n}{θ}\PY{p}{,}\PY{n}{p} \PY{o}{=} \PY{n}{set\PYZus{}variables}\PY{p}{(}\PY{k+kt}{Float64}\PY{p}{,}\PY{l+s}{\PYZdq{}}\PY{l+s}{θ}\PY{l+s}{ }\PY{l+s}{p}\PY{l+s}{\PYZdq{}}\PY{p}{,}\PY{n}{order}\PY{o}{=}\PY{l+m+mi}{1}\PY{p}{)}
\end{Verbatim}


\begin{Verbatim}[commandchars=\\\{\}]
{\color{outcolor}Out[{\color{outcolor}16}]:} 2-element Array\{TaylorSeries.TaylorN\{Float64\},1\}:
           1.0 θ + 𝒪(‖x‖²)
           1.0 p + 𝒪(‖x‖²)
\end{Verbatim}
            
    Para los nuevos polinomios agregamos los valores que calculamos en el
paso anterior, además de las variables x,p como series de Taylor.

    \begin{Verbatim}[commandchars=\\\{\}]
{\color{incolor}In [{\color{incolor}17}]:} \PY{c}{\PYZsh{}usamos los valores ya calculados en el orden 1 para x,p de primer orden, estos los incluyo en una lista junto }
         \PY{c}{\PYZsh{}con  el polinomio que representa a x,p  tipo TaylorN }
         \PY{n}{P\PYZus{}θ}\PY{o}{=}\PY{n}{Taylor1}\PY{p}{(}\PY{p}{[}\PY{l+m+mf}{0.}\PY{p}{,}\PY{n}{vec}\PY{p}{[}\PY{l+m+mi}{1}\PY{p}{]}\PY{p}{,}\PY{n}{θ}\PY{p}{]}\PY{p}{,}\PY{l+m+mi}{2}\PY{p}{)}
         \PY{n}{P\PYZus{}p}\PY{o}{=}\PY{n}{Taylor1}\PY{p}{(}\PY{p}{[}\PY{l+m+mf}{0.}\PY{p}{,}\PY{n}{vec}\PY{p}{[}\PY{l+m+mi}{2}\PY{p}{]}\PY{p}{,}\PY{n}{p}\PY{p}{]}\PY{p}{,}\PY{l+m+mi}{2}\PY{p}{)}
\end{Verbatim}


\begin{Verbatim}[commandchars=\\\{\}]
{\color{outcolor}Out[{\color{outcolor}17}]:}  ( - 0.5831757547123116 + 𝒪(‖x‖¹)) t + ( 1.0 p + 𝒪(‖x‖²)) t² + 𝒪(t³)
\end{Verbatim}
            
    \begin{Verbatim}[commandchars=\\\{\}]
{\color{incolor}In [{\color{incolor}18}]:} \PY{c}{\PYZsh{}aplicamos el mapeo}
         \PY{n}{SO}\PY{o}{=}\PY{n}{EstandarMap}\PY{p}{(}\PY{n}{P\PYZus{}θ}\PY{p}{,}\PY{n}{P\PYZus{}p}\PY{p}{,}\PY{n}{ke}\PY{p}{)}
\end{Verbatim}


\begin{Verbatim}[commandchars=\\\{\}]
{\color{outcolor}Out[{\color{outcolor}18}]:} 2-element Array\{TaylorSeries.Taylor1\{TaylorSeries.TaylorN\{Float64\}\},1\}:
           ( - 1.3955217641908624 + 𝒪(‖x‖¹)) t + ( 1.0 θ + 1.0 p + 𝒪(‖x‖²)) t² + 𝒪(t³)
           ( - 1.0018322839695704 + 𝒪(‖x‖¹)) t + ( 0.3 θ + 1.3 p + 𝒪(‖x‖²)) t² + 𝒪(t³)
\end{Verbatim}
            
    Para el lado derecho de la ecuación cohomológica necesitamos un vector
de lambdas, pues recordemos\\
\[ \Sigma_{n=1}^{\inf}a_{n}t^{n}\lambda^{n}\]
\[\Sigma_{n=1}^{\inf}b_{n}t^{n}\lambda^{n}\]

    \begin{Verbatim}[commandchars=\\\{\}]
{\color{incolor}In [{\color{incolor}19}]:} \PY{c}{\PYZsh{} como nos falta el otro lado de la ecuación cohomológica definimos un vector de lambdas}
         \PY{n}{vλ}\PY{o}{=}\PY{p}{[}\PY{l+m+mf}{0.0}\PY{p}{,}\PY{n}{λ}\PY{p}{,}\PY{n}{λ}\PY{o}{\PYZca{}}\PY{l+m+mi}{2}\PY{p}{]}
\end{Verbatim}


\begin{Verbatim}[commandchars=\\\{\}]
{\color{outcolor}Out[{\color{outcolor}19}]:} 3-element Array\{Float64,1\}:
          0.0    
          1.71789
          2.95115
\end{Verbatim}
            
    \begin{Verbatim}[commandchars=\\\{\}]
{\color{incolor}In [{\color{incolor}20}]:} \PY{c}{\PYZsh{}y definimos también un polinomio para las lambdas, del mismo orden que el polinomio de P\PYZus{}θ,P\PYZus{}p}
         \PY{c}{\PYZsh{} que juega la parte de a\PYZus{}\PYZob{}n\PYZcb{}\PYZbs{}lambda\PYZca{}\PYZob{}n\PYZcb{}}
         \PY{n}{θλt}\PY{o}{=}\PY{n}{Taylor1}\PY{p}{(}\PY{n}{vλ}\PY{o}{.*}\PY{p}{[}\PY{l+m+mf}{0.}\PY{p}{,}\PY{n}{vec}\PY{p}{[}\PY{l+m+mi}{1}\PY{p}{]}\PY{p}{,}\PY{n}{θ}\PY{p}{]}\PY{p}{,}\PY{l+m+mi}{2}\PY{p}{)}
         \PY{n}{pλt}\PY{o}{=}\PY{n}{Taylor1}\PY{p}{(}\PY{n}{vλ}\PY{o}{.*}\PY{p}{[}\PY{l+m+mf}{0.}\PY{p}{,}\PY{n}{vec}\PY{p}{[}\PY{l+m+mi}{2}\PY{p}{]}\PY{p}{,}\PY{n}{p}\PY{p}{]}\PY{p}{,}\PY{l+m+mi}{2}\PY{p}{)}
         \PY{n}{λvec}\PY{o}{=}\PY{p}{[}\PY{n}{θλt}\PY{p}{,}\PY{n}{pλt}\PY{p}{]}
\end{Verbatim}


\begin{Verbatim}[commandchars=\\\{\}]
{\color{outcolor}Out[{\color{outcolor}20}]:} 2-element Array\{TaylorSeries.Taylor1\{TaylorSeries.TaylorN\{Float64\}\},1\}:
           ( - 1.3955217641908624 + 𝒪(‖x‖¹)) t + ( 2.9511489195340634 θ + 𝒪(‖x‖²)) t² + 𝒪(t³)
           ( - 1.0018322839695704 + 𝒪(‖x‖¹)) t + ( 2.9511489195340634 p + 𝒪(‖x‖²)) t² + 𝒪(t³)
\end{Verbatim}
            
    Ahora ya tengo las dos partes de la ecuación y debo igualarlas para
resolver. SO representa la parte de la primera composición mientras que
\(λvec\) representa la parte derecha de la ecuación de invariancia. Al
restarlas estoy resolviendo \[f_{\epsilon}\circ P- P \circ \lambda=0 \].
En este caso no esta igualada a cero puesto que estamos tratando de
calcular el siguente término, es decir van a diferir justo en ese
término.

    \begin{Verbatim}[commandchars=\\\{\}]
{\color{incolor}In [{\color{incolor}21}]:} \PY{c}{\PYZsh{}Restando ambos vectores que representan la ecuación cohomológica encontraremos los términos que faltan}
         \PY{c}{\PYZsh{} de empatar}
         \PY{n}{Ecua}\PY{o}{=}\PY{n}{SO}\PY{o}{\PYZhy{}}\PY{n}{λvec}
\end{Verbatim}


\begin{Verbatim}[commandchars=\\\{\}]
{\color{outcolor}Out[{\color{outcolor}21}]:} 2-element Array\{TaylorSeries.Taylor1\{TaylorSeries.TaylorN\{Float64\}\},1\}:
           ( - 1.9511489195340634 θ + 1.0 p + 𝒪(‖x‖²)) t² + 𝒪(t³)
             ( 0.3 θ - 1.6511489195340634 p + 𝒪(‖x‖²)) t² + 𝒪(t³)
\end{Verbatim}
            
    En este paso nos podemos dar cuenta que sólo aprece el término de orden
dos puesto que el de orden uno ya lo calculamos.

Notemos que hay un pequeño error, pues aparece un número en el polinomio
asociado a \(\theta\) de orden 1, sin embargo este número es del orden
de \(10^{-16}\) lo cual nos dice que es un error numérico.

    \begin{Verbatim}[commandchars=\\\{\}]
{\color{incolor}In [{\color{incolor}22}]:} \PY{c}{\PYZsh{} de esta ecuación necesitamos solo los de segundo orden, así que los extraemos manualmente }
         \PY{n}{θ2}\PY{o}{=}\PY{n}{Ecua}\PY{p}{[}\PY{l+m+mi}{1}\PY{p}{]}\PY{o}{.}\PY{n}{coeffs}\PY{p}{[}\PY{l+m+mi}{3}\PY{p}{]}
         \PY{n}{p2}\PY{o}{=}\PY{n}{Ecua}\PY{p}{[}\PY{l+m+mi}{2}\PY{p}{]}\PY{o}{.}\PY{n}{coeffs}\PY{p}{[}\PY{l+m+mi}{3}\PY{p}{]}
         \PY{n}{vec2}\PY{o}{=}\PY{p}{[}\PY{n}{θ2}\PY{p}{,}\PY{n}{p2}\PY{p}{]}
\end{Verbatim}


\begin{Verbatim}[commandchars=\\\{\}]
{\color{outcolor}Out[{\color{outcolor}22}]:} 2-element Array\{TaylorSeries.TaylorN\{Float64\},1\}:
           - 1.9511489195340634 θ + 1.0 p + 𝒪(‖x‖²)
             0.3 θ - 1.6511489195340634 p + 𝒪(‖x‖²)
\end{Verbatim}
            
    \begin{Verbatim}[commandchars=\\\{\}]
{\color{incolor}In [{\color{incolor}23}]:} \PY{c}{\PYZsh{}calculamos ahora el jacobiano de este vector, ya que los coeficientes serán parte de la matriz que queremos}
         \PY{n}{JSO}\PY{o}{=}\PY{n}{jacobian}\PY{p}{(}\PY{n}{vec2}\PY{p}{)}
\end{Verbatim}


\begin{Verbatim}[commandchars=\\\{\}]
{\color{outcolor}Out[{\color{outcolor}23}]:} 2×2 Array\{Float64,2\}:
          -1.95115   1.0    
           0.3      -1.65115
\end{Verbatim}
            
    \begin{Verbatim}[commandchars=\\\{\}]
{\color{incolor}In [{\color{incolor}24}]:} \PY{c}{\PYZsh{}calculamos su determinante para ver si es cero, }
         \PY{n}{det}\PY{p}{(}\PY{n}{JSO}\PY{p}{)}
\end{Verbatim}


\begin{Verbatim}[commandchars=\\\{\}]
{\color{outcolor}Out[{\color{outcolor}24}]:} 2.9216374303387243
\end{Verbatim}
            
    \begin{Verbatim}[commandchars=\\\{\}]
{\color{incolor}In [{\color{incolor}25}]:} \PY{c}{\PYZsh{} como es distinto de cero y la ecuación esta igualada a cero entonces la única solución es la trivial}
         \PY{n}{θ2}\PY{o}{=}\PY{l+m+mf}{0.}
         \PY{n}{p2}\PY{o}{=}\PY{l+m+mf}{0.}
         \PY{n}{vec2}\PY{o}{=}\PY{p}{[}\PY{l+m+mf}{0.}\PY{p}{,}\PY{l+m+mf}{0.}\PY{p}{]}
\end{Verbatim}


\begin{Verbatim}[commandchars=\\\{\}]
{\color{outcolor}Out[{\color{outcolor}25}]:} 2-element Array\{Float64,1\}:
          0.0
          0.0
\end{Verbatim}
            
    Esto concuerda con lo que habíamos encontrado antes, los coeficientes de
orden par son cero.

    \begin{Verbatim}[commandchars=\\\{\}]
{\color{incolor}In [{\color{incolor}26}]:} \PY{c}{\PYZsh{}queremos revisar la estabilidad de las variables tx,x}
         \PY{n+nd}{@show}\PY{p}{(}\PY{n}{P\PYZus{}θ}\PY{p}{)}
         \PY{n+nd}{@show}\PY{p}{(}\PY{n}{typeof}\PY{p}{(}\PY{n}{P\PYZus{}θ}\PY{p}{)}\PY{p}{)}
         \PY{n+nd}{@show}\PY{p}{(}\PY{n}{SO}\PY{p}{[}\PY{l+m+mi}{1}\PY{p}{]}\PY{p}{)}
         \PY{n+nd}{@show}\PY{p}{(}\PY{n}{typeof}\PY{p}{(}\PY{n}{SO}\PY{p}{[}\PY{l+m+mi}{1}\PY{p}{]}\PY{p}{)}\PY{p}{)}
         \PY{n+nd}{@show}\PY{p}{(}\PY{n}{θλt}\PY{p}{)}
         \PY{n+nd}{@show}\PY{p}{(}\PY{n}{typeof}\PY{p}{(}\PY{n}{θλt}\PY{p}{)}\PY{p}{)}
         \PY{n+nd}{@show}\PY{p}{(}\PY{n}{Ecua}\PY{p}{[}\PY{l+m+mi}{1}\PY{p}{]}\PY{p}{)}
         \PY{n+nd}{@show}\PY{p}{(}\PY{n}{typeof}\PY{p}{(}\PY{n}{Ecua}\PY{p}{[}\PY{l+m+mi}{1}\PY{p}{]}\PY{p}{)}\PY{p}{)}
\end{Verbatim}


    \begin{Verbatim}[commandchars=\\\{\}]
P\_θ =  ( - 0.8123460094785507 + 𝒪(‖x‖¹)) t + ( 1.0 θ + 𝒪(‖x‖²)) t² + 𝒪(t³)
typeof(P\_θ) = TaylorSeries.Taylor1\{TaylorSeries.TaylorN\{Float64\}\}
SO[1] =  ( - 1.3955217641908624 + 𝒪(‖x‖¹)) t + ( 1.0 θ + 1.0 p + 𝒪(‖x‖²)) t² + 𝒪(t³)
typeof(SO[1]) = TaylorSeries.Taylor1\{TaylorSeries.TaylorN\{Float64\}\}
θλt =  ( - 1.3955217641908624 + 𝒪(‖x‖¹)) t + ( 2.9511489195340634 θ + 𝒪(‖x‖²)) t² + 𝒪(t³)
typeof(θλt) = TaylorSeries.Taylor1\{TaylorSeries.TaylorN\{Float64\}\}
Ecua[1] =  ( - 1.9511489195340634 θ + 1.0 p + 𝒪(‖x‖²)) t² + 𝒪(t³)
typeof(Ecua[1]) = TaylorSeries.Taylor1\{TaylorSeries.TaylorN\{Float64\}\}

    \end{Verbatim}

\begin{Verbatim}[commandchars=\\\{\}]
{\color{outcolor}Out[{\color{outcolor}26}]:} TaylorSeries.Taylor1\{TaylorSeries.TaylorN\{Float64\}\}
\end{Verbatim}
            
    \begin{verbatim}
                                            ORDEN 3
\end{verbatim}

    Para este orden comenzaremos a generar los vectores como en el caso
anterior, lo que cambia en este caso es que aparecen términos
independientes en la ecuación comohológica. Esto resulta en que las
solución ya no será la trivial como en el caso pasado.

    \begin{Verbatim}[commandchars=\\\{\}]
{\color{incolor}In [{\color{incolor}27}]:} \PY{c}{\PYZsh{}genramos un par de variables de tipo TaylorN de orden 3}
         \PY{n}{θ}\PY{p}{,}\PY{n}{p} \PY{o}{=} \PY{n}{set\PYZus{}variables}\PY{p}{(}\PY{k+kt}{Float64}\PY{p}{,}\PY{l+s}{\PYZdq{}}\PY{l+s}{θ}\PY{l+s}{ }\PY{l+s}{p}\PY{l+s}{\PYZdq{}}\PY{p}{,}\PY{n}{order}\PY{o}{=}\PY{l+m+mi}{3}\PY{p}{)}
\end{Verbatim}


\begin{Verbatim}[commandchars=\\\{\}]
{\color{outcolor}Out[{\color{outcolor}27}]:} 2-element Array\{TaylorSeries.TaylorN\{Float64\},1\}:
           1.0 θ + 𝒪(‖x‖⁴)
           1.0 p + 𝒪(‖x‖⁴)
\end{Verbatim}
            
    \begin{Verbatim}[commandchars=\\\{\}]
{\color{incolor}In [{\color{incolor}28}]:} \PY{c}{\PYZsh{}generamos las variables con las que evaluamos el mapeo, notemos que deben de tener orden mayor al orden que estamos }
         \PY{c}{\PYZsh{} calculando}
         \PY{n}{P\PYZus{}θ}\PY{o}{=}\PY{n}{Taylor1}\PY{p}{(}\PY{p}{[}\PY{l+m+mf}{0.}\PY{p}{,}\PY{n}{vec}\PY{p}{[}\PY{l+m+mi}{1}\PY{p}{]}\PY{p}{,}\PY{n}{θ2}\PY{p}{,}\PY{n}{θ}\PY{p}{]}\PY{p}{,}\PY{l+m+mi}{4}\PY{p}{)}
         \PY{n}{P\PYZus{}p}\PY{o}{=}\PY{n}{Taylor1}\PY{p}{(}\PY{p}{[}\PY{l+m+mf}{0.}\PY{p}{,}\PY{n}{vec}\PY{p}{[}\PY{l+m+mi}{2}\PY{p}{]}\PY{p}{,}\PY{n}{p2}\PY{p}{,}\PY{n}{p}\PY{p}{]}\PY{p}{,}\PY{l+m+mi}{4}\PY{p}{)}
\end{Verbatim}


\begin{Verbatim}[commandchars=\\\{\}]
{\color{outcolor}Out[{\color{outcolor}28}]:}  ( - 0.5831757547123116 + 𝒪(‖x‖¹)) t + ( 1.0 p + 𝒪(‖x‖⁴)) t³ + 𝒪(t⁵)
\end{Verbatim}
            
    \begin{Verbatim}[commandchars=\\\{\}]
{\color{incolor}In [{\color{incolor}29}]:} \PY{c}{\PYZsh{}aplicamos el mapeo }
         \PY{n}{TO}\PY{o}{=}\PY{n}{EstandarMap}\PY{p}{(}\PY{n}{P\PYZus{}θ}\PY{p}{,}\PY{n}{P\PYZus{}p}\PY{p}{,}\PY{n}{ke}\PY{p}{)}
\end{Verbatim}


\begin{Verbatim}[commandchars=\\\{\}]
{\color{outcolor}Out[{\color{outcolor}29}]:} 2-element Array\{TaylorSeries.Taylor1\{TaylorSeries.TaylorN\{Float64\}\},1\}:
                                ( - 1.3955217641908624 + 𝒪(‖x‖¹)) t + ( 1.0 θ + 1.0 p + 𝒪(‖x‖⁴)) t³ + 𝒪(t⁵)
           ( - 1.0018322839695704 + 𝒪(‖x‖¹)) t + ( 0.1358876056468051 + 0.3 θ + 1.3 p + 𝒪(‖x‖⁴)) t³ + 𝒪(t⁵)
\end{Verbatim}
            
    \begin{Verbatim}[commandchars=\\\{\}]
{\color{incolor}In [{\color{incolor}30}]:} \PY{c}{\PYZsh{}creamos un vector de lambdas que formará parte del lado derecho de la ecuación cohomo.}
         \PY{n}{vλ}\PY{o}{=}\PY{p}{[}\PY{l+m+mi}{0}\PY{p}{,}\PY{n}{λ}\PY{p}{,}\PY{n}{λ}\PY{o}{\PYZca{}}\PY{l+m+mi}{2}\PY{p}{,} \PY{n}{λ}\PY{o}{\PYZca{}}\PY{l+m+mi}{3}\PY{p}{]}
         \PY{n}{θλt}\PY{o}{=}\PY{n}{Taylor1}\PY{p}{(}\PY{p}{[}\PY{l+m+mf}{0.}\PY{p}{,}\PY{n}{vec}\PY{p}{[}\PY{l+m+mi}{1}\PY{p}{]}\PY{p}{,}\PY{n}{θ2}\PY{p}{,}\PY{n}{θ}\PY{p}{]}\PY{o}{.*}\PY{n}{vλ}\PY{p}{,}\PY{l+m+mi}{4}\PY{p}{)}
         \PY{n}{pλt}\PY{o}{=}\PY{n}{Taylor1}\PY{p}{(}\PY{p}{[}\PY{l+m+mf}{0.}\PY{p}{,}\PY{n}{vec}\PY{p}{[}\PY{l+m+mi}{2}\PY{p}{]}\PY{p}{,}\PY{n}{p2}\PY{p}{,}\PY{n}{p}\PY{p}{]}\PY{o}{.*}\PY{n}{vλ}\PY{p}{,}\PY{l+m+mi}{4}\PY{p}{)}
\end{Verbatim}


\begin{Verbatim}[commandchars=\\\{\}]
{\color{outcolor}Out[{\color{outcolor}30}]:}  ( - 1.0018322839695704 + 𝒪(‖x‖¹)) t + ( 5.069751680348318 p + 𝒪(‖x‖⁴)) t³ + 𝒪(t⁵)
\end{Verbatim}
            
    \begin{Verbatim}[commandchars=\\\{\}]
{\color{incolor}In [{\color{incolor}31}]:} \PY{c}{\PYZsh{}lo usaremos como vector pues el mapeo regresa un vector}
         \PY{n}{λvec}\PY{o}{=}\PY{p}{[}\PY{n}{θλt}\PY{p}{,}\PY{n}{pλt}\PY{p}{]}
\end{Verbatim}


\begin{Verbatim}[commandchars=\\\{\}]
{\color{outcolor}Out[{\color{outcolor}31}]:} 2-element Array\{TaylorSeries.Taylor1\{TaylorSeries.TaylorN\{Float64\}\},1\}:
           ( - 1.3955217641908624 + 𝒪(‖x‖¹)) t + ( 5.069751680348318 θ + 𝒪(‖x‖⁴)) t³ + 𝒪(t⁵)
           ( - 1.0018322839695704 + 𝒪(‖x‖¹)) t + ( 5.069751680348318 p + 𝒪(‖x‖⁴)) t³ + 𝒪(t⁵)
\end{Verbatim}
            
    \begin{Verbatim}[commandchars=\\\{\}]
{\color{incolor}In [{\color{incolor}32}]:} \PY{c}{\PYZsh{}Escribimos la ecuación cohomo}
         \PY{n}{Ecua}\PY{o}{=}\PY{n}{TO}\PY{o}{\PYZhy{}}\PY{n}{λvec}
\end{Verbatim}


\begin{Verbatim}[commandchars=\\\{\}]
{\color{outcolor}Out[{\color{outcolor}32}]:} 2-element Array\{TaylorSeries.Taylor1\{TaylorSeries.TaylorN\{Float64\}\},1\}:
                               ( - 4.069751680348318 θ + 1.0 p + 𝒪(‖x‖⁴)) t³ + 𝒪(t⁵)
           ( 0.1358876056468051 + 0.3 θ - 3.7697516803483184 p + 𝒪(‖x‖⁴)) t³ + 𝒪(t⁵)
\end{Verbatim}
            
    \begin{Verbatim}[commandchars=\\\{\}]
{\color{incolor}In [{\color{incolor}33}]:} \PY{c}{\PYZsh{}notamos de nuevo que sólo sobreviven los términos de orden 3 y que hay un pequeño error en el término 1 }
         \PY{n}{θ3}\PY{o}{=}\PY{n}{Ecua}\PY{p}{[}\PY{l+m+mi}{1}\PY{p}{]}\PY{o}{.}\PY{n}{coeffs}\PY{p}{[}\PY{l+m+mi}{4}\PY{p}{]}
         \PY{n}{p3}\PY{o}{=}\PY{n}{Ecua}\PY{p}{[}\PY{l+m+mi}{2}\PY{p}{]}\PY{o}{.}\PY{n}{coeffs}\PY{p}{[}\PY{l+m+mi}{4}\PY{p}{]}
\end{Verbatim}


\begin{Verbatim}[commandchars=\\\{\}]
{\color{outcolor}Out[{\color{outcolor}33}]:}  0.1358876056468051 + 0.3 θ - 3.7697516803483184 p + 𝒪(‖x‖⁴)
\end{Verbatim}
            
    \begin{Verbatim}[commandchars=\\\{\}]
{\color{incolor}In [{\color{incolor}34}]:} \PY{c}{\PYZsh{}Dado que en este momento comienzan a aparecer términos independientes en x, p necesitamos una forma de extraerlos}
         \PY{c}{\PYZsh{} primero estraemos los coeficientes de x, p como antes con el jacobiano. }
         \PY{n}{JTO}\PY{o}{=}\PY{n}{jacobian}\PY{p}{(}\PY{p}{[}\PY{n}{θ3}\PY{p}{,}\PY{n}{p3}\PY{p}{]}\PY{p}{)}
\end{Verbatim}


\begin{Verbatim}[commandchars=\\\{\}]
{\color{outcolor}Out[{\color{outcolor}34}]:} 2×2 Array\{Float64,2\}:
          -4.06975   1.0    
           0.3      -3.76975
\end{Verbatim}
            
    \begin{Verbatim}[commandchars=\\\{\}]
{\color{incolor}In [{\color{incolor}35}]:} \PY{c}{\PYZsh{}calculamos el determinante }
         \PY{n}{det}\PY{p}{(}\PY{n}{JTO}\PY{p}{)}
\end{Verbatim}


\begin{Verbatim}[commandchars=\\\{\}]
{\color{outcolor}Out[{\color{outcolor}35}]:} 15.041953235593466
\end{Verbatim}
            
    \begin{Verbatim}[commandchars=\\\{\}]
{\color{incolor}In [{\color{incolor}36}]:} \PY{c}{\PYZsh{}extraemos ahora los coeficientes independientes de x,p }
         \PY{n}{a}\PY{o}{=} \PY{n}{Ecua}\PY{p}{[}\PY{l+m+mi}{1}\PY{p}{]}\PY{o}{.}\PY{n}{coeffs}\PY{p}{[}\PY{l+m+mi}{4}\PY{p}{]}\PY{o}{.}\PY{n}{coeffs}\PY{p}{[}\PY{l+m+mi}{1}\PY{p}{]}\PY{o}{.}\PY{n}{coeffs}\PY{p}{[}\PY{l+m+mi}{1}\PY{p}{]}
         \PY{n}{b}\PY{o}{=} \PY{n}{Ecua}\PY{p}{[}\PY{l+m+mi}{2}\PY{p}{]}\PY{o}{.}\PY{n}{coeffs}\PY{p}{[}\PY{l+m+mi}{4}\PY{p}{]}\PY{o}{.}\PY{n}{coeffs}\PY{p}{[}\PY{l+m+mi}{1}\PY{p}{]}\PY{o}{.}\PY{n}{coeffs}\PY{p}{[}\PY{l+m+mi}{1}\PY{p}{]}
         \PY{n}{vecCoef}\PY{o}{=}\PY{p}{[}\PY{o}{\PYZhy{}}\PY{n}{a}\PY{p}{,}\PY{o}{\PYZhy{}}\PY{n}{b}\PY{p}{]}
\end{Verbatim}


\begin{Verbatim}[commandchars=\\\{\}]
{\color{outcolor}Out[{\color{outcolor}36}]:} 2-element Array\{Float64,1\}:
          -0.0     
          -0.135888
\end{Verbatim}
            
    Dado que el determinante de la matriz es distinto de cero y los términos
independientes son diferentes de cero entonces existe una solución no
trivial. Para resolver este sitema utilizamos que :
\[JTO [\theta, p]^{T}=[a,b]^{T}\] \[[\theta, p]^{T}=JTO^{-1}[a,b]^{T}\]

    \begin{Verbatim}[commandchars=\\\{\}]
{\color{incolor}In [{\color{incolor}37}]:} \PY{c}{\PYZsh{} entonces solo se trata de invertir el jacobiano y multiplicar con el vector del lado izquierdo}
         \PY{n}{T3}\PY{o}{=}\PY{n}{JTO} \PY{o}{\PYZbs{}} \PY{n}{vecCoef}
\end{Verbatim}


\begin{Verbatim}[commandchars=\\\{\}]
{\color{outcolor}Out[{\color{outcolor}37}]:} 2-element Array\{Float64,1\}:
          0.00903391
          0.0367658 
\end{Verbatim}
            
    \begin{Verbatim}[commandchars=\\\{\}]
{\color{incolor}In [{\color{incolor}38}]:} \PY{c}{\PYZsh{}los valores que obtuvimos son los nuevos coeficientes de orden 3}
         \PY{n}{θ3}\PY{o}{=}\PY{n}{T3}\PY{p}{[}\PY{l+m+mi}{1}\PY{p}{]}
         \PY{n}{p3}\PY{o}{=}\PY{n}{T3}\PY{p}{[}\PY{l+m+mi}{2}\PY{p}{]}
         \PY{n}{v3}\PY{o}{=}\PY{p}{[}\PY{n}{θ3}\PY{p}{,}\PY{n}{p3}\PY{p}{]}
\end{Verbatim}


\begin{Verbatim}[commandchars=\\\{\}]
{\color{outcolor}Out[{\color{outcolor}38}]:} 2-element Array\{Float64,1\}:
          0.00903391
          0.0367658 
\end{Verbatim}
            
    \begin{Verbatim}[commandchars=\\\{\}]
{\color{incolor}In [{\color{incolor}39}]:} \PY{c}{\PYZsh{}queremos revisar la estabilidad de las variables tx,x}
         \PY{n+nd}{@show}\PY{p}{(}\PY{n}{P\PYZus{}θ}\PY{p}{)}
         \PY{n+nd}{@show}\PY{p}{(}\PY{n}{typeof}\PY{p}{(}\PY{n}{P\PYZus{}θ}\PY{p}{)}\PY{p}{)}
         \PY{n+nd}{@show}\PY{p}{(}\PY{n}{TO}\PY{p}{[}\PY{l+m+mi}{1}\PY{p}{]}\PY{p}{)}
         \PY{n+nd}{@show}\PY{p}{(}\PY{n}{typeof}\PY{p}{(}\PY{n}{TO}\PY{p}{[}\PY{l+m+mi}{1}\PY{p}{]}\PY{p}{)}\PY{p}{)}
         \PY{n+nd}{@show}\PY{p}{(}\PY{n}{θλt}\PY{p}{)}
         \PY{n+nd}{@show}\PY{p}{(}\PY{n}{typeof}\PY{p}{(}\PY{n}{θλt}\PY{p}{)}\PY{p}{)}
         \PY{n+nd}{@show}\PY{p}{(}\PY{n}{Ecua}\PY{p}{[}\PY{l+m+mi}{1}\PY{p}{]}\PY{p}{)}
         \PY{n+nd}{@show}\PY{p}{(}\PY{n}{typeof}\PY{p}{(}\PY{n}{Ecua}\PY{p}{[}\PY{l+m+mi}{1}\PY{p}{]}\PY{p}{)}\PY{p}{)}
\end{Verbatim}


    \begin{Verbatim}[commandchars=\\\{\}]
P\_θ =  ( - 0.8123460094785507 + 𝒪(‖x‖¹)) t + ( 1.0 θ + 𝒪(‖x‖⁴)) t³ + 𝒪(t⁵)
typeof(P\_θ) = TaylorSeries.Taylor1\{TaylorSeries.TaylorN\{Float64\}\}
TO[1] =  ( - 1.3955217641908624 + 𝒪(‖x‖¹)) t + ( 1.0 θ + 1.0 p + 𝒪(‖x‖⁴)) t³ + 𝒪(t⁵)
typeof(TO[1]) = TaylorSeries.Taylor1\{TaylorSeries.TaylorN\{Float64\}\}
θλt =  ( - 1.3955217641908624 + 𝒪(‖x‖¹)) t + ( 5.069751680348318 θ + 𝒪(‖x‖⁴)) t³ + 𝒪(t⁵)
typeof(θλt) = TaylorSeries.Taylor1\{TaylorSeries.TaylorN\{Float64\}\}
Ecua[1] =  ( - 4.069751680348318 θ + 1.0 p + 𝒪(‖x‖⁴)) t³ + 𝒪(t⁵)
typeof(Ecua[1]) = TaylorSeries.Taylor1\{TaylorSeries.TaylorN\{Float64\}\}

    \end{Verbatim}

\begin{Verbatim}[commandchars=\\\{\}]
{\color{outcolor}Out[{\color{outcolor}39}]:} TaylorSeries.Taylor1\{TaylorSeries.TaylorN\{Float64\}\}
\end{Verbatim}
            
    \begin{verbatim}
                                                    ORDEN 4
\end{verbatim}

    \begin{Verbatim}[commandchars=\\\{\}]
{\color{incolor}In [{\color{incolor}40}]:} \PY{c}{\PYZsh{}creamos las variables de tipo TaylorN}
         \PY{n}{θ}\PY{p}{,}\PY{n}{p} \PY{o}{=} \PY{n}{set\PYZus{}variables}\PY{p}{(}\PY{k+kt}{Float64}\PY{p}{,}\PY{l+s}{\PYZdq{}}\PY{l+s}{θ}\PY{l+s}{ }\PY{l+s}{p}\PY{l+s}{\PYZdq{}}\PY{p}{,}\PY{n}{order}\PY{o}{=}\PY{l+m+mi}{1}\PY{p}{)}
\end{Verbatim}


\begin{Verbatim}[commandchars=\\\{\}]
{\color{outcolor}Out[{\color{outcolor}40}]:} 2-element Array\{TaylorSeries.TaylorN\{Float64\},1\}:
           1.0 θ + 𝒪(‖x‖²)
           1.0 p + 𝒪(‖x‖²)
\end{Verbatim}
            
    \begin{Verbatim}[commandchars=\\\{\}]
{\color{incolor}In [{\color{incolor}41}]:} \PY{c}{\PYZsh{}creamos las variables del mapeo que ya toman en cuenta los coeficientes que calculamos}
         \PY{n}{P\PYZus{}θ}\PY{o}{=}\PY{n}{Taylor1}\PY{p}{(}\PY{p}{[}\PY{l+m+mf}{0.}\PY{p}{,}\PY{n}{vec}\PY{p}{[}\PY{l+m+mi}{1}\PY{p}{]}\PY{p}{,}\PY{n}{θ2}\PY{p}{,}\PY{n}{θ3}\PY{p}{,}\PY{n}{θ}\PY{p}{]}\PY{p}{,}\PY{l+m+mi}{5}\PY{p}{)}
         \PY{n}{P\PYZus{}p}\PY{o}{=}\PY{n}{Taylor1}\PY{p}{(}\PY{p}{[}\PY{l+m+mf}{0.}\PY{p}{,}\PY{n}{vec}\PY{p}{[}\PY{l+m+mi}{2}\PY{p}{]}\PY{p}{,}\PY{n}{p2}\PY{p}{,}\PY{n}{p3}\PY{p}{,}\PY{n}{p}\PY{p}{]}\PY{p}{,}\PY{l+m+mi}{5}\PY{p}{)}
\end{Verbatim}


\begin{Verbatim}[commandchars=\\\{\}]
{\color{outcolor}Out[{\color{outcolor}41}]:}  ( - 0.5831757547123116 + 𝒪(‖x‖¹)) t + ( 0.036765757927698775 + 𝒪(‖x‖¹)) t³ + ( 1.0 p + 𝒪(‖x‖²)) t⁴ + 𝒪(t⁶)
\end{Verbatim}
            
    \begin{Verbatim}[commandchars=\\\{\}]
{\color{incolor}In [{\color{incolor}46}]:} \PY{c}{\PYZsh{}Le aplicamos el mapeo estándar}
         \PY{n}{CO}\PY{o}{=}\PY{n}{EstandarMap}\PY{p}{(}\PY{n}{P\PYZus{}θ}\PY{p}{,}\PY{n}{P\PYZus{}p}\PY{p}{,}\PY{n}{ke}\PY{p}{)}
\end{Verbatim}


\begin{Verbatim}[commandchars=\\\{\}]
{\color{outcolor}Out[{\color{outcolor}46}]:} 2-element Array\{TaylorSeries.Taylor1\{TaylorSeries.TaylorN\{Float64\}\},1\}:
                                                    ( - 1.3955217641908624 + 𝒪(‖x‖¹)) t + ( 0.04579966486242166 + 𝒪(‖x‖¹)) t³ + ( 1.0 θ + 1.0 p + 𝒪(‖x‖²)) t⁴ + 𝒪(t⁶)
           ( - 1.0018322839695704 + 𝒪(‖x‖¹)) t + ( 0.18639326303323034 + 𝒪(‖x‖¹)) t³ + ( 0.3 θ + 1.3 p + 𝒪(‖x‖²)) t⁴ + ( - 0.026611022998050755 + 𝒪(‖x‖²)) t⁵ + 𝒪(t⁶)
\end{Verbatim}
            
    \begin{Verbatim}[commandchars=\\\{\}]
{\color{incolor}In [{\color{incolor}49}]:} \PY{c}{\PYZsh{}creamos la parte derecha de la ecuación}
         \PY{n}{vλ}\PY{o}{=}\PY{p}{[}\PY{l+m+mf}{0.}\PY{p}{,} \PY{n}{λ}\PY{p}{,} \PY{n}{λ}\PY{o}{\PYZca{}}\PY{l+m+mi}{2}\PY{p}{,} \PY{n}{λ}\PY{o}{\PYZca{}}\PY{l+m+mi}{3}\PY{p}{,} \PY{n}{λ}\PY{o}{\PYZca{}}\PY{l+m+mi}{4}\PY{p}{]}
         \PY{n}{θλt}\PY{o}{=}\PY{n}{Taylor1}\PY{p}{(}\PY{p}{[}\PY{l+m+mf}{0.}\PY{p}{,}\PY{n}{vec}\PY{p}{[}\PY{l+m+mi}{1}\PY{p}{]}\PY{p}{,}\PY{n}{θ2}\PY{p}{,}\PY{n}{θ3}\PY{p}{,}\PY{n}{θ}\PY{p}{]}\PY{o}{.*}\PY{n}{vλ}\PY{p}{,}\PY{l+m+mi}{5}\PY{p}{)}
         \PY{n}{pλt}\PY{o}{=}\PY{n}{Taylor1}\PY{p}{(}\PY{p}{[}\PY{l+m+mf}{0.}\PY{p}{,}\PY{n}{vec}\PY{p}{[}\PY{l+m+mi}{2}\PY{p}{]}\PY{p}{,}\PY{n}{p2}\PY{p}{,}\PY{n}{p3}\PY{p}{,}\PY{n}{p}\PY{p}{]}\PY{o}{.*}\PY{n}{vλ}\PY{p}{,}\PY{l+m+mi}{5}\PY{p}{)}
         \PY{n}{print}\PY{p}{(}\PY{n}{xλt}\PY{p}{,}\PY{n}{pλt}\PY{p}{)}
\end{Verbatim}


    \begin{Verbatim}[commandchars=\\\{\}]
 ( - 1.3955217641908624 + 𝒪(‖x‖¹)) t + ( 0.04579966486242166 + 𝒪(‖x‖¹)) t³ + ( 8.709279945267069 θ + 𝒪(‖x‖²)) t⁴ + 𝒪(t⁶) ( - 1.0018322839695704 + 𝒪(‖x‖¹)) t + ( 0.18639326303323037 + 𝒪(‖x‖¹)) t³ + ( 8.709279945267069 p + 𝒪(‖x‖²)) t⁴ + 𝒪(t⁶)
    \end{Verbatim}

    \begin{Verbatim}[commandchars=\\\{\}]
{\color{incolor}In [{\color{incolor}50}]:} \PY{c}{\PYZsh{}es más útil tener esta parte como un vector}
         \PY{n}{λvec}\PY{o}{=}\PY{p}{[}\PY{n}{θλt}\PY{p}{,}\PY{n}{pλt}\PY{p}{]}
\end{Verbatim}


\begin{Verbatim}[commandchars=\\\{\}]
{\color{outcolor}Out[{\color{outcolor}50}]:} 2-element Array\{TaylorSeries.Taylor1\{TaylorSeries.TaylorN\{Float64\}\},1\}:
           ( - 1.3955217641908624 + 𝒪(‖x‖¹)) t + ( 0.04579966486242166 + 𝒪(‖x‖¹)) t³ + ( 8.709279945267069 θ + 𝒪(‖x‖²)) t⁴ + 𝒪(t⁶)
           ( - 1.0018322839695704 + 𝒪(‖x‖¹)) t + ( 0.18639326303323037 + 𝒪(‖x‖¹)) t³ + ( 8.709279945267069 p + 𝒪(‖x‖²)) t⁴ + 𝒪(t⁶)
\end{Verbatim}
            
    \begin{Verbatim}[commandchars=\\\{\}]
{\color{incolor}In [{\color{incolor}51}]:} \PY{c}{\PYZsh{}escribimos la ecuación cohomo, para poder sacar el jacobiano}
         \PY{n}{Ecua}\PY{o}{=}\PY{n}{CO}\PY{o}{\PYZhy{}}\PY{n}{λvec}
\end{Verbatim}


\begin{Verbatim}[commandchars=\\\{\}]
{\color{outcolor}Out[{\color{outcolor}51}]:} 2-element Array\{TaylorSeries.Taylor1\{TaylorSeries.TaylorN\{Float64\}\},1\}:
                                                                                             ( - 7.709279945267069 θ + 1.0 p + 𝒪(‖x‖²)) t⁴ + 𝒪(t⁶)
           ( - 2.7755575615628914e-17 + 𝒪(‖x‖¹)) t³ + ( 0.3 θ - 7.409279945267069 p + 𝒪(‖x‖²)) t⁴ + ( - 0.026611022998050755 + 𝒪(‖x‖²)) t⁵ + 𝒪(t⁶)
\end{Verbatim}
            
    \begin{Verbatim}[commandchars=\\\{\}]
{\color{incolor}In [{\color{incolor}52}]:} \PY{c}{\PYZsh{}obtenemos los valores de orden 4 }
         \PY{n}{θ4}\PY{o}{=}\PY{n}{Ecua}\PY{p}{[}\PY{l+m+mi}{1}\PY{p}{]}\PY{o}{.}\PY{n}{coeffs}\PY{p}{[}\PY{l+m+mi}{5}\PY{p}{]}
         \PY{n}{p4}\PY{o}{=}\PY{n}{Ecua}\PY{p}{[}\PY{l+m+mi}{2}\PY{p}{]}\PY{o}{.}\PY{n}{coeffs}\PY{p}{[}\PY{l+m+mi}{5}\PY{p}{]}
\end{Verbatim}


\begin{Verbatim}[commandchars=\\\{\}]
{\color{outcolor}Out[{\color{outcolor}52}]:}  0.3 θ - 7.409279945267069 p + 𝒪(‖x‖²)
\end{Verbatim}
            
    \begin{Verbatim}[commandchars=\\\{\}]
{\color{incolor}In [{\color{incolor}53}]:} \PY{c}{\PYZsh{}calculamos el jacobiano de estos términos, para obtener solo los coeficientes}
         \PY{n}{JCO}\PY{o}{=}\PY{n}{jacobian}\PY{p}{(}\PY{p}{[}\PY{n}{θ4}\PY{p}{,}\PY{n}{p4}\PY{p}{]}\PY{p}{)}
\end{Verbatim}


\begin{Verbatim}[commandchars=\\\{\}]
{\color{outcolor}Out[{\color{outcolor}53}]:} 2×2 Array\{Float64,2\}:
          -7.70928   1.0    
           0.3      -7.40928
\end{Verbatim}
            
    \begin{Verbatim}[commandchars=\\\{\}]
{\color{incolor}In [{\color{incolor}54}]:} \PY{c}{\PYZsh{}calculamos el determinante para saber si podemos resolver el sistema\PYZus{}}
         \PY{n}{det}\PY{p}{(}\PY{n}{JCO}\PY{p}{)}
\end{Verbatim}


\begin{Verbatim}[commandchars=\\\{\}]
{\color{outcolor}Out[{\color{outcolor}54}]:} 56.8202132909169
\end{Verbatim}
            
    \begin{Verbatim}[commandchars=\\\{\}]
{\color{incolor}In [{\color{incolor}55}]:} \PY{c}{\PYZsh{}dado que el determinante es distinto de cero y los valores independientes del sistema son cero}
         \PY{c}{\PYZsh{} la única solución es la trivial}
         \PY{n}{T4}\PY{o}{=}\PY{n}{JCO}\PY{o}{\PYZbs{}} \PY{p}{[}\PY{l+m+mf}{0.}\PY{p}{,}\PY{l+m+mf}{0.}\PY{p}{]}
\end{Verbatim}


\begin{Verbatim}[commandchars=\\\{\}]
{\color{outcolor}Out[{\color{outcolor}55}]:} 2-element Array\{Float64,1\}:
          -0.0
          -0.0
\end{Verbatim}
            
    \begin{Verbatim}[commandchars=\\\{\}]
{\color{incolor}In [{\color{incolor}57}]:} \PY{c}{\PYZsh{}esta solución son los nuevos coeficientes del polinomio}
         \PY{n}{θ4}\PY{o}{=}\PY{n}{T4}\PY{p}{[}\PY{l+m+mi}{1}\PY{p}{]}
         \PY{n}{p4}\PY{o}{=}\PY{n}{T4}\PY{p}{[}\PY{l+m+mi}{2}\PY{p}{]}
\end{Verbatim}


\begin{Verbatim}[commandchars=\\\{\}]
{\color{outcolor}Out[{\color{outcolor}57}]:} -0.0
\end{Verbatim}
            
    De nuevo dado que es un término de orden par, los coeficientes fueron
cero.

    \begin{Verbatim}[commandchars=\\\{\}]
{\color{incolor}In [{\color{incolor}58}]:} \PY{c}{\PYZsh{}queremos revisar la estabilidad de las variables tx,x}
         \PY{n+nd}{@show}\PY{p}{(}\PY{n}{P\PYZus{}θ}\PY{p}{)}
         \PY{n+nd}{@show}\PY{p}{(}\PY{n}{typeof}\PY{p}{(}\PY{n}{P\PYZus{}θ}\PY{p}{)}\PY{p}{)}
         \PY{n+nd}{@show}\PY{p}{(}\PY{n}{CO}\PY{p}{[}\PY{l+m+mi}{1}\PY{p}{]}\PY{p}{)}
         \PY{n+nd}{@show}\PY{p}{(}\PY{n}{typeof}\PY{p}{(}\PY{n}{CO}\PY{p}{[}\PY{l+m+mi}{1}\PY{p}{]}\PY{p}{)}\PY{p}{)}
         \PY{n+nd}{@show}\PY{p}{(}\PY{n}{θλt}\PY{p}{)}
         \PY{n+nd}{@show}\PY{p}{(}\PY{n}{typeof}\PY{p}{(}\PY{n}{θλt}\PY{p}{)}\PY{p}{)}
         \PY{n+nd}{@show}\PY{p}{(}\PY{n}{Ecua}\PY{p}{[}\PY{l+m+mi}{1}\PY{p}{]}\PY{p}{)}
         \PY{n+nd}{@show}\PY{p}{(}\PY{n}{typeof}\PY{p}{(}\PY{n}{Ecua}\PY{p}{[}\PY{l+m+mi}{1}\PY{p}{]}\PY{p}{)}\PY{p}{)}
\end{Verbatim}


    \begin{Verbatim}[commandchars=\\\{\}]
P\_θ =  ( - 0.8123460094785507 + 𝒪(‖x‖¹)) t + ( 0.009033906934722882 + 𝒪(‖x‖¹)) t³ + ( 1.0 θ + 𝒪(‖x‖²)) t⁴ + 𝒪(t⁶)
typeof(P\_θ) = TaylorSeries.Taylor1\{TaylorSeries.TaylorN\{Float64\}\}
CO[1] =  ( - 1.3955217641908624 + 𝒪(‖x‖¹)) t + ( 0.04579966486242166 + 𝒪(‖x‖¹)) t³ + ( 1.0 θ + 1.0 p + 𝒪(‖x‖²)) t⁴ + 𝒪(t⁶)
typeof(CO[1]) = TaylorSeries.Taylor1\{TaylorSeries.TaylorN\{Float64\}\}
θλt =  ( - 1.3955217641908624 + 𝒪(‖x‖¹)) t + ( 0.04579966486242166 + 𝒪(‖x‖¹)) t³ + ( 8.709279945267069 θ + 𝒪(‖x‖²)) t⁴ + 𝒪(t⁶)
typeof(θλt) = TaylorSeries.Taylor1\{TaylorSeries.TaylorN\{Float64\}\}
Ecua[1] =  ( - 7.709279945267069 θ + 1.0 p + 𝒪(‖x‖²)) t⁴ + 𝒪(t⁶)
typeof(Ecua[1]) = TaylorSeries.Taylor1\{TaylorSeries.TaylorN\{Float64\}\}

    \end{Verbatim}

\begin{Verbatim}[commandchars=\\\{\}]
{\color{outcolor}Out[{\color{outcolor}58}]:} TaylorSeries.Taylor1\{TaylorSeries.TaylorN\{Float64\}\}
\end{Verbatim}
            
    \begin{verbatim}
                                             QUINTO ORDEN
\end{verbatim}

    \begin{Verbatim}[commandchars=\\\{\}]
{\color{incolor}In [{\color{incolor}59}]:} \PY{c}{\PYZsh{}creamos las variables que representan a x,p de tipo TaylorN}
         \PY{n}{θ}\PY{p}{,}\PY{n}{p} \PY{o}{=} \PY{n}{set\PYZus{}variables}\PY{p}{(}\PY{k+kt}{Float64}\PY{p}{,}\PY{l+s}{\PYZdq{}}\PY{l+s}{θ}\PY{l+s}{ }\PY{l+s}{p}\PY{l+s}{\PYZdq{}}\PY{p}{,}\PY{n}{order}\PY{o}{=}\PY{l+m+mi}{1}\PY{p}{)}
\end{Verbatim}


\begin{Verbatim}[commandchars=\\\{\}]
{\color{outcolor}Out[{\color{outcolor}59}]:} 2-element Array\{TaylorSeries.TaylorN\{Float64\},1\}:
           1.0 θ + 𝒪(‖x‖²)
           1.0 p + 𝒪(‖x‖²)
\end{Verbatim}
            
    \begin{Verbatim}[commandchars=\\\{\}]
{\color{incolor}In [{\color{incolor}60}]:} \PY{c}{\PYZsh{}definimos nuestros nuevos polinomios tomando en cuenta todos los coeficientes ya calculados}
         \PY{n}{P\PYZus{}θ}\PY{o}{=}\PY{n}{Taylor1}\PY{p}{(}\PY{p}{[}\PY{l+m+mf}{0.}\PY{p}{,}\PY{n}{vec}\PY{p}{[}\PY{l+m+mi}{1}\PY{p}{]}\PY{p}{,}\PY{n}{θ2}\PY{p}{,}\PY{n}{θ3}\PY{p}{,}\PY{n}{θ4}\PY{p}{,}\PY{n}{θ}\PY{p}{]}\PY{p}{,}\PY{l+m+mi}{6}\PY{p}{)}
         \PY{n}{P\PYZus{}p}\PY{o}{=}\PY{n}{Taylor1}\PY{p}{(}\PY{p}{[}\PY{l+m+mf}{0.}\PY{p}{,}\PY{n}{vec}\PY{p}{[}\PY{l+m+mi}{2}\PY{p}{]}\PY{p}{,}\PY{n}{p2}\PY{p}{,}\PY{n}{p3}\PY{p}{,}\PY{n}{p4}\PY{p}{,}\PY{n}{p}\PY{p}{]}\PY{p}{,}\PY{l+m+mi}{6}\PY{p}{)}
\end{Verbatim}


\begin{Verbatim}[commandchars=\\\{\}]
{\color{outcolor}Out[{\color{outcolor}60}]:}  ( - 0.5831757547123116 + 𝒪(‖x‖¹)) t + ( 0.036765757927698775 + 𝒪(‖x‖¹)) t³ + ( 1.0 p + 𝒪(‖x‖²)) t⁵ + 𝒪(t⁷)
\end{Verbatim}
            
    \begin{Verbatim}[commandchars=\\\{\}]
{\color{incolor}In [{\color{incolor}62}]:} \PY{c}{\PYZsh{}aplicamos el mapeo}
         \PY{n}{QO}\PY{o}{=}\PY{n}{EstandarMap}\PY{p}{(}\PY{n}{P\PYZus{}θ}\PY{p}{,}\PY{n}{P\PYZus{}p}\PY{p}{,}\PY{n}{ke}\PY{p}{)}
\end{Verbatim}


\begin{Verbatim}[commandchars=\\\{\}]
{\color{outcolor}Out[{\color{outcolor}62}]:} 2-element Array\{TaylorSeries.Taylor1\{TaylorSeries.TaylorN\{Float64\}\},1\}:
                                    ( - 1.3955217641908624 + 𝒪(‖x‖¹)) t + ( 0.04579966486242166 + 𝒪(‖x‖¹)) t³ + ( 1.0 θ + 1.0 p + 𝒪(‖x‖²)) t⁵ + 𝒪(t⁷)
           ( - 1.0018322839695704 + 𝒪(‖x‖¹)) t + ( 0.18639326303323034 + 𝒪(‖x‖¹)) t³ + ( - 0.026611022998050755 + 0.3 θ + 1.3 p + 𝒪(‖x‖²)) t⁵ + 𝒪(t⁷)
\end{Verbatim}
            
    \begin{Verbatim}[commandchars=\\\{\}]
{\color{incolor}In [{\color{incolor}64}]:} \PY{c}{\PYZsh{}creamos la otra parte de la ecuación cohomo}
         \PY{n}{vλ}\PY{o}{=}\PY{p}{[}\PY{l+m+mf}{0.}\PY{p}{,}\PY{n}{λ}\PY{p}{,}\PY{n}{λ}\PY{o}{\PYZca{}}\PY{l+m+mi}{2}\PY{p}{,}\PY{n}{λ}\PY{o}{\PYZca{}}\PY{l+m+mi}{3}\PY{p}{,}\PY{n}{λ}\PY{o}{\PYZca{}}\PY{l+m+mi}{4}\PY{p}{,}\PY{n}{λ}\PY{o}{\PYZca{}}\PY{l+m+mi}{5}\PY{p}{]}
         \PY{n}{θλt}\PY{o}{=}\PY{n}{Taylor1}\PY{p}{(}\PY{p}{[}\PY{l+m+mf}{0.}\PY{p}{,}\PY{n}{vec}\PY{p}{[}\PY{l+m+mi}{1}\PY{p}{]}\PY{p}{,}\PY{n}{θ2}\PY{p}{,}\PY{n}{θ3}\PY{p}{,}\PY{n}{θ4}\PY{p}{,}\PY{n}{θ}\PY{p}{]}\PY{o}{.*}\PY{n}{vλ}\PY{p}{,}\PY{l+m+mi}{6}\PY{p}{)}
         \PY{n}{pλt}\PY{o}{=}\PY{n}{Taylor1}\PY{p}{(}\PY{p}{[}\PY{l+m+mf}{0.}\PY{p}{,}\PY{n}{vec}\PY{p}{[}\PY{l+m+mi}{2}\PY{p}{]}\PY{p}{,}\PY{n}{p2}\PY{p}{,}\PY{n}{p3}\PY{p}{,}\PY{n}{p4}\PY{p}{,}\PY{n}{p}\PY{p}{]}\PY{o}{.*}\PY{n}{vλ}\PY{p}{,}\PY{l+m+mi}{6}\PY{p}{)}
         \PY{n}{print}\PY{p}{(}\PY{n}{θλt}\PY{p}{,}\PY{n}{pλt}\PY{p}{)}
\end{Verbatim}


    \begin{Verbatim}[commandchars=\\\{\}]
 ( - 1.3955217641908624 + 𝒪(‖x‖¹)) t + ( 0.04579966486242166 + 𝒪(‖x‖¹)) t³ + ( 14.96159219376594 θ + 𝒪(‖x‖²)) t⁵ + 𝒪(t⁷) ( - 1.0018322839695704 + 𝒪(‖x‖¹)) t + ( 0.18639326303323037 + 𝒪(‖x‖¹)) t³ + ( 14.96159219376594 p + 𝒪(‖x‖²)) t⁵ + 𝒪(t⁷)
    \end{Verbatim}

    \begin{Verbatim}[commandchars=\\\{\}]
{\color{incolor}In [{\color{incolor}65}]:} \PY{c}{\PYZsh{}nos es más útil como vector}
         \PY{n}{λvec}\PY{o}{=}\PY{p}{[}\PY{n}{θλt}\PY{p}{,}\PY{n}{pλt}\PY{p}{]}
\end{Verbatim}


\begin{Verbatim}[commandchars=\\\{\}]
{\color{outcolor}Out[{\color{outcolor}65}]:} 2-element Array\{TaylorSeries.Taylor1\{TaylorSeries.TaylorN\{Float64\}\},1\}:
           ( - 1.3955217641908624 + 𝒪(‖x‖¹)) t + ( 0.04579966486242166 + 𝒪(‖x‖¹)) t³ + ( 14.96159219376594 θ + 𝒪(‖x‖²)) t⁵ + 𝒪(t⁷)
           ( - 1.0018322839695704 + 𝒪(‖x‖¹)) t + ( 0.18639326303323037 + 𝒪(‖x‖¹)) t³ + ( 14.96159219376594 p + 𝒪(‖x‖²)) t⁵ + 𝒪(t⁷)
\end{Verbatim}
            
    \begin{Verbatim}[commandchars=\\\{\}]
{\color{incolor}In [{\color{incolor}66}]:} \PY{c}{\PYZsh{}escribimos la ecuación cohomo}
         \PY{n}{Ecua}\PY{o}{=}\PY{n}{QO}\PY{o}{\PYZhy{}}\PY{n}{λvec}
\end{Verbatim}


\begin{Verbatim}[commandchars=\\\{\}]
{\color{outcolor}Out[{\color{outcolor}66}]:} 2-element Array\{TaylorSeries.Taylor1\{TaylorSeries.TaylorN\{Float64\}\},1\}:
                                                                              ( - 13.96159219376594 θ + 1.0 p + 𝒪(‖x‖²)) t⁵ + 𝒪(t⁷)
           ( - 2.7755575615628914e-17 + 𝒪(‖x‖¹)) t³ + ( - 0.026611022998050755 + 0.3 θ - 13.661592193765939 p + 𝒪(‖x‖²)) t⁵ + 𝒪(t⁷)
\end{Verbatim}
            
    \begin{Verbatim}[commandchars=\\\{\}]
{\color{incolor}In [{\color{incolor}67}]:} \PY{c}{\PYZsh{}obtenemos los coeficientes  que no son cero, es decir los de orden 5}
         \PY{n}{θ5}\PY{o}{=}\PY{n}{Ecua}\PY{p}{[}\PY{l+m+mi}{1}\PY{p}{]}\PY{o}{.}\PY{n}{coeffs}\PY{p}{[}\PY{l+m+mi}{6}\PY{p}{]}
         \PY{n}{p5}\PY{o}{=}\PY{n}{Ecua}\PY{p}{[}\PY{l+m+mi}{2}\PY{p}{]}\PY{o}{.}\PY{n}{coeffs}\PY{p}{[}\PY{l+m+mi}{6}\PY{p}{]}
\end{Verbatim}


\begin{Verbatim}[commandchars=\\\{\}]
{\color{outcolor}Out[{\color{outcolor}67}]:}  - 0.026611022998050755 + 0.3 θ - 13.661592193765939 p + 𝒪(‖x‖²)
\end{Verbatim}
            
    \begin{Verbatim}[commandchars=\\\{\}]
{\color{incolor}In [{\color{incolor}68}]:} \PY{c}{\PYZsh{}les calculamos el jacobiano}
         \PY{n}{JQO}\PY{o}{=}\PY{n}{jacobian}\PY{p}{(}\PY{p}{[}\PY{n}{θ5}\PY{p}{,}\PY{n}{p5}\PY{p}{]}\PY{p}{)}
\end{Verbatim}


\begin{Verbatim}[commandchars=\\\{\}]
{\color{outcolor}Out[{\color{outcolor}68}]:} 2×2 Array\{Float64,2\}:
          -13.9616    1.0   
            0.3     -13.6616
\end{Verbatim}
            
    \begin{Verbatim}[commandchars=\\\{\}]
{\color{incolor}In [{\color{incolor}69}]:} \PY{c}{\PYZsh{}calculamos el determinante para ver si podemos encontrar una solución}
         \PY{n}{det}\PY{p}{(}\PY{n}{JQO}\PY{p}{)}
\end{Verbatim}


\begin{Verbatim}[commandchars=\\\{\}]
{\color{outcolor}Out[{\color{outcolor}69}]:} 190.43757892689624
\end{Verbatim}
            
    \begin{Verbatim}[commandchars=\\\{\}]
{\color{incolor}In [{\color{incolor}70}]:} \PY{c}{\PYZsh{}Dado que el determinante es distinto de cero buscamos los términos independientes}
         \PY{n}{a}\PY{o}{=} \PY{n}{Ecua}\PY{p}{[}\PY{l+m+mi}{1}\PY{p}{]}\PY{o}{.}\PY{n}{coeffs}\PY{p}{[}\PY{l+m+mi}{6}\PY{p}{]}\PY{o}{.}\PY{n}{coeffs}\PY{p}{[}\PY{l+m+mi}{1}\PY{p}{]}\PY{o}{.}\PY{n}{coeffs}\PY{p}{[}\PY{l+m+mi}{1}\PY{p}{]}
         \PY{n}{b}\PY{o}{=} \PY{n}{Ecua}\PY{p}{[}\PY{l+m+mi}{2}\PY{p}{]}\PY{o}{.}\PY{n}{coeffs}\PY{p}{[}\PY{l+m+mi}{6}\PY{p}{]}\PY{o}{.}\PY{n}{coeffs}\PY{p}{[}\PY{l+m+mi}{1}\PY{p}{]}\PY{o}{.}\PY{n}{coeffs}\PY{p}{[}\PY{l+m+mi}{1}\PY{p}{]}
         \PY{n}{vecCoef}\PY{o}{=}\PY{p}{[}\PY{o}{\PYZhy{}}\PY{n}{a}\PY{p}{,}\PY{o}{\PYZhy{}}\PY{n}{b}\PY{p}{]}
\end{Verbatim}


\begin{Verbatim}[commandchars=\\\{\}]
{\color{outcolor}Out[{\color{outcolor}70}]:} 2-element Array\{Float64,1\}:
          -0.0     
           0.026611
\end{Verbatim}
            
    \begin{Verbatim}[commandchars=\\\{\}]
{\color{incolor}In [{\color{incolor}71}]:} \PY{c}{\PYZsh{}como los térmimos independientes son diferentes de cero, y el determinante de la matriz también }
         \PY{c}{\PYZsh{} podemos encontrar una solución }
         
         \PY{n}{T4}\PY{o}{=}\PY{n}{JQO} \PY{o}{\PYZbs{}} \PY{n}{vecCoef}
\end{Verbatim}


\begin{Verbatim}[commandchars=\\\{\}]
{\color{outcolor}Out[{\color{outcolor}71}]:} 2-element Array\{Float64,1\}:
          -0.000139736
          -0.00195094 
\end{Verbatim}
            
    \begin{Verbatim}[commandchars=\\\{\}]
{\color{incolor}In [{\color{incolor}72}]:} \PY{c}{\PYZsh{}estos són los nuevos coeficientes de orden 5}
         \PY{n}{θ5}\PY{o}{=}\PY{n}{T4}\PY{p}{[}\PY{l+m+mi}{1}\PY{p}{]}
         \PY{n}{p5}\PY{o}{=}\PY{n}{T4}\PY{p}{[}\PY{l+m+mi}{2}\PY{p}{]}
\end{Verbatim}


\begin{Verbatim}[commandchars=\\\{\}]
{\color{outcolor}Out[{\color{outcolor}72}]:} -0.0019509397937700751
\end{Verbatim}
            
    \begin{Verbatim}[commandchars=\\\{\}]
{\color{incolor}In [{\color{incolor}73}]:} \PY{c}{\PYZsh{}queremos revisar la estabilidad de las variables tx,x}
         \PY{n+nd}{@show}\PY{p}{(}\PY{n}{P\PYZus{}θ}\PY{p}{)}
         \PY{n+nd}{@show}\PY{p}{(}\PY{n}{typeof}\PY{p}{(}\PY{n}{P\PYZus{}θ}\PY{p}{)}\PY{p}{)}
         \PY{n+nd}{@show}\PY{p}{(}\PY{n}{QO}\PY{p}{[}\PY{l+m+mi}{1}\PY{p}{]}\PY{p}{)}
         \PY{n+nd}{@show}\PY{p}{(}\PY{n}{typeof}\PY{p}{(}\PY{n}{QO}\PY{p}{[}\PY{l+m+mi}{1}\PY{p}{]}\PY{p}{)}\PY{p}{)}
         \PY{n+nd}{@show}\PY{p}{(}\PY{n}{θλt}\PY{p}{)}
         \PY{n+nd}{@show}\PY{p}{(}\PY{n}{typeof}\PY{p}{(}\PY{n}{θλt}\PY{p}{)}\PY{p}{)}
         \PY{n+nd}{@show}\PY{p}{(}\PY{n}{Ecua}\PY{p}{[}\PY{l+m+mi}{1}\PY{p}{]}\PY{p}{)}
         \PY{n+nd}{@show}\PY{p}{(}\PY{n}{typeof}\PY{p}{(}\PY{n}{Ecua}\PY{p}{[}\PY{l+m+mi}{1}\PY{p}{]}\PY{p}{)}\PY{p}{)}
\end{Verbatim}


    \begin{Verbatim}[commandchars=\\\{\}]
P\_θ =  ( - 0.8123460094785507 + 𝒪(‖x‖¹)) t + ( 0.009033906934722882 + 𝒪(‖x‖¹)) t³ + ( 1.0 θ + 𝒪(‖x‖²)) t⁵ + 𝒪(t⁷)
typeof(P\_θ) = TaylorSeries.Taylor1\{TaylorSeries.TaylorN\{Float64\}\}
QO[1] =  ( - 1.3955217641908624 + 𝒪(‖x‖¹)) t + ( 0.04579966486242166 + 𝒪(‖x‖¹)) t³ + ( 1.0 θ + 1.0 p + 𝒪(‖x‖²)) t⁵ + 𝒪(t⁷)
typeof(QO[1]) = TaylorSeries.Taylor1\{TaylorSeries.TaylorN\{Float64\}\}
θλt =  ( - 1.3955217641908624 + 𝒪(‖x‖¹)) t + ( 0.04579966486242166 + 𝒪(‖x‖¹)) t³ + ( 14.96159219376594 θ + 𝒪(‖x‖²)) t⁵ + 𝒪(t⁷)
typeof(θλt) = TaylorSeries.Taylor1\{TaylorSeries.TaylorN\{Float64\}\}
Ecua[1] =  ( - 13.96159219376594 θ + 1.0 p + 𝒪(‖x‖²)) t⁵ + 𝒪(t⁷)
typeof(Ecua[1]) = TaylorSeries.Taylor1\{TaylorSeries.TaylorN\{Float64\}\}

    \end{Verbatim}

\begin{Verbatim}[commandchars=\\\{\}]
{\color{outcolor}Out[{\color{outcolor}73}]:} TaylorSeries.Taylor1\{TaylorSeries.TaylorN\{Float64\}\}
\end{Verbatim}
            
    \begin{verbatim}
                                                Sexto orden
\end{verbatim}

    \begin{Verbatim}[commandchars=\\\{\}]
{\color{incolor}In [{\color{incolor}74}]:} \PY{c}{\PYZsh{}creamos las variables que representan a x,p}
         \PY{n}{θ}\PY{p}{,}\PY{n}{p} \PY{o}{=} \PY{n}{set\PYZus{}variables}\PY{p}{(}\PY{k+kt}{Float64}\PY{p}{,}\PY{l+s}{\PYZdq{}}\PY{l+s}{θ}\PY{l+s}{ }\PY{l+s}{p}\PY{l+s}{\PYZdq{}}\PY{p}{,}\PY{n}{order}\PY{o}{=}\PY{l+m+mi}{1}\PY{p}{)}
\end{Verbatim}


\begin{Verbatim}[commandchars=\\\{\}]
{\color{outcolor}Out[{\color{outcolor}74}]:} 2-element Array\{TaylorSeries.TaylorN\{Float64\},1\}:
           1.0 θ + 𝒪(‖x‖²)
           1.0 p + 𝒪(‖x‖²)
\end{Verbatim}
            
    \begin{Verbatim}[commandchars=\\\{\}]
{\color{incolor}In [{\color{incolor}75}]:} \PY{c}{\PYZsh{}creamos las variables que meteremos en el mapeo, las cuales son polinomios que ya tienen hasta el quinto orden calculado}
         \PY{n}{P\PYZus{}θ}\PY{o}{=}\PY{n}{Taylor1}\PY{p}{(}\PY{p}{[}\PY{l+m+mf}{0.}\PY{p}{,}\PY{n}{vec}\PY{p}{[}\PY{l+m+mi}{1}\PY{p}{]}\PY{p}{,}\PY{n}{θ2}\PY{p}{,}\PY{n}{θ3}\PY{p}{,}\PY{n}{θ4}\PY{p}{,}\PY{n}{θ5}\PY{p}{,}\PY{n}{θ}\PY{p}{]}\PY{p}{,}\PY{l+m+mi}{7}\PY{p}{)}
         \PY{n}{P\PYZus{}p}\PY{o}{=}\PY{n}{Taylor1}\PY{p}{(}\PY{p}{[}\PY{l+m+mf}{0.}\PY{p}{,}\PY{n}{vec}\PY{p}{[}\PY{l+m+mi}{2}\PY{p}{]}\PY{p}{,}\PY{n}{p2}\PY{p}{,}\PY{n}{p3}\PY{p}{,}\PY{n}{p4}\PY{p}{,}\PY{n}{p5}\PY{p}{,}\PY{n}{p}\PY{p}{]}\PY{p}{,}\PY{l+m+mi}{7}\PY{p}{)}
\end{Verbatim}


\begin{Verbatim}[commandchars=\\\{\}]
{\color{outcolor}Out[{\color{outcolor}75}]:}  ( - 0.5831757547123116 + 𝒪(‖x‖¹)) t + ( 0.036765757927698775 + 𝒪(‖x‖¹)) t³ + ( - 0.0019509397937700751 + 𝒪(‖x‖¹)) t⁵ + ( 1.0 p + 𝒪(‖x‖²)) t⁶ + 𝒪(t⁸)
\end{Verbatim}
            
    \begin{Verbatim}[commandchars=\\\{\}]
{\color{incolor}In [{\color{incolor}76}]:} \PY{c}{\PYZsh{}les aplicamos el mapeo}
         \PY{n}{SO}\PY{o}{=}\PY{n}{EstandarMap}\PY{p}{(}\PY{n}{P\PYZus{}θ}\PY{p}{,}\PY{n}{P\PYZus{}p}\PY{p}{,}\PY{n}{ke}\PY{p}{)}
\end{Verbatim}


\begin{Verbatim}[commandchars=\\\{\}]
{\color{outcolor}Out[{\color{outcolor}76}]:} 2-element Array\{TaylorSeries.Taylor1\{TaylorSeries.TaylorN\{Float64\}\},1\}:
                                                  ( - 1.3955217641908624 + 𝒪(‖x‖¹)) t + ( 0.04579966486242166 + 𝒪(‖x‖¹)) t³ + ( - 0.0020906759905228492 + 𝒪(‖x‖¹)) t⁵ + ( 1.0 θ + 1.0 p + 𝒪(‖x‖²)) t⁶ + 𝒪(t⁸)
           ( - 1.0018322839695704 + 𝒪(‖x‖¹)) t + ( 0.18639326303323034 + 𝒪(‖x‖¹)) t³ + ( - 0.029189165588977682 + 𝒪(‖x‖¹)) t⁵ + ( 0.3 θ + 1.3 p + 𝒪(‖x‖²)) t⁶ + ( 0.0038346622037622514 + 𝒪(‖x‖²)) t⁷ + 𝒪(t⁸)
\end{Verbatim}
            
    \begin{Verbatim}[commandchars=\\\{\}]
{\color{incolor}In [{\color{incolor}77}]:} \PY{c}{\PYZsh{}creamos el otro lado de la ecuación cohomo}
         \PY{n}{vλ}\PY{o}{=}\PY{p}{[}\PY{l+m+mf}{0.}\PY{p}{,} \PY{n}{λ}\PY{p}{,} \PY{n}{λ}\PY{o}{\PYZca{}}\PY{l+m+mi}{2}\PY{p}{,}\PY{n}{λ}\PY{o}{\PYZca{}}\PY{l+m+mi}{3}\PY{p}{,} \PY{n}{λ}\PY{o}{\PYZca{}}\PY{l+m+mi}{4}\PY{p}{,} \PY{n}{λ}\PY{o}{\PYZca{}}\PY{l+m+mi}{5}\PY{p}{,} \PY{n}{λ}\PY{o}{\PYZca{}}\PY{l+m+mi}{6}\PY{p}{]}
         \PY{n}{θλt}\PY{o}{=}\PY{n}{Taylor1}\PY{p}{(}\PY{p}{[}\PY{l+m+mf}{0.}\PY{p}{,}\PY{n}{vec}\PY{p}{[}\PY{l+m+mi}{1}\PY{p}{]}\PY{p}{,}\PY{n}{θ2}\PY{p}{,}\PY{n}{θ3}\PY{p}{,}\PY{n}{θ4}\PY{p}{,}\PY{n}{θ5}\PY{p}{,}\PY{n}{θ}\PY{p}{]}\PY{o}{.*}\PY{n}{vλ}\PY{p}{,}\PY{l+m+mi}{7}\PY{p}{)}
         \PY{n}{pλt}\PY{o}{=}\PY{n}{Taylor1}\PY{p}{(}\PY{p}{[}\PY{l+m+mf}{0.}\PY{p}{,}\PY{n}{vec}\PY{p}{[}\PY{l+m+mi}{2}\PY{p}{]}\PY{p}{,}\PY{n}{p2}\PY{p}{,}\PY{n}{p3}\PY{p}{,}\PY{n}{p4}\PY{p}{,}\PY{n}{p5}\PY{p}{,}\PY{n}{p}\PY{p}{]}\PY{o}{.*}\PY{n}{vλ}\PY{p}{,}\PY{l+m+mi}{7}\PY{p}{)}
         \PY{n}{print}\PY{p}{(}\PY{n}{θλt}\PY{p}{,}\PY{n}{pλt}\PY{p}{)}
\end{Verbatim}


    \begin{Verbatim}[commandchars=\\\{\}]
 ( - 1.3955217641908624 + 𝒪(‖x‖¹)) t + ( 0.04579966486242166 + 𝒪(‖x‖¹)) t³ + ( - 0.0020906759905228492 + 𝒪(‖x‖¹)) t⁵ + ( 25.702382100394594 θ + 𝒪(‖x‖²)) t⁶ + 𝒪(t⁸) ( - 1.0018322839695704 + 𝒪(‖x‖¹)) t + ( 0.18639326303323037 + 𝒪(‖x‖¹)) t³ + ( - 0.02918916558897769 + 𝒪(‖x‖¹)) t⁵ + ( 25.702382100394594 p + 𝒪(‖x‖²)) t⁶ + 𝒪(t⁸)
    \end{Verbatim}

    \begin{Verbatim}[commandchars=\\\{\}]
{\color{incolor}In [{\color{incolor}78}]:} \PY{n}{λvec}\PY{o}{=}\PY{p}{[}\PY{n}{θλt}\PY{p}{,}\PY{n}{pλt}\PY{p}{]}
\end{Verbatim}


\begin{Verbatim}[commandchars=\\\{\}]
{\color{outcolor}Out[{\color{outcolor}78}]:} 2-element Array\{TaylorSeries.Taylor1\{TaylorSeries.TaylorN\{Float64\}\},1\}:
           ( - 1.3955217641908624 + 𝒪(‖x‖¹)) t + ( 0.04579966486242166 + 𝒪(‖x‖¹)) t³ + ( - 0.0020906759905228492 + 𝒪(‖x‖¹)) t⁵ + ( 25.702382100394594 θ + 𝒪(‖x‖²)) t⁶ + 𝒪(t⁸)
             ( - 1.0018322839695704 + 𝒪(‖x‖¹)) t + ( 0.18639326303323037 + 𝒪(‖x‖¹)) t³ + ( - 0.02918916558897769 + 𝒪(‖x‖¹)) t⁵ + ( 25.702382100394594 p + 𝒪(‖x‖²)) t⁶ + 𝒪(t⁸)
\end{Verbatim}
            
    \begin{Verbatim}[commandchars=\\\{\}]
{\color{incolor}In [{\color{incolor}79}]:} \PY{c}{\PYZsh{}escribimos la ecuación cohomo para obtener el sistema asociado a este orden}
         \PY{n}{Ecua}\PY{o}{=}\PY{n}{SO}\PY{o}{\PYZhy{}}\PY{n}{λvec}
\end{Verbatim}


\begin{Verbatim}[commandchars=\\\{\}]
{\color{outcolor}Out[{\color{outcolor}79}]:} 2-element Array\{TaylorSeries.Taylor1\{TaylorSeries.TaylorN\{Float64\}\},1\}:
                                                                                                                                    ( - 24.702382100394594 θ + 1.0 p + 𝒪(‖x‖²)) t⁶ + 𝒪(t⁸)
           ( - 2.7755575615628914e-17 + 𝒪(‖x‖¹)) t³ + ( 6.938893903907228e-18 + 𝒪(‖x‖¹)) t⁵ + ( 0.3 θ - 24.402382100394593 p + 𝒪(‖x‖²)) t⁶ + ( 0.0038346622037622514 + 𝒪(‖x‖²)) t⁷ + 𝒪(t⁸)
\end{Verbatim}
            
    \begin{Verbatim}[commandchars=\\\{\}]
{\color{incolor}In [{\color{incolor}80}]:} \PY{c}{\PYZsh{}extraemos solo los coeficientes que nos interesa calcular, en principio los demás términos deben ser cero pues ya los }
         \PY{c}{\PYZsh{} calculamos previamente}
         \PY{n}{θ6}\PY{o}{=}\PY{n}{Ecua}\PY{p}{[}\PY{l+m+mi}{1}\PY{p}{]}\PY{o}{.}\PY{n}{coeffs}\PY{p}{[}\PY{l+m+mi}{7}\PY{p}{]}
         \PY{n}{p6}\PY{o}{=}\PY{n}{Ecua}\PY{p}{[}\PY{l+m+mi}{2}\PY{p}{]}\PY{o}{.}\PY{n}{coeffs}\PY{p}{[}\PY{l+m+mi}{7}\PY{p}{]}
\end{Verbatim}


\begin{Verbatim}[commandchars=\\\{\}]
{\color{outcolor}Out[{\color{outcolor}80}]:}  0.3 θ - 24.402382100394593 p + 𝒪(‖x‖²)
\end{Verbatim}
            
    \begin{Verbatim}[commandchars=\\\{\}]
{\color{incolor}In [{\color{incolor}81}]:} \PY{c}{\PYZsh{}calculamos el jacobiano para obetener el sistema}
         \PY{n}{JSO}\PY{o}{=}\PY{n}{jacobian}\PY{p}{(}\PY{p}{[}\PY{n}{θ6}\PY{p}{,}\PY{n}{p6}\PY{p}{]}\PY{p}{)}
\end{Verbatim}


\begin{Verbatim}[commandchars=\\\{\}]
{\color{outcolor}Out[{\color{outcolor}81}]:} 2×2 Array\{Float64,2\}:
          -24.7024    1.0   
            0.3     -24.4024
\end{Verbatim}
            
    \begin{Verbatim}[commandchars=\\\{\}]
{\color{incolor}In [{\color{incolor}82}]:} \PY{c}{\PYZsh{}calculamos el determinante del jacobiano para saber si hay solución}
         \PY{n}{det}\PY{p}{(}\PY{n}{JSO}\PY{p}{)}
\end{Verbatim}


\begin{Verbatim}[commandchars=\\\{\}]
{\color{outcolor}Out[{\color{outcolor}82}]:} 602.4969668037768
\end{Verbatim}
            
    \begin{Verbatim}[commandchars=\\\{\}]
{\color{incolor}In [{\color{incolor}83}]:} \PY{c}{\PYZsh{}dado que el determinante del jacobiano es distinto de cero y los términos independientes son iguales a cero }
         \PY{c}{\PYZsh{} la solución a este sistema es la trivial}
         \PY{n}{T6}\PY{o}{=}\PY{n}{JSO}\PY{o}{\PYZbs{}} \PY{p}{[}\PY{l+m+mf}{0.}\PY{p}{,}\PY{l+m+mf}{0.}\PY{p}{]}
\end{Verbatim}


\begin{Verbatim}[commandchars=\\\{\}]
{\color{outcolor}Out[{\color{outcolor}83}]:} 2-element Array\{Float64,1\}:
          -0.0
          -0.0
\end{Verbatim}
            
    \begin{Verbatim}[commandchars=\\\{\}]
{\color{incolor}In [{\color{incolor}84}]:} \PY{c}{\PYZsh{}por tanto los coeficientes de orden 6 son cero}
         \PY{n}{θ6}\PY{o}{=}\PY{n}{T6}\PY{p}{[}\PY{l+m+mi}{1}\PY{p}{]}
         \PY{n}{p6}\PY{o}{=}\PY{n}{T6}\PY{p}{[}\PY{l+m+mi}{2}\PY{p}{]}
\end{Verbatim}


\begin{Verbatim}[commandchars=\\\{\}]
{\color{outcolor}Out[{\color{outcolor}84}]:} -0.0
\end{Verbatim}
            
    \begin{Verbatim}[commandchars=\\\{\}]
{\color{incolor}In [{\color{incolor}85}]:} \PY{c}{\PYZsh{}queremos revisar la estabilidad de las variables tx,x}
         \PY{n+nd}{@show}\PY{p}{(}\PY{n}{P\PYZus{}θ}\PY{p}{)}
         \PY{n+nd}{@show}\PY{p}{(}\PY{n}{typeof}\PY{p}{(}\PY{n}{P\PYZus{}θ}\PY{p}{)}\PY{p}{)}
         \PY{n+nd}{@show}\PY{p}{(}\PY{n}{SO}\PY{p}{[}\PY{l+m+mi}{1}\PY{p}{]}\PY{p}{)}
         \PY{n+nd}{@show}\PY{p}{(}\PY{n}{typeof}\PY{p}{(}\PY{n}{SO}\PY{p}{[}\PY{l+m+mi}{1}\PY{p}{]}\PY{p}{)}\PY{p}{)}
         \PY{n+nd}{@show}\PY{p}{(}\PY{n}{θλt}\PY{p}{)}
         \PY{n+nd}{@show}\PY{p}{(}\PY{n}{typeof}\PY{p}{(}\PY{n}{θλt}\PY{p}{)}\PY{p}{)}
         \PY{n+nd}{@show}\PY{p}{(}\PY{n}{Ecua}\PY{p}{[}\PY{l+m+mi}{1}\PY{p}{]}\PY{p}{)}
         \PY{n+nd}{@show}\PY{p}{(}\PY{n}{typeof}\PY{p}{(}\PY{n}{Ecua}\PY{p}{[}\PY{l+m+mi}{1}\PY{p}{]}\PY{p}{)}\PY{p}{)}
\end{Verbatim}


    \begin{Verbatim}[commandchars=\\\{\}]
P\_θ =  ( - 0.8123460094785507 + 𝒪(‖x‖¹)) t + ( 0.009033906934722882 + 𝒪(‖x‖¹)) t³ + ( - 0.0001397361967527743 + 𝒪(‖x‖¹)) t⁵ + ( 1.0 θ + 𝒪(‖x‖²)) t⁶ + 𝒪(t⁸)
typeof(P\_θ) = TaylorSeries.Taylor1\{TaylorSeries.TaylorN\{Float64\}\}
SO[1] =  ( - 1.3955217641908624 + 𝒪(‖x‖¹)) t + ( 0.04579966486242166 + 𝒪(‖x‖¹)) t³ + ( - 0.0020906759905228492 + 𝒪(‖x‖¹)) t⁵ + ( 1.0 θ + 1.0 p + 𝒪(‖x‖²)) t⁶ + 𝒪(t⁸)
typeof(SO[1]) = TaylorSeries.Taylor1\{TaylorSeries.TaylorN\{Float64\}\}
θλt =  ( - 1.3955217641908624 + 𝒪(‖x‖¹)) t + ( 0.04579966486242166 + 𝒪(‖x‖¹)) t³ + ( - 0.0020906759905228492 + 𝒪(‖x‖¹)) t⁵ + ( 25.702382100394594 θ + 𝒪(‖x‖²)) t⁶ + 𝒪(t⁸)
typeof(θλt) = TaylorSeries.Taylor1\{TaylorSeries.TaylorN\{Float64\}\}
Ecua[1] =  ( - 24.702382100394594 θ + 1.0 p + 𝒪(‖x‖²)) t⁶ + 𝒪(t⁸)
typeof(Ecua[1]) = TaylorSeries.Taylor1\{TaylorSeries.TaylorN\{Float64\}\}

    \end{Verbatim}

\begin{Verbatim}[commandchars=\\\{\}]
{\color{outcolor}Out[{\color{outcolor}85}]:} TaylorSeries.Taylor1\{TaylorSeries.TaylorN\{Float64\}\}
\end{Verbatim}
            
    \begin{verbatim}
                                            Séptimo orden
\end{verbatim}

    \begin{Verbatim}[commandchars=\\\{\}]
{\color{incolor}In [{\color{incolor}86}]:} \PY{c}{\PYZsh{}creamos las variables de x,p}
         \PY{n}{θ}\PY{p}{,}\PY{n}{p} \PY{o}{=} \PY{n}{set\PYZus{}variables}\PY{p}{(}\PY{k+kt}{Float64}\PY{p}{,}\PY{l+s}{\PYZdq{}}\PY{l+s}{θ}\PY{l+s}{ }\PY{l+s}{p}\PY{l+s}{\PYZdq{}}\PY{p}{,}\PY{n}{order}\PY{o}{=}\PY{l+m+mi}{1}\PY{p}{)}
\end{Verbatim}


\begin{Verbatim}[commandchars=\\\{\}]
{\color{outcolor}Out[{\color{outcolor}86}]:} 2-element Array\{TaylorSeries.TaylorN\{Float64\},1\}:
           1.0 θ + 𝒪(‖x‖²)
           1.0 p + 𝒪(‖x‖²)
\end{Verbatim}
            
    \begin{Verbatim}[commandchars=\\\{\}]
{\color{incolor}In [{\color{incolor}88}]:} \PY{c}{\PYZsh{}calculamos los polinomios asociados a la variedad que ya tienen todos los términos previos}
         \PY{n}{P\PYZus{}θ}\PY{o}{=}\PY{n}{Taylor1}\PY{p}{(}\PY{p}{[}\PY{l+m+mf}{0.}\PY{p}{,}\PY{n}{vec}\PY{p}{[}\PY{l+m+mi}{1}\PY{p}{]}\PY{p}{,}\PY{n}{θ2}\PY{p}{,}\PY{n}{θ3}\PY{p}{,}\PY{n}{θ4}\PY{p}{,}\PY{n}{θ5}\PY{p}{,}\PY{n}{θ6}\PY{p}{,}\PY{n}{θ}\PY{p}{]}\PY{p}{,}\PY{l+m+mi}{8}\PY{p}{)}
         \PY{n}{P\PYZus{}p}\PY{o}{=}\PY{n}{Taylor1}\PY{p}{(}\PY{p}{[}\PY{l+m+mf}{0.}\PY{p}{,}\PY{n}{vec}\PY{p}{[}\PY{l+m+mi}{2}\PY{p}{]}\PY{p}{,}\PY{n}{p2}\PY{p}{,}\PY{n}{p3}\PY{p}{,}\PY{n}{p4}\PY{p}{,}\PY{n}{p5}\PY{p}{,}\PY{n}{p6}\PY{p}{,}\PY{n}{p}\PY{p}{]}\PY{p}{,}\PY{l+m+mi}{8}\PY{p}{)}
\end{Verbatim}


\begin{Verbatim}[commandchars=\\\{\}]
{\color{outcolor}Out[{\color{outcolor}88}]:}  ( - 0.5831757547123116 + 𝒪(‖x‖¹)) t + ( 0.036765757927698775 + 𝒪(‖x‖¹)) t³ + ( - 0.0019509397937700751 + 𝒪(‖x‖¹)) t⁵ + ( 1.0 p + 𝒪(‖x‖²)) t⁷ + 𝒪(t⁹)
\end{Verbatim}
            
    \begin{Verbatim}[commandchars=\\\{\}]
{\color{incolor}In [{\color{incolor}89}]:} \PY{c}{\PYZsh{}les aplicamos el mapeo}
         \PY{n}{SEO}\PY{o}{=}\PY{n}{EstandarMap}\PY{p}{(}\PY{n}{P\PYZus{}θ}\PY{p}{,}\PY{n}{P\PYZus{}p}\PY{p}{,}\PY{n}{ke}\PY{p}{)}
\end{Verbatim}


\begin{Verbatim}[commandchars=\\\{\}]
{\color{outcolor}Out[{\color{outcolor}89}]:} 2-element Array\{TaylorSeries.Taylor1\{TaylorSeries.TaylorN\{Float64\}\},1\}:
                                  ( - 1.3955217641908624 + 𝒪(‖x‖¹)) t + ( 0.04579966486242166 + 𝒪(‖x‖¹)) t³ + ( - 0.0020906759905228492 + 𝒪(‖x‖¹)) t⁵ + ( 1.0 θ + 1.0 p + 𝒪(‖x‖²)) t⁷ + 𝒪(t⁹)
           ( - 1.0018322839695704 + 𝒪(‖x‖¹)) t + ( 0.18639326303323034 + 𝒪(‖x‖¹)) t³ + ( - 0.029189165588977682 + 𝒪(‖x‖¹)) t⁵ + ( 0.0038346622037622514 + 0.3 θ + 1.3 p + 𝒪(‖x‖²)) t⁷ + 𝒪(t⁹)
\end{Verbatim}
            
    \begin{Verbatim}[commandchars=\\\{\}]
{\color{incolor}In [{\color{incolor}90}]:} \PY{c}{\PYZsh{}calculamos el otro lado de la ecuación cohomo}
         \PY{n}{vλ}\PY{o}{=}\PY{p}{[}\PY{l+m+mf}{0.}\PY{p}{,} \PY{n}{λ}\PY{p}{,} \PY{n}{λ}\PY{o}{\PYZca{}}\PY{l+m+mi}{2}\PY{p}{,}\PY{n}{λ}\PY{o}{\PYZca{}}\PY{l+m+mi}{3}\PY{p}{,} \PY{n}{λ}\PY{o}{\PYZca{}}\PY{l+m+mi}{4}\PY{p}{,} \PY{n}{λ}\PY{o}{\PYZca{}}\PY{l+m+mi}{5}\PY{p}{,} \PY{n}{λ}\PY{o}{\PYZca{}}\PY{l+m+mi}{6}\PY{p}{,}\PY{n}{λ}\PY{o}{\PYZca{}}\PY{l+m+mi}{7}\PY{p}{]}
         \PY{n}{θλt}\PY{o}{=}\PY{n}{Taylor1}\PY{p}{(}\PY{p}{[}\PY{l+m+mi}{0}\PY{p}{,}\PY{n}{vec}\PY{p}{[}\PY{l+m+mi}{1}\PY{p}{]}\PY{p}{,}\PY{n}{θ2}\PY{p}{,}\PY{n}{θ3}\PY{p}{,}\PY{n}{θ4}\PY{p}{,}\PY{n}{θ5}\PY{p}{,}\PY{n}{θ6}\PY{p}{,}\PY{n}{θ}\PY{p}{]}\PY{o}{.*}\PY{n}{vλ}\PY{p}{,}\PY{l+m+mi}{8}\PY{p}{)}
         \PY{n}{pλt}\PY{o}{=}\PY{n}{Taylor1}\PY{p}{(}\PY{p}{[}\PY{l+m+mi}{0}\PY{p}{,}\PY{n}{vec}\PY{p}{[}\PY{l+m+mi}{2}\PY{p}{]}\PY{p}{,}\PY{n}{p2}\PY{p}{,}\PY{n}{p3}\PY{p}{,}\PY{n}{p4}\PY{p}{,}\PY{n}{p5}\PY{p}{,}\PY{n}{p6}\PY{p}{,}\PY{n}{p}\PY{p}{]}\PY{o}{.*}\PY{n}{vλ}\PY{p}{,}\PY{l+m+mi}{8}\PY{p}{)}
         \PY{n}{print}\PY{p}{(}\PY{n}{θλt}\PY{p}{,}\PY{n}{pλt}\PY{p}{)}
\end{Verbatim}


    \begin{Verbatim}[commandchars=\\\{\}]
 ( - 1.3955217641908624 + 𝒪(‖x‖¹)) t + ( 0.04579966486242166 + 𝒪(‖x‖¹)) t³ + ( - 0.0020906759905228492 + 𝒪(‖x‖¹)) t⁵ + ( 44.15388663714163 θ + 𝒪(‖x‖²)) t⁷ + 𝒪(t⁹) ( - 1.0018322839695704 + 𝒪(‖x‖¹)) t + ( 0.18639326303323037 + 𝒪(‖x‖¹)) t³ + ( - 0.02918916558897769 + 𝒪(‖x‖¹)) t⁵ + ( 44.15388663714163 p + 𝒪(‖x‖²)) t⁷ + 𝒪(t⁹)
    \end{Verbatim}

    \begin{Verbatim}[commandchars=\\\{\}]
{\color{incolor}In [{\color{incolor}91}]:} \PY{n}{λvec}\PY{o}{=}\PY{p}{[}\PY{n}{θλt}\PY{p}{,}\PY{n}{pλt}\PY{p}{]}
\end{Verbatim}


\begin{Verbatim}[commandchars=\\\{\}]
{\color{outcolor}Out[{\color{outcolor}91}]:} 2-element Array\{TaylorSeries.Taylor1\{TaylorSeries.TaylorN\{Float64\}\},1\}:
           ( - 1.3955217641908624 + 𝒪(‖x‖¹)) t + ( 0.04579966486242166 + 𝒪(‖x‖¹)) t³ + ( - 0.0020906759905228492 + 𝒪(‖x‖¹)) t⁵ + ( 44.15388663714163 θ + 𝒪(‖x‖²)) t⁷ + 𝒪(t⁹)
             ( - 1.0018322839695704 + 𝒪(‖x‖¹)) t + ( 0.18639326303323037 + 𝒪(‖x‖¹)) t³ + ( - 0.02918916558897769 + 𝒪(‖x‖¹)) t⁵ + ( 44.15388663714163 p + 𝒪(‖x‖²)) t⁷ + 𝒪(t⁹)
\end{Verbatim}
            
    \begin{Verbatim}[commandchars=\\\{\}]
{\color{incolor}In [{\color{incolor}92}]:} \PY{c}{\PYZsh{}escribimos la ecua. cohomo}
         \PY{n}{Ecua}\PY{o}{=}\PY{n}{SEO}\PY{o}{\PYZhy{}}\PY{n}{λvec}
\end{Verbatim}


\begin{Verbatim}[commandchars=\\\{\}]
{\color{outcolor}Out[{\color{outcolor}92}]:} 2-element Array\{TaylorSeries.Taylor1\{TaylorSeries.TaylorN\{Float64\}\},1\}:
                                                                                                                     ( - 43.15388663714163 θ + 1.0 p + 𝒪(‖x‖²)) t⁷ + 𝒪(t⁹)
           ( - 2.7755575615628914e-17 + 𝒪(‖x‖¹)) t³ + ( 6.938893903907228e-18 + 𝒪(‖x‖¹)) t⁵ + ( 0.0038346622037622514 + 0.3 θ - 42.853886637141635 p + 𝒪(‖x‖²)) t⁷ + 𝒪(t⁹)
\end{Verbatim}
            
    \begin{Verbatim}[commandchars=\\\{\}]
{\color{incolor}In [{\color{incolor}93}]:} \PY{c}{\PYZsh{}dado que los términos anteriores ya los calculamos solo extraemos los términos de orden 7 }
         \PY{n}{θ7}\PY{o}{=}\PY{n}{Ecua}\PY{p}{[}\PY{l+m+mi}{1}\PY{p}{]}\PY{o}{.}\PY{n}{coeffs}\PY{p}{[}\PY{l+m+mi}{8}\PY{p}{]}
         \PY{n}{p7}\PY{o}{=}\PY{n}{Ecua}\PY{p}{[}\PY{l+m+mi}{2}\PY{p}{]}\PY{o}{.}\PY{n}{coeffs}\PY{p}{[}\PY{l+m+mi}{8}\PY{p}{]}
\end{Verbatim}


\begin{Verbatim}[commandchars=\\\{\}]
{\color{outcolor}Out[{\color{outcolor}93}]:}  0.0038346622037622514 + 0.3 θ - 42.853886637141635 p + 𝒪(‖x‖²)
\end{Verbatim}
            
    \begin{Verbatim}[commandchars=\\\{\}]
{\color{incolor}In [{\color{incolor}94}]:} \PY{c}{\PYZsh{}los ponemos como un vector y le calculamos el jacobianos para obtener el sistema}
         \PY{n}{JSEO}\PY{o}{=}\PY{n}{jacobian}\PY{p}{(}\PY{p}{[}\PY{n}{θ7}\PY{p}{,}\PY{n}{p7}\PY{p}{]}\PY{p}{)}
\end{Verbatim}


\begin{Verbatim}[commandchars=\\\{\}]
{\color{outcolor}Out[{\color{outcolor}94}]:} 2×2 Array\{Float64,2\}:
          -43.1539    1.0   
            0.3     -42.8539
\end{Verbatim}
            
    \begin{Verbatim}[commandchars=\\\{\}]
{\color{incolor}In [{\color{incolor}95}]:} \PY{c}{\PYZsh{}calculamos el determinante del sistema para saber si hay solución}
         \PY{n}{det}\PY{p}{(}\PY{n}{JSEO}\PY{p}{)}
\end{Verbatim}


\begin{Verbatim}[commandchars=\\\{\}]
{\color{outcolor}Out[{\color{outcolor}95}]:} 1849.0117659001287
\end{Verbatim}
            
    \begin{Verbatim}[commandchars=\\\{\}]
{\color{incolor}In [{\color{incolor}96}]:} \PY{c}{\PYZsh{}extraemos los términos independienets de la ecuación cohomo}
         \PY{n}{a}\PY{o}{=} \PY{n}{Ecua}\PY{p}{[}\PY{l+m+mi}{1}\PY{p}{]}\PY{o}{.}\PY{n}{coeffs}\PY{p}{[}\PY{l+m+mi}{8}\PY{p}{]}\PY{o}{.}\PY{n}{coeffs}\PY{p}{[}\PY{l+m+mi}{1}\PY{p}{]}\PY{o}{.}\PY{n}{coeffs}\PY{p}{[}\PY{l+m+mi}{1}\PY{p}{]}
         \PY{n}{b}\PY{o}{=} \PY{n}{Ecua}\PY{p}{[}\PY{l+m+mi}{2}\PY{p}{]}\PY{o}{.}\PY{n}{coeffs}\PY{p}{[}\PY{l+m+mi}{8}\PY{p}{]}\PY{o}{.}\PY{n}{coeffs}\PY{p}{[}\PY{l+m+mi}{1}\PY{p}{]}\PY{o}{.}\PY{n}{coeffs}\PY{p}{[}\PY{l+m+mi}{1}\PY{p}{]}
         \PY{n}{vecCoef}\PY{o}{=}\PY{p}{[}\PY{o}{\PYZhy{}}\PY{n}{a}\PY{p}{,}\PY{o}{\PYZhy{}}\PY{n}{b}\PY{p}{]}
\end{Verbatim}


\begin{Verbatim}[commandchars=\\\{\}]
{\color{outcolor}Out[{\color{outcolor}96}]:} 2-element Array\{Float64,1\}:
          -0.0       
          -0.00383466
\end{Verbatim}
            
    \begin{Verbatim}[commandchars=\\\{\}]
{\color{incolor}In [{\color{incolor}97}]:} \PY{c}{\PYZsh{}dado que el determinante de la matriz es distinto de cero y los términos independientes también, podemos encontrar }
         \PY{c}{\PYZsh{} una solución distinta de la trivial}
         \PY{n}{T7}\PY{o}{=}\PY{n}{JSEO}\PY{o}{\PYZbs{}}\PY{n}{vecCoef}
\end{Verbatim}


\begin{Verbatim}[commandchars=\\\{\}]
{\color{outcolor}Out[{\color{outcolor}97}]:} 2-element Array\{Float64,1\}:
          2.0739e-6 
          8.94968e-5
\end{Verbatim}
            
    \begin{Verbatim}[commandchars=\\\{\}]
{\color{incolor}In [{\color{incolor}98}]:} \PY{c}{\PYZsh{}estos nuevos elementos son los coeficientes de orden 7 en el polinomio}
         \PY{n}{θ7}\PY{o}{=}\PY{n}{T7}\PY{p}{[}\PY{l+m+mi}{1}\PY{p}{]}
         \PY{n}{p7}\PY{o}{=}\PY{n}{T7}\PY{p}{[}\PY{l+m+mi}{2}\PY{p}{]}
\end{Verbatim}


\begin{Verbatim}[commandchars=\\\{\}]
{\color{outcolor}Out[{\color{outcolor}98}]:} 8.94967685358829e-5
\end{Verbatim}
            
    \begin{Verbatim}[commandchars=\\\{\}]
{\color{incolor}In [{\color{incolor}99}]:} \PY{c}{\PYZsh{}queremos revisar la estabilidad de las variables tx,x}
         \PY{n+nd}{@show}\PY{p}{(}\PY{n}{P\PYZus{}θ}\PY{p}{)}
         \PY{n+nd}{@show}\PY{p}{(}\PY{n}{typeof}\PY{p}{(}\PY{n}{P\PYZus{}θ}\PY{p}{)}\PY{p}{)}
         \PY{n+nd}{@show}\PY{p}{(}\PY{n}{SEO}\PY{p}{[}\PY{l+m+mi}{1}\PY{p}{]}\PY{p}{)}
         \PY{n+nd}{@show}\PY{p}{(}\PY{n}{typeof}\PY{p}{(}\PY{n}{SEO}\PY{p}{[}\PY{l+m+mi}{1}\PY{p}{]}\PY{p}{)}\PY{p}{)}
         \PY{n+nd}{@show}\PY{p}{(}\PY{n}{θλt}\PY{p}{)}
         \PY{n+nd}{@show}\PY{p}{(}\PY{n}{typeof}\PY{p}{(}\PY{n}{θλt}\PY{p}{)}\PY{p}{)}
         \PY{n+nd}{@show}\PY{p}{(}\PY{n}{Ecua}\PY{p}{[}\PY{l+m+mi}{1}\PY{p}{]}\PY{p}{)}
         \PY{n+nd}{@show}\PY{p}{(}\PY{n}{typeof}\PY{p}{(}\PY{n}{Ecua}\PY{p}{[}\PY{l+m+mi}{1}\PY{p}{]}\PY{p}{)}\PY{p}{)}
\end{Verbatim}


    \begin{Verbatim}[commandchars=\\\{\}]
P\_θ =  ( - 0.8123460094785507 + 𝒪(‖x‖¹)) t + ( 0.009033906934722882 + 𝒪(‖x‖¹)) t³ + ( - 0.0001397361967527743 + 𝒪(‖x‖¹)) t⁵ + ( 1.0 θ + 𝒪(‖x‖²)) t⁷ + 𝒪(t⁹)
typeof(P\_θ) = TaylorSeries.Taylor1\{TaylorSeries.TaylorN\{Float64\}\}
SEO[1] =  ( - 1.3955217641908624 + 𝒪(‖x‖¹)) t + ( 0.04579966486242166 + 𝒪(‖x‖¹)) t³ + ( - 0.0020906759905228492 + 𝒪(‖x‖¹)) t⁵ + ( 1.0 θ + 1.0 p + 𝒪(‖x‖²)) t⁷ + 𝒪(t⁹)
typeof(SEO[1]) = TaylorSeries.Taylor1\{TaylorSeries.TaylorN\{Float64\}\}
θλt =  ( - 1.3955217641908624 + 𝒪(‖x‖¹)) t + ( 0.04579966486242166 + 𝒪(‖x‖¹)) t³ + ( - 0.0020906759905228492 + 𝒪(‖x‖¹)) t⁵ + ( 44.15388663714163 θ + 𝒪(‖x‖²)) t⁷ + 𝒪(t⁹)
typeof(θλt) = TaylorSeries.Taylor1\{TaylorSeries.TaylorN\{Float64\}\}
Ecua[1] =  ( - 43.15388663714163 θ + 1.0 p + 𝒪(‖x‖²)) t⁷ + 𝒪(t⁹)
typeof(Ecua[1]) = TaylorSeries.Taylor1\{TaylorSeries.TaylorN\{Float64\}\}

    \end{Verbatim}

\begin{Verbatim}[commandchars=\\\{\}]
{\color{outcolor}Out[{\color{outcolor}99}]:} TaylorSeries.Taylor1\{TaylorSeries.TaylorN\{Float64\}\}
\end{Verbatim}
            
    \begin{verbatim}
                                        Resultados gráficos
\end{verbatim}

    Nos quedaremos solo con estos términos, ya que el fin de este notebook
era visualizar como funcionaba el método en específico para poder
implementarlo. Graficaremos la variedad que calculamos al igual que el
error cometido. Para ello primero debemos tomar el polinomio de grado 7
que calculamos y evaluarlo hasta cierto parámetro t y graficarlo sobre
el espacio fase del mapeo. Para el error recordemos que en el notebook 4
lo calculamos, dado que el error no forma parte del método en si, si no
que es un desarrollo aparte aquí solo anexaremos las funciones que ya
usamos en el notebook pasado.

    \begin{Verbatim}[commandchars=\\\{\}]
{\color{incolor}In [{\color{incolor}100}]:} \PY{c}{\PYZsh{}definimos dos polinomios con los coeficientes que calculamos.}
          \PY{n}{P\PYZus{}θ}\PY{o}{=}\PY{n}{Taylor1}\PY{p}{(}\PY{p}{[}\PY{l+m+mf}{0.}\PY{p}{,}\PY{n}{vec}\PY{p}{[}\PY{l+m+mi}{1}\PY{p}{]}\PY{p}{,}\PY{n}{θ2}\PY{p}{,}\PY{n}{θ3}\PY{p}{,}\PY{n}{θ4}\PY{p}{,}\PY{n}{θ5}\PY{p}{,}\PY{n}{θ6}\PY{p}{,}\PY{n}{θ7}\PY{p}{]}\PY{p}{,}\PY{l+m+mi}{7}\PY{p}{)}
          \PY{n}{P\PYZus{}p}\PY{o}{=}\PY{n}{Taylor1}\PY{p}{(}\PY{p}{[}\PY{l+m+mf}{0.}\PY{p}{,}\PY{n}{vec}\PY{p}{[}\PY{l+m+mi}{2}\PY{p}{]}\PY{p}{,}\PY{n}{p2}\PY{p}{,}\PY{n}{p3}\PY{p}{,}\PY{n}{p4}\PY{p}{,}\PY{n}{p5}\PY{p}{,}\PY{n}{p6}\PY{p}{,}\PY{n}{p7}\PY{p}{]}\PY{p}{,}\PY{l+m+mi}{7}\PY{p}{)}
\end{Verbatim}


\begin{Verbatim}[commandchars=\\\{\}]
{\color{outcolor}Out[{\color{outcolor}100}]:}  - 0.5831757547123116 t + 0.036765757927698775 t³ - 0.0019509397937700751 t⁵ + 8.94967685358829e-5 t⁷ + 𝒪(t⁸)
\end{Verbatim}
            
    \begin{Verbatim}[commandchars=\\\{\}]
{\color{incolor}In [{\color{incolor}101}]:} \PY{c}{\PYZsh{} los ponemos dentro un un vector solución}
          \PY{n}{E}\PY{o}{=}\PY{p}{[}\PY{o}{\PYZhy{}}\PY{n}{P\PYZus{}θ}\PY{p}{,}\PY{o}{\PYZhy{}}\PY{n}{P\PYZus{}p}\PY{p}{]}
\end{Verbatim}


\begin{Verbatim}[commandchars=\\\{\}]
{\color{outcolor}Out[{\color{outcolor}101}]:} 2-element Array\{TaylorSeries.Taylor1\{Float64\},1\}:
            0.8123460094785507 t - 0.009033906934722882 t³ + 0.0001397361967527743 t⁵ - 2.0738982165943525e-6 t⁷ + 𝒪(t⁸)
              0.5831757547123116 t - 0.036765757927698775 t³ + 0.0019509397937700751 t⁵ - 8.94967685358829e-5 t⁷ + 𝒪(t⁸)
\end{Verbatim}
            
    \begin{Verbatim}[commandchars=\\\{\}]
{\color{incolor}In [{\color{incolor}102}]:} \PY{c}{\PYZsh{}también necesitaremos el vector de lambas }
          \PY{n}{vλ}\PY{o}{=}\PY{p}{[}\PY{l+m+mf}{0.}\PY{p}{,} \PY{n}{λ}\PY{p}{,} \PY{n}{λ}\PY{o}{\PYZca{}}\PY{l+m+mi}{2}\PY{p}{,}\PY{n}{λ}\PY{o}{\PYZca{}}\PY{l+m+mi}{3}\PY{p}{,} \PY{n}{λ}\PY{o}{\PYZca{}}\PY{l+m+mi}{4}\PY{p}{,} \PY{n}{λ}\PY{o}{\PYZca{}}\PY{l+m+mi}{5}\PY{p}{,} \PY{n}{λ}\PY{o}{\PYZca{}}\PY{l+m+mi}{6}\PY{p}{,} \PY{n}{λ}\PY{o}{\PYZca{}}\PY{l+m+mi}{7}\PY{p}{]}
          \PY{n}{λθ\PYZus{}pol}\PY{o}{=}\PY{n}{Taylor1}\PY{p}{(}\PY{p}{[}\PY{l+m+mi}{0}\PY{p}{,}\PY{n}{vec}\PY{p}{[}\PY{l+m+mi}{1}\PY{p}{]}\PY{p}{,}\PY{n}{θ2}\PY{p}{,}\PY{n}{θ3}\PY{p}{,}\PY{n}{θ4}\PY{p}{,}\PY{n}{θ5}\PY{p}{,}\PY{n}{θ6}\PY{p}{,}\PY{n}{θ7}\PY{p}{]}\PY{o}{.*}\PY{n}{vλ}\PY{p}{,}\PY{l+m+mi}{7}\PY{p}{)}
          \PY{n}{λp\PYZus{}pol}\PY{o}{=}\PY{n}{Taylor1}\PY{p}{(}\PY{p}{[}\PY{l+m+mi}{0}\PY{p}{,}\PY{n}{vec}\PY{p}{[}\PY{l+m+mi}{2}\PY{p}{]}\PY{p}{,}\PY{n}{p2}\PY{p}{,}\PY{n}{p3}\PY{p}{,}\PY{n}{p4}\PY{p}{,}\PY{n}{p5}\PY{p}{,}\PY{n}{p6}\PY{p}{,}\PY{n}{p7}\PY{p}{]}\PY{o}{.*}\PY{n}{vλ}\PY{p}{,}\PY{l+m+mi}{7}\PY{p}{)}
\end{Verbatim}


\begin{Verbatim}[commandchars=\\\{\}]
{\color{outcolor}Out[{\color{outcolor}102}]:}  - 1.0018322839695704 t + 0.18639326303323037 t³ - 0.02918916558897769 t⁵ + 0.003951630172323878 t⁷ + 𝒪(t⁸)
\end{Verbatim}
            
    \begin{Verbatim}[commandchars=\\\{\}]
{\color{incolor}In [{\color{incolor}103}]:} \PY{n}{λvec}\PY{o}{=}\PY{p}{[}\PY{o}{\PYZhy{}}\PY{n}{λθ\PYZus{}pol}\PY{p}{,}\PY{o}{\PYZhy{}}\PY{n}{λp\PYZus{}pol}\PY{p}{]}
\end{Verbatim}


\begin{Verbatim}[commandchars=\\\{\}]
{\color{outcolor}Out[{\color{outcolor}103}]:} 2-element Array\{TaylorSeries.Taylor1\{Float64\},1\}:
            1.3955217641908624 t - 0.04579966486242166 t³ + 0.0020906759905228492 t⁵ - 9.157066675247724e-5 t⁷ + 𝒪(t⁸)
              1.0018322839695704 t - 0.18639326303323037 t³ + 0.02918916558897769 t⁵ - 0.003951630172323878 t⁷ + 𝒪(t⁸)
\end{Verbatim}
            
    Queremos observar como se ve el polinomio que obtuvimos hasta ahora.
Para ello vamos a graficarlo.

    \begin{Verbatim}[commandchars=\\\{\}]
{\color{incolor}In [{\color{incolor}104}]:} \PY{k}{using} \PY{n}{PyPlot}
\end{Verbatim}


    \begin{Verbatim}[commandchars=\\\{\}]
{\color{incolor}In [{\color{incolor}105}]:} \PY{c}{\PYZsh{}esta es una función para evaluar el los polinomios que calculamos}
          \PY{c}{\PYZsh{} recibe como entrada el tiempo que es el valor del parámetro t, el paso}
          \PY{c}{\PYZsh{} que define en pasos de cuanto queremos ir evaluando y claro los polinomios A,B}
          \PY{k}{function} \PY{n}{evaluar}\PY{p}{(}\PY{n}{Tiempo}\PY{p}{,} \PY{n}{paso}\PY{p}{,}\PY{n}{A}\PY{p}{,}\PY{n}{B}\PY{p}{)}
              \PY{n}{ValX}\PY{o}{=}\PY{k+kt}{Float64}\PY{p}{[}\PY{p}{]}
              
              \PY{n}{ValY}\PY{o}{=}\PY{k+kt}{Float64}\PY{p}{[}\PY{p}{]}
          
              \PY{k}{for} \PY{n}{t} \PY{o}{=} \PY{l+m+mi}{0}\PY{o}{:}\PY{n}{paso}\PY{o}{:}\PY{n}{Tiempo}
                  \PY{n}{x} \PY{o}{=} \PY{n}{mod2pi}\PY{p}{(}\PY{n}{evaluate}\PY{p}{(}\PY{n}{A}\PY{p}{,}\PY{n}{t}\PY{p}{)}\PY{p}{)}
                  \PY{n}{y} \PY{o}{=} \PY{n}{mod2pi}\PY{p}{(}\PY{n}{evaluate}\PY{p}{(}\PY{n}{B}\PY{p}{,}\PY{n}{t}\PY{p}{)}\PY{p}{)}
                  \PY{n}{push!}\PY{p}{(}\PY{n}{ValX}\PY{p}{,}\PY{n}{x}\PY{p}{)}
                  \PY{n}{push!}\PY{p}{(}\PY{n}{ValY}\PY{p}{,}\PY{n}{y}\PY{p}{)}
                  
              \PY{k}{end}
              \PY{n}{p} \PY{o}{=} \PY{n}{plot}\PY{p}{(}\PY{n}{ValX}\PY{p}{,}\PY{n}{ValY}\PY{p}{,}\PY{n}{color}\PY{o}{=}\PY{l+s}{\PYZdq{}}\PY{l+s}{r}\PY{l+s}{e}\PY{l+s}{d}\PY{l+s}{\PYZdq{}}\PY{p}{,}\PY{n}{linestyle}\PY{o}{=}\PY{l+s}{\PYZdq{}}\PY{l+s}{\PYZhy{}}\PY{l+s}{\PYZdq{}}\PY{p}{,}\PY{n}{marker}\PY{o}{=}\PY{l+s}{\PYZdq{}}\PY{l+s}{.}\PY{l+s}{\PYZdq{}}\PY{p}{)}
              
          \PY{k}{end}
\end{Verbatim}


\begin{Verbatim}[commandchars=\\\{\}]
{\color{outcolor}Out[{\color{outcolor}105}]:} evaluar (generic function with 1 method)
\end{Verbatim}
            
    \begin{Verbatim}[commandchars=\\\{\}]
{\color{incolor}In [{\color{incolor}106}]:} \PY{k}{function} \PY{n}{iterarMap}\PY{p}{(}\PY{n}{f}\PY{p}{,}\PY{n}{x\PYZus{}i}\PY{p}{,}\PY{n}{p\PYZus{}i}\PY{p}{,}\PY{n}{n}\PY{p}{,} \PY{n}{k}\PY{p}{)}   \PY{c}{\PYZsh{}Definimos una función para iterar el mapeo}
              
              \PY{n}{x} \PY{o}{=} \PY{n}{x\PYZus{}i}                      \PY{c}{\PYZsh{}    }
                                               \PY{c}{\PYZsh{}Damos condiciones iniciales}
              \PY{n}{y} \PY{o}{=} \PY{n}{p\PYZus{}i}                          \PY{c}{\PYZsh{}}
          
              \PY{n}{iteradosMX} \PY{o}{=} \PY{p}{[}\PY{n}{x\PYZus{}i}\PY{p}{]}
          
              \PY{n}{iteradosMY} \PY{o}{=} \PY{p}{[}\PY{n}{p\PYZus{}i}\PY{p}{]}               \PY{c}{\PYZsh{}Definimos dos listas que tendran los valores de cada par ordenado de theta y P, y agregamos las condiciones iniciales}
          
              \PY{k}{for} \PY{n}{i}\PY{o}{=}\PY{l+m+mi}{0}\PY{o}{:}\PY{n}{n}              \PY{c}{\PYZsh{}iniciamos un ciclo de iteraciones donde se calculan x\PYZus{}n, y\PYZus{}n y se agregan a lalista correspodiente            }
          
                  \PY{n}{F} \PY{o}{=} \PY{n}{f}\PY{p}{(}\PY{n}{x\PYZus{}i}\PY{p}{,}\PY{n}{p\PYZus{}i}\PY{p}{,} \PY{n}{k}\PY{p}{)}
                  
                  \PY{n}{push!}\PY{p}{(}\PY{n}{iteradosMX}\PY{p}{,}\PY{n}{F}\PY{p}{[}\PY{l+m+mi}{1}\PY{p}{]}\PY{p}{)}
                  
                  \PY{n}{push!}\PY{p}{(}\PY{n}{iteradosMY}\PY{p}{,}\PY{n}{F}\PY{p}{[}\PY{l+m+mi}{2}\PY{p}{]}\PY{p}{)}
                      
          
                  \PY{n}{x\PYZus{}i} \PY{o}{=} \PY{n}{F}\PY{p}{[}\PY{l+m+mi}{1}\PY{p}{]}
          
                  \PY{n}{p\PYZus{}i} \PY{o}{=} \PY{n}{F}\PY{p}{[}\PY{l+m+mi}{2}\PY{p}{]}
              \PY{k}{end}
          
              \PY{k}{return} \PY{n}{iteradosMX}\PY{p}{,} \PY{n}{iteradosMY}  \PY{c}{\PYZsh{}La funcion iterados regresa las listas que corresponden a la trayectoria del }
          \PY{k}{end}
\end{Verbatim}


\begin{Verbatim}[commandchars=\\\{\}]
{\color{outcolor}Out[{\color{outcolor}106}]:} iterarMap (generic function with 1 method)
\end{Verbatim}
            
    \begin{Verbatim}[commandchars=\\\{\}]
{\color{incolor}In [{\color{incolor}119}]:} \PY{c}{\PYZsh{}esta función llama la función que itera el mapeo para crear su gráfica.}
          \PY{k}{function} \PY{n}{graficarMap}\PY{p}{(}\PY{n}{k}\PY{p}{)}
              \PY{n}{n} \PY{o}{=} \PY{l+m+mi}{100}
              \PY{n}{s}\PY{o}{=}\PY{l+m+mi}{2}\PY{n+nb}{pi}\PY{o}{/}\PY{l+m+mi}{18}
          
              \PY{k}{for} \PY{n}{p\PYZus{}i}\PY{o}{=}\PY{l+m+mi}{0}\PY{o}{:}\PY{n}{s}\PY{o}{:}\PY{l+m+mi}{2}\PY{n+nb}{pi}
                  \PY{k}{for} \PY{n}{x\PYZus{}i} \PY{o}{=}\PY{l+m+mi}{0}\PY{o}{:}\PY{n}{s}\PY{o}{:}\PY{l+m+mi}{2}\PY{n+nb}{pi}
                      \PY{n}{t}\PY{p}{,}\PY{n}{p} \PY{o}{=} \PY{n}{iterarMap}\PY{p}{(}\PY{n}{EstandarMap}\PY{p}{,}\PY{n}{x\PYZus{}i}\PY{p}{,}\PY{n}{p\PYZus{}i}\PY{p}{,}\PY{n}{n}\PY{p}{,} \PY{n}{k}\PY{p}{)}
                      \PY{c}{\PYZsh{}p = scatter(p,t,marker=\PYZdq{}.\PYZdq{},s=0.1)}
                      \PY{n}{p} \PY{o}{=} \PY{n}{scatter}\PY{p}{(}\PY{n}{t}\PY{p}{,}\PY{n}{p}\PY{p}{,}\PY{n}{marker}\PY{o}{=}\PY{l+s}{\PYZdq{}}\PY{l+s}{.}\PY{l+s}{\PYZdq{}}\PY{p}{,}\PY{n}{s}\PY{o}{=}\PY{l+m+mf}{0.1}\PY{p}{)}
                  \PY{k}{end}
              \PY{k}{end}
              
          \PY{k}{end}
\end{Verbatim}


\begin{Verbatim}[commandchars=\\\{\}]
{\color{outcolor}Out[{\color{outcolor}119}]:} graficarMap (generic function with 1 method)
\end{Verbatim}
            
    \begin{Verbatim}[commandchars=\\\{\}]
{\color{incolor}In [{\color{incolor}120}]:} \PY{c}{\PYZsh{}esta función solo llama a las anteriores para que todo sea más compacto, y grafica tamto el mapeo como la variedad}
          \PY{k}{function} \PY{n}{Graficar}\PY{p}{(}\PY{n}{Tiempo}\PY{p}{,} \PY{n}{paso}\PY{p}{,} \PY{n}{k}\PY{p}{,}\PY{n}{A}\PY{p}{,}\PY{n}{B}\PY{p}{)}
              \PY{n}{graficarMap}\PY{p}{(}\PY{n}{k}\PY{p}{)}
              \PY{n}{evaluar}\PY{p}{(}\PY{n}{Tiempo}\PY{p}{,}\PY{n}{paso}\PY{p}{,}\PY{n}{A}\PY{p}{,}\PY{n}{B}\PY{p}{)}
          \PY{k}{end}
\end{Verbatim}


\begin{Verbatim}[commandchars=\\\{\}]
{\color{outcolor}Out[{\color{outcolor}120}]:} Graficar (generic function with 1 method)
\end{Verbatim}
            
    \begin{Verbatim}[commandchars=\\\{\}]
{\color{incolor}In [{\color{incolor}121}]:} \PY{k}{function} \PY{n}{PolinomioCohomo}\PY{p}{(}\PY{n}{Pol\PYZus{}vec}\PY{p}{,}\PY{n}{λvec}\PY{p}{,} \PY{n}{k}\PY{p}{)}
              \PY{n}{Map\PYZus{}vec}\PY{o}{=}\PY{n}{EstandarMap}\PY{p}{(}\PY{n}{Pol\PYZus{}vec}\PY{p}{[}\PY{l+m+mi}{1}\PY{p}{]}\PY{p}{,}\PY{n}{Pol\PYZus{}vec}\PY{p}{[}\PY{l+m+mi}{2}\PY{p}{]}\PY{p}{,}\PY{n}{k}\PY{p}{)}
              \PY{n}{Ec\PYZus{}Cohomo} \PY{o}{=} \PY{n}{Map\PYZus{}vec}\PY{o}{\PYZhy{}}\PY{n}{λvec}
              \PY{k}{return} \PY{n}{Ec\PYZus{}Cohomo}
          \PY{k}{end}
\end{Verbatim}


\begin{Verbatim}[commandchars=\\\{\}]
{\color{outcolor}Out[{\color{outcolor}121}]:} PolinomioCohomo (generic function with 1 method)
\end{Verbatim}
            
    \begin{Verbatim}[commandchars=\\\{\}]
{\color{incolor}In [{\color{incolor}122}]:} \PY{k}{function} \PY{n}{Err}\PY{p}{(}\PY{n}{Ec\PYZus{}Cohomo}\PY{p}{,}\PY{n}{Tiempo}\PY{p}{,}\PY{n}{paso}\PY{p}{)}
              \PY{k+kt}{Val}\PY{o}{=}\PY{k+kt}{Float64}\PY{p}{[}\PY{p}{]}
              \PY{n}{Tiem}\PY{o}{=}\PY{k+kt}{Float64}\PY{p}{[}\PY{p}{]}
              \PY{n}{E1}\PY{o}{=}\PY{n}{Ec\PYZus{}Cohomo}\PY{p}{[}\PY{l+m+mi}{1}\PY{p}{]}
              \PY{n}{E2}\PY{o}{=}\PY{n}{Ec\PYZus{}Cohomo}\PY{p}{[}\PY{l+m+mi}{2}\PY{p}{]}
              \PY{k}{for} \PY{n}{t} \PY{o}{=} \PY{l+m+mi}{0}\PY{o}{:}\PY{n}{paso}\PY{o}{:}\PY{n}{Tiempo}
                  \PY{n}{x} \PY{o}{=} \PY{n}{evaluate}\PY{p}{(}\PY{n}{E1}\PY{p}{,}\PY{n}{t}\PY{p}{)}
                  \PY{n}{y} \PY{o}{=} \PY{n}{evaluate}\PY{p}{(}\PY{n}{E2}\PY{p}{,}\PY{n}{t}\PY{p}{)}
                  \PY{n}{E}\PY{o}{=}\PY{p}{[}\PY{n}{x}\PY{p}{,}\PY{n}{y}\PY{p}{]}
                  
                  \PY{n}{norma} \PY{o}{=} \PY{n}{norm}\PY{p}{(}\PY{n}{E}\PY{p}{,}\PY{n+nb}{Inf}\PY{p}{)}
                  \PY{c}{\PYZsh{}print(norma)}
                  \PY{c}{\PYZsh{}print(\PYZdq{}\PYZbs{}n\PYZdq{})}
                  \PY{n}{push!}\PY{p}{(}\PY{k+kt}{Val}\PY{p}{,}\PY{n}{norma}\PY{p}{)}
                  \PY{n}{push!}\PY{p}{(}\PY{n}{Tiem}\PY{p}{,}\PY{n}{t}\PY{p}{)}
              
              \PY{k}{end}
              \PY{k}{return} \PY{n}{Tiem}\PY{p}{,}\PY{k+kt}{Val}
          \PY{k}{end}
\end{Verbatim}


\begin{Verbatim}[commandchars=\\\{\}]
{\color{outcolor}Out[{\color{outcolor}122}]:} Err (generic function with 1 method)
\end{Verbatim}
            
    \begin{Verbatim}[commandchars=\\\{\}]
{\color{incolor}In [{\color{incolor}123}]:} \PY{k}{function} \PY{n}{ErrorE}\PY{p}{(}\PY{n}{Pol\PYZus{}vec}\PY{p}{,}\PY{n}{λvec}\PY{p}{,}\PY{n}{k}\PY{p}{,}\PY{n}{T}\PY{p}{,}\PY{n}{paso}\PY{p}{)}
              \PY{n}{EcuaCohomo} \PY{o}{=} \PY{n}{PolinomioCohomo}\PY{p}{(}\PY{n}{Pol\PYZus{}vec}\PY{p}{,}\PY{n}{λvec}\PY{p}{,}\PY{n}{k}\PY{p}{)}
              \PY{n}{Tiempo}\PY{p}{,}\PY{n}{valor}\PY{o}{=}\PY{n}{Err}\PY{p}{(}\PY{n}{EcuaCohomo}\PY{p}{,}\PY{n}{T}\PY{p}{,}\PY{n}{paso}\PY{p}{)}
              \PY{k}{return} \PY{n}{Tiempo}\PY{p}{,} \PY{n}{valor}
          \PY{k}{end}
\end{Verbatim}


\begin{Verbatim}[commandchars=\\\{\}]
{\color{outcolor}Out[{\color{outcolor}123}]:} ErrorE (generic function with 1 method)
\end{Verbatim}
            
    \begin{Verbatim}[commandchars=\\\{\}]
{\color{incolor}In [{\color{incolor}124}]:} \PY{k}{function} \PY{n}{CalculoVariedad}\PY{p}{(}\PY{n}{k}\PY{p}{,}\PY{n}{Tiempo}\PY{p}{,}\PY{n}{paso}\PY{p}{,}\PY{n}{Pol\PYZus{}vec}\PY{p}{,}\PY{n}{λvec}\PY{p}{)}
              \PY{n}{B}\PY{o}{=}\PY{n}{Pol\PYZus{}vec}\PY{p}{[}\PY{l+m+mi}{2}\PY{p}{]}
              \PY{n}{A}\PY{o}{=}\PY{n}{Pol\PYZus{}vec}\PY{p}{[}\PY{l+m+mi}{1}\PY{p}{]}
              \PY{c}{\PYZsh{}Graficar(Tiempo,paso,k,B,A)}
              \PY{n}{Graficar}\PY{p}{(}\PY{n}{Tiempo}\PY{p}{,}\PY{n}{paso}\PY{p}{,}\PY{n}{k}\PY{p}{,}\PY{n}{A}\PY{p}{,}\PY{n}{B}\PY{p}{)}
              \PY{n}{x1}\PY{p}{,}\PY{n}{y1}\PY{o}{=}\PY{n}{ErrorE}\PY{p}{(}\PY{n}{Pol\PYZus{}vec}\PY{p}{,}\PY{n}{λvec}\PY{p}{,} \PY{n}{k}\PY{p}{,}\PY{n}{Tiempo}\PY{p}{,}\PY{n}{paso}\PY{p}{)}
              
          
              \PY{k}{return} \PY{n}{A}\PY{p}{,}\PY{n}{B}\PY{p}{,}\PY{n}{x1}\PY{p}{,}\PY{n}{y1}
          \PY{k}{end}
\end{Verbatim}


\begin{Verbatim}[commandchars=\\\{\}]
{\color{outcolor}Out[{\color{outcolor}124}]:} CalculoVariedad (generic function with 1 method)
\end{Verbatim}
            
    Tenemos todas las funciones resumidas en la función CalculoVariedad.
Dados los parámetros iniciales como son el punto fijo, la constante del
mapeo , el valor hasta el cual se quiere evaluar el parámetro y el paso
que debe ir tomando, calcula los polinomios y genera un gráfica tanto de
la variedad sobre el espacio fase como del error.

    \begin{Verbatim}[commandchars=\\\{\}]
{\color{incolor}In [{\color{incolor}125}]:} \PY{n}{VariedadX}\PY{p}{,} \PY{n}{VariedadP}\PY{p}{,}\PY{n}{ErrorT}\PY{p}{,}\PY{n}{ErrorX}\PY{o}{=}\PY{n}{CalculoVariedad}\PY{p}{(}\PY{n}{ke}\PY{p}{,}\PY{l+m+mf}{2.9}\PY{p}{,}\PY{l+m+mf}{0.125}\PY{p}{,}\PY{n}{E}\PY{p}{,}\PY{n}{λvec}\PY{p}{)}
\end{Verbatim}


    \begin{center}
    \adjustimage{max size={0.9\linewidth}{0.9\paperheight}}{output_128_0.png}
    \end{center}
    { \hspace*{\fill} \\}
    
\begin{Verbatim}[commandchars=\\\{\}]
{\color{outcolor}Out[{\color{outcolor}125}]:} ( 0.8123460094785507 t - 0.009033906934722882 t³ + 0.0001397361967527743 t⁵ - 2.0738982165943525e-6 t⁷ + 𝒪(t⁸),  0.5831757547123116 t - 0.036765757927698775 t³ + 0.0019509397937700751 t⁵ - 8.94967685358829e-5 t⁷ + 𝒪(t⁸), [0.0, 0.125, 0.25, 0.375, 0.5, 0.625, 0.75, 0.875, 1.0, 1.125  …  1.75, 1.875, 2.0, 2.125, 2.25, 2.375, 2.5, 2.625, 2.75, 2.875], [0.0, 5.44223e-20, 4.4051e-19, 1.51603e-18, 3.69306e-18, 7.47032e-18, 1.34718e-17, 2.24937e-17, 3.55618e-17, 5.40015e-17  …  3.06239e-16, 4.14429e-16, 5.55112e-16, 7.36713e-16, 9.69493e-16, 1.26586e-15, 1.6407e-15, 2.11179e-15, 2.7002e-15, 3.43073e-15])
\end{Verbatim}
            
    \begin{Verbatim}[commandchars=\\\{\}]
{\color{incolor}In [{\color{incolor}126}]:} \PY{n}{plot}\PY{p}{(}\PY{n}{ErrorT}\PY{p}{,}\PY{n}{ErrorX}\PY{p}{)}
\end{Verbatim}


    \begin{center}
    \adjustimage{max size={0.9\linewidth}{0.9\paperheight}}{output_129_0.png}
    \end{center}
    { \hspace*{\fill} \\}
    
\begin{Verbatim}[commandchars=\\\{\}]
{\color{outcolor}Out[{\color{outcolor}126}]:} 1-element Array\{PyCall.PyObject,1\}:
           PyObject <matplotlib.lines.Line2D object at 0x7fee509c27d0>
\end{Verbatim}
            
    Podemos notar que el error crece de manera casi exponencial, aunque el
orden del error para t=3.0 es menor a \(10^{-15}\)


    % Add a bibliography block to the postdoc
    
    
    
    \end{document}
